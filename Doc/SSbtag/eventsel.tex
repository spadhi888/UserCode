\section{Baseline Event Selection}
\label{sec:eventsel}

This analysis is based on the same-sign dilepton search documented in AN-2011/468~\cite{ssnote2011} and corresponds to an
integrated luminosity of \intLumi. 
In that study we searched for events with two isolated same-sign leptons
in association with 2 additional jets and \met. 
Here we re-use most of the baseline event selection as summarized below. 
In addition, we require at least 2 b-tagged jets using Simple Secondary Vertex High Efficiency 
Medium (SSVHEM) working point tagger.
This tagger relies on reconstructed secondary vertices
with at least two tracks and an IP significance of at least 1.74 and provides a b-jet tagging 
efficiency of about 60\% with a roughly 5\% (15\%) systematic uncertainty for jet $p_T<240 (>240)~\GeV$ 
and a tagging rate of light flavor jets in the 2--5\% range, increasing with the jet momentum~\cite{BTVPAS2011}. 


We thus discuss here only differences and briefly summarize the basic kinematics and triggers.
For more details, we refer to~\cite{ssnote2011}.

\begin{itemize}
	\item Events have to pass one of the dilepton triggers without an HT requirement.
	\item There should be at least two isolated same-sign leptons ($ee$, $e\mu$, and $\mu\mu$) with $|\eta| < 2.4$.
	\item We require both leptons to have $p_T>$20~\GeV.
	\item We tighten the isolation cut on the leptons to 0.1.
	\item At least two particle flow jets tagged using SSVHEM tagger with $p_T > 40$ GeV and $|\eta| < 2.4$
		corrected with L1FastL2L3 corrections.
	\item The selected jets must be separated from the leptons by $\Delta R > 0.4$ (any lepton with $\pt>20~\GeV$ 
		passing the ID and isolation selections).
	\item \met $> 30$ GeV.
	\item We remove dilepton events with invariant mass $M_{ll} < 8$ GeV.
	\item We veto events if a third lepton is satisfying the following:
	\begin{itemize}
		\item has $\pt>10~\GeV$;
		\item (an electron)  passes  $|\eta|<2.5$, and a loosened identification, 
			as the WP95 ID-only without any cut on $h/e$ in the endcaps;
		\item (for a muon) passes all identification requirements of the signal selection except 
			for the calorimeter veto requirements;
		\item has relative isolation $<0.2$; 
		\item makes an opposite-sign same-flavor pair with either of the hypothesis leptons such 
			that the pair has a mass within 15 GeV of the $Z$ mass.
	\end{itemize}
\end{itemize}


