\section{Results}
\label{sec:ssresults}

In absence of any significant deviation from the predicted background, we set 95\% CL. on the number of observed events. 
Two statistical methods have been used for the upper limit. 
Both methods assume the uncertainties on signal and background are un-correlated and use a log-normal distribution for error pdfs. 

The first method used to compute the upper limit is based on Bayesian statistics~\cite{bayesian}.
A posterior probability $p(r)$ is used as a function of the signal strength $r = \sigma/\sigma_{SM}$ 
assuming a uniform prior for $r$ integrating the nuisance parameters associated with the uncertainties.
The upper limit at 95\% confidence level is then determined by integrating $p(r)$ to determine $r'$, 
which satisfies $\int_{r'}^{\inf} p(r) dr = 0.05$.

We use the hybrid frequentist-bayesian $CLs$ approach~\cite{CLSxx} as the second method. 
Although the two statistical approaches are not equivalent, in this case we get similar results. 

\begin{itemize}
\item Upper limit using high-$p_T$ analysis at 95\% CL. with 24\% signal systematic error using Bayesian approach = 6.1  
\item Upper limit using high-$p_T$ analysis at 95\% CL. with 24\% signal systematic error using $CLs$ = 5.8  
\item Upper limit using low-$p_T$ analysis at 95\% CL. with 24\% signal systematic error using Bayesian approach = 4.8  
\item Upper limit using low-$p_T$ analysis at 95\% CL. with 24\% signal systematic error using $CLs$ = 4.6  
\end{itemize}

We use 6.1 and 4.8 events as the upper limit for the rest of this document for high- and low-$p_T$ analyses.
