\section{Search Regions}
\label{sec:regions}

As we will demonstrate later in this note, the 
expected count of events passing the baseline selection 
is quite small (order 10).  Thus, this selection could potentially 
be used as a ``search region'' for new physics.  In addition,
we define tighter search regions that can be used
to increase our sensitivity to various models.
We define these search regions by adding the following requirements
on top of those of the baseline selection.

% can 
% be used to test the background predictions with the best available
% stastistical precision as this is the largest sample.
% We increase the sensitivity of this analysis to the models selected in 
% Section~\ref{sec:intro} by considering events passing the following
% search region selections applied on top of the baseline selections.

\begin{itemize}
 \item Scenarios with leptons arising from $W$ decays naturally have \met~ from the accompanying neutrino.  To improve sensitivity to these signatures, a region with \met~ $>$ 30 \GeV~is introduced.
  \item In addition to the \met~ $> $ 30 \GeV~requirement, a ++ region, including only positively charged lepton pairs is considered.
  This region is appropriate for final states with same-sign top
quarks, {\it e.g.}, the $Z^\prime$ and MaxFV models.   
    This selection reduces the fake-lepton and charge misidentification backgrounds,
    while keeping essentially all the same-sign top signal, which is produced primarily from the $uu$ initial state,
    due to the available PDF luminosities.
  \item The following tighter \Ht\ and \met\ regions are defined to improve 
sensitivity to SUSY
    production scenarios. 
    As mentioned in Section~\ref{sec:intro}, 
 all of the SUSY scenarios that we consider explicitely
have four b-quarks, up to two hadronically decaying W bosons,
    and at least two neutrinos and two LSPs to make up for \Ht\ and \met,
    varying between the model points.
    When setting limits on a particular model, the search region with 
the best expected limit is to be used in every particular case.
  \begin{enumerate}
     \item Low-\Ht\ low-\met region: $\Ht>200~\GeV, \met>50~\GeV$.
     \item Low-\Ht\ high-\met region: $\Ht>200~\GeV, \met>120~\GeV$.
     \item High-\Ht\ low-\met region: $\Ht>320~\GeV, \met>50~\GeV$.
     \item High-\Ht\ high-\met region: $\Ht>320~\GeV, \met>120~\GeV$.
     \item Low-\Ht\ low-\met region: $\Ht>200~\GeV, \met>50~\GeV$ with 
$\geq 3$ btags (all other regions have $\geq 2$ btags -- The 
inclusion of a region with $\geq 3$ btags was suggested by the 
Florida group\cite{ufl2}).
  \end{enumerate}
We note that the exact choices of \met~ and \Ht\ are somewhat arbitrary.
They are similar to those made in 
the 2011 untagged same-sign analysis~\cite{sspaper2011}.
 \item A region with a tight \Ht~requirement, $\Ht > 320~\GeV$, and no \met~requirement.  This region is potentially sensitive to certain SUSY scenarios without R-parity conservation (RPV).  
\end{itemize}

