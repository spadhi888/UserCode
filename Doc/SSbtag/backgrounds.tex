\section{Background Contributions}
\label{sec:bkgds}

We are following the same strategy in estimating the background contributions as in
the pre-tagged sample analysis~\cite{ssnote2011}.
Contributions with genuine same-sign isolated lepton pairs are estimated from simulation,
while the contributions from leptons arising from jets (fakes) and from genuine opposite-sign pairs
with a lepton charge misreconstruction (charge flips) are measured in data using control samples.
The data-driven estimates are described in the next section.
In addition, as a reference, we are using all relevant available simulated samples to get a feeling of the expected yields
from simulation alone.
As will be shown later in Section~\ref{sec:yields}, contributions with genuine same-sign isolated dileptons
are comparable to those estimated from events with fake leptons, while the predictions from charge flips are
relatively low.
These findings are in fair agreement with direct estimates from simulation.


We use MC to estimate contributions from the following SM production processes with genuine same-sign isolated dileptons:
\begin{itemize}
\item $qqW^\pm W^\pm, WWW, WWZ, WZZ, ZZZ, WW\gamma, t\bar{t}W, t\bar{t}Z, t\bar{t}\gamma$ and double parton $W^\pm W^\pm$ with two real leptons in the final state.
\item $WZ$, $W\gamma^\star$ ($0.25~\GeV < m_{\gamma^\star}<12~\GeV$), and $ZZ$ with two real leptons in the final state.
\item $W\gamma$ with one real lepton and a photon conversion. 
This background is a priori not estimated by the fake rate method
because the photon is generally isolated. 
%In practice, this background is completely negligible.
\end{itemize}
Details on the samples used and the corresponding cross sections can be found in Ref.~\cite{ssnote2011}.
As in the pre-tagged sample analysis, we are assigning a 50\% uncertainty to the expected
number of events from these samples.