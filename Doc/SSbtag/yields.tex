\section{Event Yields and Background Estimation}
\label{sec:yields}

In the following Tables~\ref{tab:yieldBase_hpt} to~\ref{tab:yieldsnu_hpt} summarize the background estimates and compare them to the
observed counts of events in data for the baseline and search region selections
defined in Sections~\ref{sec:eventsel} and~\ref{sec:regions}.
In each table the expectations from the simulation alone are given in the upper part,
these overall are used to get a feeling of the expected contributions.
Of these, only backgrounds with actual final state same-sign leptons are used for the final result
as described in Section~\ref{sec:bkgds}.
The lower part of each table is the main result of the analysis
used for comparisons with data and setting constraints on various models.
These include the predictions for the fake-lepton and charge misidentification
bacgrounds derived as descibed in Section~\ref{sec:datadriven}.

Details on the systematics of the background predictions are given in the corresponding sections.
The final estimates on these uncertainties are rather simple:
\begin{itemize}
  \item the simulated backgrounds (from $V\gamma$ down to and including the tribosons) have a total
    uncertainty of 50\%;
  \item the predicted number of fakes has a 50\% systematics in all modes;
  \item the charge flip contribution has a 20\% systematics.
\end{itemize}

\newcommand{\ttdilS}{\ensuremath{\ttbar\to\ell\ell X}}
\newcommand{\ttslbS}{\ensuremath{\ttbar\to\ell(b\to\ell) X}}
\newcommand{\ttsloS}{\ensuremath{\ttbar\to\ell(b\!\!\!/\to\ell) X}}
\newcommand{\SFnn}{\ensuremath{N^{\rm Wj,raw}_{nn}}}

\begin{table}[h]
\begin{center}
\begin{tabular}{l | l l l l}
\hline\hline
 Source  &  ee  &  $\mu\mu$  &  e$\mu$  &  all \\
\hline
$t\overline{t}\rightarrow \ell\ell X$ &  0.469 $\pm$  0.305 &  0.000 $\pm$  0.199 &  1.133 $\pm$  0.580 &  1.603 $\pm$  0.656\\
$t\overline{t}$ other &  0.000 $\pm$  0.199 &  0.000 $\pm$  0.199 &  0.000 $\pm$  0.199 &  0.000 $\pm$  0.199\\
$t\overline{t}\rightarrow \ell(b\rightarrow \ell)X$ &  0.000 $\pm$  0.199 &  0.193 $\pm$  0.143 &  0.000 $\pm$  0.199 &  0.193 $\pm$  0.143\\
$t\overline{t}\rightarrow \ell(\slashed{b}\rightarrow \ell)X$ &  0.563 $\pm$  0.399 &  0.000 $\pm$  0.199 &  0.143 $\pm$  0.131 &  0.706 $\pm$  0.420\\
\hline
$t$, s-channel &  0.000 $\pm$  0.057 &  0.000 $\pm$  0.057 &  0.000 $\pm$  0.057 &  0.000 $\pm$  0.057\\
$t$, t-channel &  0.077 $\pm$  0.077 &  0.000 $\pm$  0.055 &  0.000 $\pm$  0.055 &  0.077 $\pm$  0.077\\
$tW$ &  0.000 $\pm$  0.045 &  0.000 $\pm$  0.045 &  0.016 $\pm$  0.045 &  0.016 $\pm$  0.045\\
\hline
$Z\rightarrow ee$ &  0.000 $\pm$  0.429 &  0.000 $\pm$  0.429 &  0.000 $\pm$  0.429 &  0.000 $\pm$  0.429\\
$Z\rightarrow\mu\mu$ &  0.000 $\pm$  0.429 &  0.000 $\pm$  0.429 &  0.000 $\pm$  0.429 &  0.000 $\pm$  0.429\\
$Z\rightarrow\tau\tau$ &  0.000 $\pm$  0.429 &  0.000 $\pm$  0.429 &  0.000 $\pm$  0.429 &  0.000 $\pm$  0.429\\
$W$+jets &  0.000 $\pm$  1.808 &  0.000 $\pm$  1.808 &  0.000 $\pm$  1.808 &  0.000 $\pm$  1.808\\
$WW$ &  0.000 $\pm$  0.019 &  0.000 $\pm$  0.019 &  0.000 $\pm$  0.019 &  0.000 $\pm$  0.019\\
\hline
V$\gamma$ &  0.000 $\pm$  0.248 &  0.000 $\pm$  0.248 &  0.000 $\pm$  0.248 &  0.000 $\pm$  0.248\\
$W\gamma^{*}\rightarrow\ell\nu e e$ &  0.000 $\pm$  0.097 &  0.000 $\pm$  0.097 &  0.000 $\pm$  0.097 &  0.000 $\pm$  0.097\\
$W\gamma^{*}\rightarrow\ell\nu\mu\mu$ &  0.000 $\pm$  0.075 &  0.000 $\pm$  0.075 &  0.000 $\pm$  0.075 &  0.000 $\pm$  0.075\\
$W\gamma^{*}\rightarrow\ell\nu\tau\tau$ &  0.000 $\pm$  0.028 &  0.000 $\pm$  0.028 &  0.000 $\pm$  0.028 &  0.000 $\pm$  0.028\\
$WZ$ &  0.034 $\pm$  0.012 &  0.017 $\pm$  0.008 &  0.031 $\pm$  0.012 &  0.082 $\pm$  0.019\\
$ZZ$ &   0.000 $\pm$   0.000 &  0.001 $\pm$  0.001 &  0.002 $\pm$  0.001 &  0.003 $\pm$  0.001\\
\hline
dp$W^{\pm}W^{\pm}$ &  0.000 $\pm$  0.004 &  0.000 $\pm$  0.004 &  0.000 $\pm$  0.004 &  0.000 $\pm$  0.004\\
sp$W^{-}W^{-}$ &  0.000 $\pm$  0.001 &  0.000 $\pm$  0.001 &  0.003 $\pm$  0.002 &  0.003 $\pm$  0.002\\
sp$W^{+}W^{+}$ &  0.000 $\pm$  0.006 &  0.000 $\pm$  0.006 &  0.000 $\pm$  0.006 &  0.000 $\pm$  0.006\\
$t\overline{t}\gamma$ &  0.000 $\pm$  0.059 &  0.000 $\pm$  0.059 &  0.000 $\pm$  0.059 &  0.000 $\pm$  0.059\\
$t\overline{t}W$ &  0.572 $\pm$  0.025 &  0.733 $\pm$  0.028 &  1.286 $\pm$  0.038 &  2.591 $\pm$  0.053\\
$t\overline{t}Z$ &  0.118 $\pm$  0.009 &  0.158 $\pm$  0.010 &  0.268 $\pm$  0.013 &  0.544 $\pm$  0.019\\
$WW\gamma$ &  0.000 $\pm$  0.015 &  0.000 $\pm$  0.015 &  0.000 $\pm$  0.015 &  0.000 $\pm$  0.015\\
$WWW$ &  0.001 $\pm$   0.000 &  0.001 $\pm$   0.000 &  0.001 $\pm$  0.001 &  0.003 $\pm$  0.001\\
$WWZ$ &   0.000 $\pm$   0.000 &   0.000 $\pm$   0.000 &  0.001 $\pm$  0.001 &  0.001 $\pm$  0.001\\
$WZZ$ &   0.000 $\pm$   0.000 &  0.001 $\pm$   0.000 &  0.001 $\pm$   0.000 &  0.002 $\pm$  0.001\\
$ZZZ$ &   0.000 $\pm$   0.000 &   0.000 $\pm$   0.000 &   0.000 $\pm$   0.000 &   0.000 $\pm$   0.000\\
\hline
Total MC &  1.834 $\pm$  0.509 &  1.103 $\pm$  0.146 &  2.885 $\pm$  0.597 &  5.822 $\pm$  0.797\\
\hline\hline
\hline
LM6 &  0.000 $\pm$  0.000 &  0.186 $\pm$  0.186 &  0.383 $\pm$  0.275 &  0.569 $\pm$  0.332\\
\hline\hline
\hline\hline
 SF  & 1.13 $\pm$ 0.67 & 0.30 $\pm$ 0.20 & 1.92 $\pm$ 0.74 & 3.36 $\pm$ 1.02\\
 DF  & 0.04 $\pm$ 0.12 & 0.02 $\pm$ 0.02 & 0.02 $\pm$ 0.09 & 0.08 $\pm$ 0.16\\
\hline
 SF + DF  & 1.17 $\pm$ 0.63 $\pm$ 0.58 & 0.32 $\pm$ 0.20 $\pm$ 0.16 & 1.95 $\pm$ 0.72 $\pm$ 0.97 & 3.43 $\pm$ 0.98 $\pm$ 1.72\\
\hline\hline
Charge Flips & 0.390 $\pm$ 0.032 $\pm$ 0.078 & - $\pm$ - & 0.544 $\pm$ 0.032 $\pm$ 0.109 & 0.934 $\pm$ 0.045 $\pm$ 0.187\\
\hline\hline
\hline
MC Pred &  0.725 $\pm$  0.029 $\pm$  0.362 &  0.912 $\pm$  0.031 $\pm$  0.456 &  1.595 $\pm$  0.042 $\pm$  0.797 &  3.231 $\pm$  0.059 $\pm$  1.616\\
\hline\hline
Total Pred &  2.281 $\pm$  0.633 $\pm$  0.691 &  1.232 $\pm$  0.199 $\pm$  0.483 &  4.086 $\pm$  0.725 $\pm$  1.263 &  7.600 $\pm$  0.983 $\pm$  2.365\\
\hline\hline
data & 2 & 2 & 3 & 7\\
\hline\hline
\end{tabular}

\end{center}
\caption{\label{tab:yieldBase_hpt}Observed event yields in baseline 
(\met $>$ 30 GeV, at least 2 jets with pT $>$ 40 GeV, and at least two of these jets b-tagged using SSVHEM) 
high-\pt\ (pT $>$ 20/20) dileptons
compared to expectations from simulation alone, and from the data-driven methods.
The upper part of the table is based on simulation only and is used only as a reference.
The lower part is the main result of the analysis.
The SF (DF) contributions are for events with one (two) fake leptons.
The {\em MC Pred} contribution includes contributions from genuine  same-sign lepton
pairs (a sum of the rows from $V\gamma$ down to $ZZZ$).
Entries with zero contributing events are reported with an uncertainty corresponding to one event.
This uncertainty is not added to the total MC contribution.
Systematic uncertainties (the second uncertainty if present)
 are displayed only for the final combined type of background, no systematic
uncertainty is added for estimates with zero entries.
Systematic uncertainties are 100\% correlated among the channels.
}
\end{table}

\clearpage

\begin{table}[hbt]
\begin{center}
\begin{tabular}{l | l l l l}
\hline\hline
 Source  &  ee  &  $\mu\mu$  &  e$\mu$  &  all \\
\hline
$t\overline{t}\rightarrow \ell\ell X$ &  0.069 $\pm$  0.199 &  0.000 $\pm$  0.199 &  0.000 $\pm$  0.199 &  0.069 $\pm$  0.199\\
$t\overline{t}$ other &  0.000 $\pm$  0.199 &  0.000 $\pm$  0.199 &  0.000 $\pm$  0.199 &  0.000 $\pm$  0.199\\
$t\overline{t}\rightarrow \ell(b\rightarrow \ell)X$ &  0.000 $\pm$  0.199 &  0.000 $\pm$  0.199 &  0.000 $\pm$  0.199 &  0.000 $\pm$  0.199\\
$t\overline{t}\rightarrow \ell(\slashed{b}\rightarrow \ell)X$ &  0.266 $\pm$  0.266 &  0.000 $\pm$  0.199 &  0.128 $\pm$  0.128 &  0.394 $\pm$  0.295\\
\hline
$t$, s-channel &  0.000 $\pm$  0.057 &  0.000 $\pm$  0.057 &  0.000 $\pm$  0.057 &  0.000 $\pm$  0.057\\
$t$, t-channel &  0.000 $\pm$  0.055 &  0.000 $\pm$  0.055 &  0.000 $\pm$  0.055 &  0.000 $\pm$  0.055\\
$tW$ &  0.000 $\pm$  0.045 &  0.000 $\pm$  0.045 &  0.000 $\pm$  0.045 &  0.000 $\pm$  0.045\\
\hline
$Z\rightarrow ee$ &  0.000 $\pm$  0.429 &  0.000 $\pm$  0.429 &  0.000 $\pm$  0.429 &  0.000 $\pm$  0.429\\
$Z\rightarrow\mu\mu$ &  0.000 $\pm$  0.429 &  0.000 $\pm$  0.429 &  0.000 $\pm$  0.429 &  0.000 $\pm$  0.429\\
$Z\rightarrow\tau\tau$ &  0.000 $\pm$  0.429 &  0.000 $\pm$  0.429 &  0.000 $\pm$  0.429 &  0.000 $\pm$  0.429\\
$W$+jets &  0.000 $\pm$  1.808 &  0.000 $\pm$  1.808 &  0.000 $\pm$  1.808 &  0.000 $\pm$  1.808\\
$WW$ &  0.000 $\pm$  0.019 &  0.000 $\pm$  0.019 &  0.000 $\pm$  0.019 &  0.000 $\pm$  0.019\\
\hline
V$\gamma$ &  0.000 $\pm$  0.248 &  0.000 $\pm$  0.248 &  0.000 $\pm$  0.248 &  0.000 $\pm$  0.248\\
$W\gamma^{*}\rightarrow\ell\nu e e$ &  0.000 $\pm$  0.097 &  0.000 $\pm$  0.097 &  0.000 $\pm$  0.097 &  0.000 $\pm$  0.097\\
$W\gamma^{*}\rightarrow\ell\nu\mu\mu$ &  0.000 $\pm$  0.075 &  0.000 $\pm$  0.075 &  0.000 $\pm$  0.075 &  0.000 $\pm$  0.075\\
$W\gamma^{*}\rightarrow\ell\nu\tau\tau$ &  0.000 $\pm$  0.028 &  0.000 $\pm$  0.028 &  0.000 $\pm$  0.028 &  0.000 $\pm$  0.028\\
$WZ$ &  0.009 $\pm$  0.006 &  0.001 $\pm$  0.003 &  0.006 $\pm$  0.004 &  0.016 $\pm$  0.008\\
$ZZ$ &  0.000 $\pm$   0.000 &  0.000 $\pm$   0.000 &   0.000 $\pm$   0.000 &   0.000 $\pm$   0.000\\
\hline
dp$W^{\pm}W^{\pm}$ &  0.000 $\pm$  0.004 &  0.000 $\pm$  0.004 &  0.000 $\pm$  0.004 &  0.000 $\pm$  0.004\\
sp$W^{-}W^{-}$ &  0.000 $\pm$  0.001 &  0.000 $\pm$  0.001 &  0.001 $\pm$  0.001 &  0.001 $\pm$  0.001\\
sp$W^{+}W^{+}$ &  0.000 $\pm$  0.006 &  0.000 $\pm$  0.006 &  0.000 $\pm$  0.006 &  0.000 $\pm$  0.006\\
$t\overline{t}\gamma$ &  0.000 $\pm$  0.059 &  0.000 $\pm$  0.059 &  0.000 $\pm$  0.059 &  0.000 $\pm$  0.059\\
$t\overline{t}W$ &  0.315 $\pm$  0.020 &  0.363 $\pm$  0.021 &  0.643 $\pm$  0.028 &  1.321 $\pm$  0.040\\
$t\overline{t}Z$ &  0.061 $\pm$  0.007 &  0.086 $\pm$  0.008 &  0.167 $\pm$  0.012 &  0.314 $\pm$  0.016\\
$WW\gamma$ &  0.000 $\pm$  0.015 &  0.000 $\pm$  0.015 &  0.000 $\pm$  0.015 &  0.000 $\pm$  0.015\\
$WWW$ &   0.000 $\pm$   0.000 &   0.000 $\pm$   0.000 &  0.001 $\pm$  0.001 &  0.002 $\pm$  0.001\\
$WWZ$ &  0.000 $\pm$   0.000 &  0.000 $\pm$   0.000 &  0.000 $\pm$   0.000 &  0.000 $\pm$   0.000\\
$WZZ$ &   0.000 $\pm$   0.000 &  0.000 $\pm$   0.000 &   0.000 $\pm$   0.000 &   0.000 $\pm$   0.000\\
$ZZZ$ &  0.000 $\pm$   0.000 &   0.000 $\pm$   0.000 &   0.000 $\pm$   0.000 &   0.000 $\pm$   0.000\\
\hline
Total MC &  0.721 $\pm$  0.276 &  0.451 $\pm$  0.022 &  0.946 $\pm$  0.131 &  2.118 $\pm$  0.306\\
\hline\hline
\hline
LM6 &  0.000 $\pm$  0.000 &  0.000 $\pm$  0.000 &  0.000 $\pm$  0.000 &  0.000 $\pm$  0.000\\
\hline\hline
\hline\hline
 SF  & 0.27 $\pm$ 0.54 & 0.00 $\pm$ 0.37 & 0.54 $\pm$ 0.59 & 0.81 $\pm$ 0.74\\
 DF  & 0.00 $\pm$ 0.14 & 0.00 $\pm$ 0.10 & 0.00 $\pm$ 0.16 & 0.00 $\pm$ 0.16\\
\hline
 SF + DF  & 0.27 $\pm$ 0.45 $\pm$ 0.14 & 0.00 $\pm$ 0.31 $\pm$ 0.00 & 0.54 $\pm$ 0.50 $\pm$ 0.27 & 0.81 $\pm$ 0.67 $\pm$ 0.40\\
\hline\hline
Charge Flips & 0.036 $\pm$ 0.010 $\pm$ 0.007 & - $\pm$ - & 0.069 $\pm$ 0.012 $\pm$ 0.014 & 0.104 $\pm$ 0.015 $\pm$ 0.021\\
\hline\hline
\hline
MC Pred &  0.386 $\pm$  0.022 $\pm$  0.193 &  0.451 $\pm$  0.022 $\pm$  0.225 &  0.818 $\pm$  0.031 $\pm$  0.409 &  1.655 $\pm$  0.044 $\pm$  0.827\\
\hline\hline
Total Pred &  0.695 $\pm$  0.453 $\pm$  0.236 &  0.451 $\pm$  0.315 $\pm$  0.225 &  1.422 $\pm$  0.496 $\pm$  0.489 &  2.568 $\pm$  0.672 $\pm$  0.921\\
\hline\hline
data & 1 & 1 & 0 & 2\\
\hline\hline
\end{tabular}

\end{center}
\caption{\label{tab:yieldlom0_hpt}Observed event yields in high-\pt\ (pT $>$ 20/20) dileptons
passing the {\em low-$m_0$} signal selections ($H_T > $ 320 GeV, \met $>$ 50 GeV)
compared to expectations from simulation alone, and from the data-driven methods.
The upper part of the table is based on simulation only and is used only as a reference.
The lower part is the main result of the analysis.
The SF (DF) contributions are for events with one (two) fake leptons.
The {\em MC Pred} contribution includes contributions from genuine  same-sign lepton
pairs (a sum of the rows from $V\gamma$ down to $ZZZ$).
Entries with zero contributing events are reported with an uncertainty corresponding to one event.
This uncertainty is not added to the total MC contribution.
Systematic uncertainties (the second uncertainty if present)
 are displayed only for the final combined type of background, no systematic
uncertainty is added for estimates with zero entries.
Systematic uncertainties are 100\% correlated among the channels.
}
\end{table}
\clearpage

\begin{table}[hbt]
\begin{center}
\begin{tabular}{l | l l l l}
\hline\hline
 Source  &  ee  &  $\mu\mu$  &  e$\mu$  &  all \\
\hline
$t\overline{t}\rightarrow \ell\ell X$ &  0.000 $\pm$  0.199 &  0.000 $\pm$  0.199 &  0.000 $\pm$  0.199 &  0.000 $\pm$  0.199\\
$t\overline{t}$ other &  0.000 $\pm$  0.199 &  0.000 $\pm$  0.199 &  0.000 $\pm$  0.199 &  0.000 $\pm$  0.199\\
$t\overline{t}\rightarrow \ell(b\rightarrow \ell)X$ &  0.000 $\pm$  0.199 &  0.000 $\pm$  0.199 &  0.000 $\pm$  0.199 &  0.000 $\pm$  0.199\\
$t\overline{t}\rightarrow \ell(\slashed{b}\rightarrow \ell)X$ &  0.266 $\pm$  0.266 &  0.000 $\pm$  0.199 &  0.000 $\pm$  0.199 &  0.266 $\pm$  0.266\\
\hline
$t$, s-channel &  0.000 $\pm$  0.057 &  0.000 $\pm$  0.057 &  0.000 $\pm$  0.057 &  0.000 $\pm$  0.057\\
$t$, t-channel &  0.000 $\pm$  0.055 &  0.000 $\pm$  0.055 &  0.000 $\pm$  0.055 &  0.000 $\pm$  0.055\\
$tW$ &  0.000 $\pm$  0.045 &  0.000 $\pm$  0.045 &  0.000 $\pm$  0.045 &  0.000 $\pm$  0.045\\
\hline
$Z\rightarrow ee$ &  0.000 $\pm$  0.429 &  0.000 $\pm$  0.429 &  0.000 $\pm$  0.429 &  0.000 $\pm$  0.429\\
$Z\rightarrow\mu\mu$ &  0.000 $\pm$  0.429 &  0.000 $\pm$  0.429 &  0.000 $\pm$  0.429 &  0.000 $\pm$  0.429\\
$Z\rightarrow\tau\tau$ &  0.000 $\pm$  0.429 &  0.000 $\pm$  0.429 &  0.000 $\pm$  0.429 &  0.000 $\pm$  0.429\\
$W$+jets &  0.000 $\pm$  1.808 &  0.000 $\pm$  1.808 &  0.000 $\pm$  1.808 &  0.000 $\pm$  1.808\\
$WW$ &  0.000 $\pm$  0.019 &  0.000 $\pm$  0.019 &  0.000 $\pm$  0.019 &  0.000 $\pm$  0.019\\
\hline
V$\gamma$ &  0.000 $\pm$  0.248 &  0.000 $\pm$  0.248 &  0.000 $\pm$  0.248 &  0.000 $\pm$  0.248\\
$W\gamma^{*}\rightarrow\ell\nu e e$ &  0.000 $\pm$  0.097 &  0.000 $\pm$  0.097 &  0.000 $\pm$  0.097 &  0.000 $\pm$  0.097\\
$W\gamma^{*}\rightarrow\ell\nu\mu\mu$ &  0.000 $\pm$  0.075 &  0.000 $\pm$  0.075 &  0.000 $\pm$  0.075 &  0.000 $\pm$  0.075\\
$W\gamma^{*}\rightarrow\ell\nu\tau\tau$ &  0.000 $\pm$  0.028 &  0.000 $\pm$  0.028 &  0.000 $\pm$  0.028 &  0.000 $\pm$  0.028\\
$WZ$ &  0.000 $\pm$  0.003 &  0.001 $\pm$  0.003 &  0.004 $\pm$  0.004 &  0.005 $\pm$  0.004\\
$ZZ$ &  0.000 $\pm$   0.000 &  0.000 $\pm$   0.000 &  0.000 $\pm$   0.000 &  0.000 $\pm$   0.000\\
\hline
dp$W^{\pm}W^{\pm}$ &  0.000 $\pm$  0.004 &  0.000 $\pm$  0.004 &  0.000 $\pm$  0.004 &  0.000 $\pm$  0.004\\
sp$W^{-}W^{-}$ &  0.000 $\pm$  0.001 &  0.000 $\pm$  0.001 &  0.001 $\pm$  0.001 &  0.001 $\pm$  0.001\\
sp$W^{+}W^{+}$ &  0.000 $\pm$  0.006 &  0.000 $\pm$  0.006 &  0.000 $\pm$  0.006 &  0.000 $\pm$  0.006\\
$t\overline{t}\gamma$ &  0.000 $\pm$  0.059 &  0.000 $\pm$  0.059 &  0.000 $\pm$  0.059 &  0.000 $\pm$  0.059\\
$t\overline{t}W$ &  0.143 $\pm$  0.013 &  0.168 $\pm$  0.014 &  0.302 $\pm$  0.019 &  0.614 $\pm$  0.027\\
$t\overline{t}Z$ &  0.031 $\pm$  0.005 &  0.046 $\pm$  0.006 &  0.091 $\pm$  0.008 &  0.168 $\pm$  0.011\\
$WW\gamma$ &  0.000 $\pm$  0.015 &  0.000 $\pm$  0.015 &  0.000 $\pm$  0.015 &  0.000 $\pm$  0.015\\
$WWW$ &   0.000 $\pm$   0.000 &   0.000 $\pm$   0.000 &  0.001 $\pm$   0.000 &  0.001 $\pm$  0.001\\
$WWZ$ &  0.000 $\pm$   0.000 &  0.000 $\pm$   0.000 &  0.000 $\pm$   0.000 &  0.000 $\pm$   0.000\\
$WZZ$ &  0.000 $\pm$   0.000 &  0.000 $\pm$   0.000 &  0.000 $\pm$   0.000 &  0.000 $\pm$   0.000\\
$ZZZ$ &  0.000 $\pm$   0.000 &  0.000 $\pm$   0.000 &  0.000 $\pm$   0.000 &  0.000 $\pm$   0.000\\
\hline
Total MC &  0.441 $\pm$  0.267 &  0.216 $\pm$  0.015 &  0.400 $\pm$  0.021 &  1.056 $\pm$  0.268\\
\hline\hline
\hline
LM6 &  0.000 $\pm$  0.000 &  0.000 $\pm$  0.000 &  0.000 $\pm$  0.000 &  0.000 $\pm$  0.000\\
\hline\hline
\hline\hline
 SF  & 0.00 $\pm$ 0.58 & 0.00 $\pm$ 0.37 & 0.15 $\pm$ 0.54 & 0.15 $\pm$ 0.54\\
 DF  & 0.00 $\pm$ 0.14 & 0.00 $\pm$ 0.10 & 0.00 $\pm$ 0.16 & 0.00 $\pm$ 0.16\\
\hline
 SF + DF  & 0.00 $\pm$ 0.50 $\pm$ 0.00 & 0.00 $\pm$ 0.31 $\pm$ 0.00 & 0.15 $\pm$ 0.44 $\pm$ 0.07 & 0.15 $\pm$ 0.44 $\pm$ 0.07\\
\hline\hline
Charge Flips & 0.011 $\pm$ 0.004 $\pm$ 0.002 & - $\pm$ - & 0.016 $\pm$ 0.005 $\pm$ 0.003 & 0.027 $\pm$ 0.006 $\pm$ 0.005\\
\hline\hline
\hline
MC Pred &  0.175 $\pm$  0.014 $\pm$  0.087 &  0.216 $\pm$  0.015 $\pm$  0.108 &  0.400 $\pm$  0.021 $\pm$  0.200 &  0.790 $\pm$  0.030 $\pm$  0.395\\
\hline\hline
Total Pred &  0.186 $\pm$  0.501 $\pm$  0.087 &  0.216 $\pm$  0.315 $\pm$  0.108 &  0.565 $\pm$  0.438 $\pm$  0.213 &  0.966 $\pm$  0.439 $\pm$  0.402\\
\hline\hline
data & 1 & 0 & 0 & 1\\
\hline\hline
\end{tabular}

\end{center}
\caption{\label{tab:yieldhim0_hpt}Observed event yields in high-\pt\ (pT $>$ 20/20) dileptons
passing the {\em high-$m_0$} signal selections ($H_T > 440$ GeV, \met $>$ 50 GeV)
compared to expectations from simulation alone, and from the data-driven methods.
The upper part of the table is based on simulation only and is used only as a reference.
The lower part is the main result of the analysis.
The SF (DF) contributions are for events with one (two) fake leptons.
The {\em MC Pred} contribution includes contributions from genuine  same-sign lepton
pairs (a sum of the rows from $V\gamma$ down to $ZZZ$).
Entries with zero contributing events are reported with an uncertainty corresponding to one event.
This uncertainty is not added to the total MC contribution.
Systematic uncertainties (the second uncertainty if present)
 are displayed only for the final combined type of background, no systematic
uncertainty is added for estimates with zero entries.
Systematic uncertainties are 100\% correlated among the channels.
}
\end{table}
\clearpage

\begin{table}[hbt]
\begin{center}
\begin{tabular}{l | l l l l}
\hline\hline
 Source  &  ee  &  $\mu\mu$  &  e$\mu$  &  all \\
\hline
$t\overline{t}\rightarrow \ell\ell X$ &  0.000 $\pm$  0.199 &  0.000 $\pm$  0.199 &  0.000 $\pm$  0.199 &  0.000 $\pm$  0.199\\
$t\overline{t}$ other &  0.000 $\pm$  0.199 &  0.000 $\pm$  0.199 &  0.000 $\pm$  0.199 &  0.000 $\pm$  0.199\\
$t\overline{t}\rightarrow \ell(b\rightarrow \ell)X$ &  0.000 $\pm$  0.199 &  0.000 $\pm$  0.199 &  0.000 $\pm$  0.199 &  0.000 $\pm$  0.199\\
$t\overline{t}\rightarrow \ell(\slashed{b}\rightarrow \ell)X$ &  0.000 $\pm$  0.199 &  0.000 $\pm$  0.199 &  0.000 $\pm$  0.199 &  0.000 $\pm$  0.199\\
\hline
$t$, s-channel &  0.000 $\pm$  0.057 &  0.000 $\pm$  0.057 &  0.000 $\pm$  0.057 &  0.000 $\pm$  0.057\\
$t$, t-channel &  0.000 $\pm$  0.055 &  0.000 $\pm$  0.055 &  0.000 $\pm$  0.055 &  0.000 $\pm$  0.055\\
$tW$ &  0.000 $\pm$  0.045 &  0.000 $\pm$  0.045 &  0.000 $\pm$  0.045 &  0.000 $\pm$  0.045\\
\hline
$Z\rightarrow ee$ &  0.000 $\pm$  0.429 &  0.000 $\pm$  0.429 &  0.000 $\pm$  0.429 &  0.000 $\pm$  0.429\\
$Z\rightarrow\mu\mu$ &  0.000 $\pm$  0.429 &  0.000 $\pm$  0.429 &  0.000 $\pm$  0.429 &  0.000 $\pm$  0.429\\
$Z\rightarrow\tau\tau$ &  0.000 $\pm$  0.429 &  0.000 $\pm$  0.429 &  0.000 $\pm$  0.429 &  0.000 $\pm$  0.429\\
$W$+jets &  0.000 $\pm$  1.808 &  0.000 $\pm$  1.808 &  0.000 $\pm$  1.808 &  0.000 $\pm$  1.808\\
$WW$ &  0.000 $\pm$  0.019 &  0.000 $\pm$  0.019 &  0.000 $\pm$  0.019 &  0.000 $\pm$  0.019\\
\hline
V$\gamma$ &  0.000 $\pm$  0.248 &  0.000 $\pm$  0.248 &  0.000 $\pm$  0.248 &  0.000 $\pm$  0.248\\
$W\gamma^{*}\rightarrow\ell\nu e e$ &  0.000 $\pm$  0.097 &  0.000 $\pm$  0.097 &  0.000 $\pm$  0.097 &  0.000 $\pm$  0.097\\
$W\gamma^{*}\rightarrow\ell\nu\mu\mu$ &  0.000 $\pm$  0.075 &  0.000 $\pm$  0.075 &  0.000 $\pm$  0.075 &  0.000 $\pm$  0.075\\
$W\gamma^{*}\rightarrow\ell\nu\tau\tau$ &  0.000 $\pm$  0.028 &  0.000 $\pm$  0.028 &  0.000 $\pm$  0.028 &  0.000 $\pm$  0.028\\
$WZ$ &  0.005 $\pm$  0.005 &  0.000 $\pm$  0.003 &  0.000 $\pm$  0.003 &  0.005 $\pm$  0.005\\
$ZZ$ &  0.000 $\pm$   0.000 &  0.000 $\pm$   0.000 &   0.000 $\pm$   0.000 &   0.000 $\pm$   0.000\\
\hline
dp$W^{\pm}W^{\pm}$ &  0.000 $\pm$  0.004 &  0.000 $\pm$  0.004 &  0.000 $\pm$  0.004 &  0.000 $\pm$  0.004\\
sp$W^{-}W^{-}$ &  0.000 $\pm$  0.001 &  0.000 $\pm$  0.001 &  0.000 $\pm$  0.001 &  0.000 $\pm$  0.001\\
sp$W^{+}W^{+}$ &  0.000 $\pm$  0.006 &  0.000 $\pm$  0.006 &  0.000 $\pm$  0.006 &  0.000 $\pm$  0.006\\
$t\overline{t}\gamma$ &  0.000 $\pm$  0.059 &  0.000 $\pm$  0.059 &  0.000 $\pm$  0.059 &  0.000 $\pm$  0.059\\
$t\overline{t}W$ &  0.028 $\pm$  0.006 &  0.041 $\pm$  0.006 &  0.076 $\pm$  0.009 &  0.145 $\pm$  0.012\\
$t\overline{t}Z$ &  0.003 $\pm$  0.001 &  0.009 $\pm$  0.003 &  0.008 $\pm$  0.002 &  0.020 $\pm$  0.004\\
$WW\gamma$ &  0.000 $\pm$  0.015 &  0.000 $\pm$  0.015 &  0.000 $\pm$  0.015 &  0.000 $\pm$  0.015\\
$WWW$ &   0.000 $\pm$   0.000 &  0.000 $\pm$   0.000 &   0.000 $\pm$   0.000 &   0.000 $\pm$   0.000\\
$WWZ$ &  0.000 $\pm$   0.000 &  0.000 $\pm$   0.000 &  0.000 $\pm$   0.000 &  0.000 $\pm$   0.000\\
$WZZ$ &  0.000 $\pm$   0.000 &  0.000 $\pm$   0.000 &  0.000 $\pm$   0.000 &  0.000 $\pm$   0.000\\
$ZZZ$ &  0.000 $\pm$   0.000 &  0.000 $\pm$   0.000 &   0.000 $\pm$   0.000 &   0.000 $\pm$   0.000\\
\hline
Total MC &  0.036 $\pm$  0.008 &  0.049 $\pm$  0.007 &  0.085 $\pm$  0.009 &  0.170 $\pm$  0.014\\
\hline\hline
\hline
LM6 &  0.000 $\pm$  0.000 &  0.000 $\pm$  0.000 &  0.000 $\pm$  0.000 &  0.000 $\pm$  0.000\\
\hline\hline
\hline\hline
 SF  & 0.00 $\pm$ 0.58 & 0.00 $\pm$ 0.37 & 0.18 $\pm$ 0.55 & 0.18 $\pm$ 0.55\\
 DF  & 0.00 $\pm$ 0.14 & 0.00 $\pm$ 0.10 & 0.00 $\pm$ 0.16 & 0.00 $\pm$ 0.16\\
\hline
 SF + DF  & 0.00 $\pm$ 0.50 $\pm$ 0.00 & 0.00 $\pm$ 0.31 $\pm$ 0.00 & 0.18 $\pm$ 0.45 $\pm$ 0.09 & 0.18 $\pm$ 0.45 $\pm$ 0.09\\
\hline\hline
Charge Flips & 0.007 $\pm$ 0.003 $\pm$ 0.001 & - $\pm$ - & 0.010 $\pm$ 0.004 $\pm$ 0.002 & 0.017 $\pm$ 0.005 $\pm$ 0.003\\
\hline\hline
\hline
MC Pred &  0.036 $\pm$  0.008 $\pm$  0.018 &  0.049 $\pm$  0.007 $\pm$  0.025 &  0.085 $\pm$  0.009 $\pm$  0.042 &  0.170 $\pm$  0.014 $\pm$  0.085\\
\hline\hline
Total Pred &  0.043 $\pm$  0.501 $\pm$  0.018 &  0.049 $\pm$  0.315 $\pm$  0.025 &  0.270 $\pm$  0.447 $\pm$  0.097 &  0.363 $\pm$  0.447 $\pm$  0.122\\
\hline\hline
data & 1 & 0 & 1 & 2\\
\hline\hline
\end{tabular}

\end{center}
\caption{\label{tab:yieldsimple_hpt}Observed event yields in high-\pt\ (pT $>$ 20/20) dileptons
passing the {\em simplified model} signal selections ($H_T > 200$ GeV, \met $>$ 120 GeV)
compared to expectations from simulation alone, and from the data-driven methods.
The upper part of the table is based on simulation only and is used only as a reference.
The lower part is the main result of the analysis.
The SF (DF) contributions are for events with one (two) fake leptons.
The {\em MC Pred} contribution includes contributions from genuine  same-sign lepton
pairs (a sum of the rows from $V\gamma$ down to $ZZZ$).
Entries with zero contributing events are reported with an uncertainty corresponding to one event.
This uncertainty is not added to the total MC contribution.
Systematic uncertainties (the second uncertainty if present)
 are displayed only for the final combined type of background, no systematic
uncertainty is added for estimates with zero entries.
Systematic uncertainties are 100\% correlated among the channels.
}
\end{table}

\clearpage

\begin{table}[hbt]
\begin{center}
\begin{tabular}{l | l l l l}
\hline\hline
 Source  &  ee  &  $\mu\mu$  &  e$\mu$  &  all \\
\hline
$t\overline{t}\rightarrow \ell\ell X$ &  0.000 $\pm$  0.199 &  0.000 $\pm$  0.199 &  0.000 $\pm$  0.199 &  0.000 $\pm$  0.199\\
$t\overline{t}$ other &  0.000 $\pm$  0.199 &  0.000 $\pm$  0.199 &  0.000 $\pm$  0.199 &  0.000 $\pm$  0.199\\
$t\overline{t}\rightarrow \ell(b\rightarrow \ell)X$ &  0.000 $\pm$  0.199 &  0.000 $\pm$  0.199 &  0.000 $\pm$  0.199 &  0.000 $\pm$  0.199\\
$t\overline{t}\rightarrow \ell(\slashed{b}\rightarrow \ell)X$ &  0.000 $\pm$  0.199 &  0.000 $\pm$  0.199 &  0.000 $\pm$  0.199 &  0.000 $\pm$  0.199\\
\hline
$t$, s-channel &  0.000 $\pm$  0.057 &  0.000 $\pm$  0.057 &  0.000 $\pm$  0.057 &  0.000 $\pm$  0.057\\
$t$, t-channel &  0.000 $\pm$  0.055 &  0.000 $\pm$  0.055 &  0.000 $\pm$  0.055 &  0.000 $\pm$  0.055\\
$tW$ &  0.000 $\pm$  0.045 &  0.000 $\pm$  0.045 &  0.000 $\pm$  0.045 &  0.000 $\pm$  0.045\\
\hline
$Z\rightarrow ee$ &  0.000 $\pm$  0.429 &  0.000 $\pm$  0.429 &  0.000 $\pm$  0.429 &  0.000 $\pm$  0.429\\
$Z\rightarrow\mu\mu$ &  0.000 $\pm$  0.429 &  0.000 $\pm$  0.429 &  0.000 $\pm$  0.429 &  0.000 $\pm$  0.429\\
$Z\rightarrow\tau\tau$ &  0.000 $\pm$  0.429 &  0.000 $\pm$  0.429 &  0.000 $\pm$  0.429 &  0.000 $\pm$  0.429\\
$W$+jets &  0.000 $\pm$  1.808 &  0.000 $\pm$  1.808 &  0.000 $\pm$  1.808 &  0.000 $\pm$  1.808\\
$WW$ &  0.000 $\pm$  0.019 &  0.000 $\pm$  0.019 &  0.000 $\pm$  0.019 &  0.000 $\pm$  0.019\\
\hline
V$\gamma$ &  0.000 $\pm$  0.248 &  0.000 $\pm$  0.248 &  0.000 $\pm$  0.248 &  0.000 $\pm$  0.248\\
$W\gamma^{*}\rightarrow\ell\nu e e$ &  0.000 $\pm$  0.097 &  0.000 $\pm$  0.097 &  0.000 $\pm$  0.097 &  0.000 $\pm$  0.097\\
$W\gamma^{*}\rightarrow\ell\nu\mu\mu$ &  0.000 $\pm$  0.075 &  0.000 $\pm$  0.075 &  0.000 $\pm$  0.075 &  0.000 $\pm$  0.075\\
$W\gamma^{*}\rightarrow\ell\nu\tau\tau$ &  0.000 $\pm$  0.028 &  0.000 $\pm$  0.028 &  0.000 $\pm$  0.028 &  0.000 $\pm$  0.028\\
$WZ$ &  0.004 $\pm$  0.004 &  0.001 $\pm$  0.003 &   0.000 $\pm$  0.003 &  0.006 $\pm$  0.005\\
$ZZ$ &  0.000 $\pm$   0.000 &  0.000 $\pm$   0.000 &   0.000 $\pm$   0.000 &   0.000 $\pm$   0.000\\
\hline
dp$W^{\pm}W^{\pm}$ &  0.000 $\pm$  0.004 &  0.000 $\pm$  0.004 &  0.000 $\pm$  0.004 &  0.000 $\pm$  0.004\\
sp$W^{-}W^{-}$ &  0.000 $\pm$  0.001 &  0.000 $\pm$  0.001 &  0.001 $\pm$  0.001 &  0.001 $\pm$  0.001\\
sp$W^{+}W^{+}$ &  0.000 $\pm$  0.006 &  0.000 $\pm$  0.006 &  0.000 $\pm$  0.006 &  0.000 $\pm$  0.006\\
$t\overline{t}\gamma$ &  0.000 $\pm$  0.059 &  0.000 $\pm$  0.059 &  0.000 $\pm$  0.059 &  0.000 $\pm$  0.059\\
$t\overline{t}W$ &  0.117 $\pm$  0.012 &  0.141 $\pm$  0.013 &  0.250 $\pm$  0.017 &  0.508 $\pm$  0.025\\
$t\overline{t}Z$ &  0.017 $\pm$  0.004 &  0.026 $\pm$  0.004 &  0.053 $\pm$  0.007 &  0.096 $\pm$  0.009\\
$WW\gamma$ &  0.000 $\pm$  0.015 &  0.000 $\pm$  0.015 &  0.000 $\pm$  0.015 &  0.000 $\pm$  0.015\\
$WWW$ &  0.000 $\pm$   0.000 &   0.000 $\pm$   0.000 &   0.000 $\pm$   0.000 &   0.000 $\pm$   0.000\\
$WWZ$ &  0.000 $\pm$   0.000 &  0.000 $\pm$   0.000 &  0.000 $\pm$   0.000 &  0.000 $\pm$   0.000\\
$WZZ$ &   0.000 $\pm$   0.000 &  0.000 $\pm$   0.000 &   0.000 $\pm$   0.000 &   0.000 $\pm$   0.000\\
$ZZZ$ &  0.000 $\pm$   0.000 &  0.000 $\pm$   0.000 &  0.000 $\pm$   0.000 &  0.000 $\pm$   0.000\\
\hline
Total MC &  0.138 $\pm$  0.013 &  0.168 $\pm$  0.014 &  0.305 $\pm$  0.019 &  0.611 $\pm$  0.027\\
\hline\hline
\hline
LM6 &  0.000 $\pm$  0.000 &  0.000 $\pm$  0.000 &  0.000 $\pm$  0.000 &  0.000 $\pm$  0.000\\
\hline\hline
\hline\hline
 SF  & 0.00 $\pm$ 0.58 & 0.00 $\pm$ 0.37 & 0.15 $\pm$ 0.54 & 0.15 $\pm$ 0.54\\
 DF  & 0.00 $\pm$ 0.14 & 0.00 $\pm$ 0.10 & 0.00 $\pm$ 0.16 & 0.00 $\pm$ 0.16\\
\hline
 SF + DF  & 0.00 $\pm$ 0.50 $\pm$ 0.00 & 0.00 $\pm$ 0.31 $\pm$ 0.00 & 0.15 $\pm$ 0.44 $\pm$ 0.07 & 0.15 $\pm$ 0.44 $\pm$ 0.07\\
\hline\hline
Charge Flips & 0.014 $\pm$ 0.006 $\pm$ 0.003 & - $\pm$ - & 0.011 $\pm$ 0.005 $\pm$ 0.002 & 0.025 $\pm$ 0.008 $\pm$ 0.005\\
\hline\hline
\hline
MC Pred &  0.138 $\pm$  0.013 $\pm$  0.069 &  0.168 $\pm$  0.014 $\pm$  0.084 &  0.305 $\pm$  0.019 $\pm$  0.153 &  0.612 $\pm$  0.027 $\pm$  0.306\\
\hline\hline
Total Pred &  0.152 $\pm$  0.501 $\pm$  0.069 &  0.168 $\pm$  0.315 $\pm$  0.084 &  0.466 $\pm$  0.438 $\pm$  0.170 &  0.787 $\pm$  0.439 $\pm$  0.315\\
\hline\hline
data & 0 & 0 & 0 & 0\\
\hline\hline
\end{tabular}

\end{center}
\caption{\label{tab:yieldsnu_hpt}Observed event yields in high-\pt\ (pT $>$ 20/20) dileptons
passing the {\em pMSSW/sneutrino} signal selections ($H_T > 320$ GeV, \met $>$ 120 GeV)
compared to expectations from simulation alone, and from the data-driven methods.
The upper part of the table is based on simulation only and is used only as a reference.
The lower part is the main result of the analysis.
The SF (DF) contributions are for events with one (two) fake leptons.
The {\em MC Pred} contribution includes contributions from genuine  same-sign lepton
pairs (a sum of the rows from $V\gamma$ down to $ZZZ$).
Entries with zero contributing events are reported with an uncertainty corresponding to one event.
This uncertainty is not added to the total MC contribution.
Systematic uncertainties (the second uncertainty if present)
 are displayed only for the final combined type of background, no systematic
uncertainty is added for estimates with zero entries.
Systematic uncertainties are 100\% correlated among the channels.
}
\end{table}

\clearpage

The estimation
is in a good agreement with the observation. 
We also note that the backgrounds with respect to the inclusive same-sign
dilepton search is suppressed by an order of magnitude due to the b-tag requirements.
As seen in the yield tables above, the contribution from fake leptons from $b$-quark decays in \ttbar,
which is the dominant in the pre-tagged sample analysis~\cite{ssnote2011},
is now suppressed and is no longer a dominant one.
This confirms our initial expectation from requiring two b-tagged jets.


{\bf This section is missing two things. 
First, we will add a \met\ vs $H_T$ plot that shows the data and
one of the models overlayed. 
Second, we will add a summary of the events we see. 
}


 




