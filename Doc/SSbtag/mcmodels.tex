\section{Simulated Event samples}
\label{sec:mcsignal}

In the MSSM, the scalar partners of the left- and right-handed squarks, $\tilde{q_L}$
and $\tilde{q_R}$, can mix to form mass eigenstates. These mixing effects thus leads to different fermion
masses and therefore becomes important for the third generation. The large mixing can yield sbottom ($\tilde{b_1}$)
and stop ($\tilde{t_1}$) eigenstates which are significantly lighter than other squarks. This also implies  $\tilde{b_1}$
and $\tilde{t_1}$ could be produced with larger cross sections then the light flavour squarks at the LHC. Consequently, 
the subsequent decays of $\tilde{b_1}$ and $\tilde{t_1}$ can lead to same sign dileptons in association with b quarks.
Furthermore, the $\tilde{g}\tilde{g}$ productions via $\tilde{\chi}^\pm_1$ such as $\tilde{g} \rightarrow \tilde{\chi}^{+}_{1} b \bar{t}$
can also give rise to same sign dileptons with b jets.
