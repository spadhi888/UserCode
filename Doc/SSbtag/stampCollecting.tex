\section{Searches for Specific Models}
\label{sec:stampCollecting}

Our signature, two isolated same-sign leptons plus, at least two b-tagged jets, and \met, 
is common to many different new physics scenarios.
Here we refine our analysis to define dedicated signal regions for a few of these scenarios,
and provide 95\% C.L. upper limits on their respective model parameter space.


\subsection{Same sign top production due a $Z'$}
\label{sec:sstops}

This is an extension of the 2010 CMS published CMS analysis\cite{sstop}.
The main difference is that here in order to improve the signal-to-noise
we require two b-tagged jets.  This would not have made sense in 2010, since 
at the time the integrated luminosity was low enough that the analysis was almost
background free without requiring b-tags.

\subsubsection{Theoretical Discussion, $Z'$ model}

Recent measurements of the inclusive forward-backward $t\bar{t}$ production 
asymmetry ($A_{FB}$) from the 
Tevatron experiments show deviations from the standard model 
(SM) expectations~\cite{d0:fwtop, cdf:fwtop1, cdf:fwtop2}.
% The largest (3$\sigma$) deviation~\cite{cdf:fwtop2} is found 
% to be in the region of high invariant mass with $M_{t\bar{t}} >  450$ GeV. 
Several attempts have been made to explain this asymmetry~\cite{fcnczprime, Buckley, Gresham, zoltan}. 
One of the most natural ways to induce such an asymmetry would be through
Flavor Changing Neutral Currents (FCNC) in the top quark sector. 
The forward-backward asymmetry in $u\bar{u} \to t\bar{t}$ would then be generated
by t-channel exchange of a new massive $Z'$ boson that couples chirally to
$u$ and $t$ at the same vertex, as shown in Fig.~\ref{fig:ttbar}~\cite{fcnczprime}.
The same type of interaction would also give rise to same-sign top pair production, 
as illustrated in Fig.~\ref{fig:tchannel} and Fig.~\ref{fig:schannel}. 
In this case, the initial state involves two $u-$quarks and 
thus the cross section at the LHC is enhanced due 
to the large valence quark parton density of the proton. 

\begin{figure}[htb]
\begin{center}
\includegraphics[width=0.35\linewidth, height=0.25\linewidth]{figs/ttbar_Z.pdf}
\caption{ Diagram for $t\bar{t}$ production induced by $Z'$ exchange which
can generate a forward-backward asymmetry. \label{fig:ttbar}}
\end{center}
\end{figure}

\begin{figure}[htb]
\begin{center}
\includegraphics[width=0.7\linewidth, height=0.2\linewidth]{figs/sstop1.pdf}
\caption{ Diagrams for $tt$ pair production induced by $Z'$ exchange in the t-channel. 
\label{fig:tchannel}}
\end{center}
\end{figure}

\begin{figure}[htb]
\begin{center}
\includegraphics[width=0.7\linewidth, height=0.25\linewidth]{figs/sstop2.pdf}
\caption{ Diagrams for $tt\bar{u}$ production induced by $Z'$ exchange in the s-channel 
\label{fig:schannel}}
\end{center}
\end{figure}


We consider the model of Reference~\cite{fcnczprime}.  
The relevant $u-t-Z'$ interaction term in the Lagrangian is:

\begin{equation}
\label{eqn:L_berger}
  \mathcal{L} = g_W \bar{u} \gamma^\mu (f_L P_L + f_R P_R)tZ'_\mu + h.c
\end{equation}

where $g_W$ is the weak coupling strength. The left-handed coupling is set to $f_L = 0$, due 
to the $B_d-\bar{B_d}$ mixing constraint~\cite{Cao}. 
The right-handed coupling $f_R$ and the $Z'$ mass are free parameters in the model.
Within this model there is a narrow range of parameter space
consistent with the TeVatron measurements of $\sigma(p\bar{p} \to t\bar{t})$ 
and $A_{FB}$, which is not excluded by direct searches for same sign tops.
This region is illustrated in Fig.~\ref{fig:berger_limit}.

% Fig.~\ref{fig:tchannel} shows the t-channel exchange diagrams that can lead to the same-sign $tt$ final state. 
% As expected the coupling appears twice in the Feynman diagrams, thus the predicated rate is proportional to $f_R^4$. 

\begin{figure}[htb]
\begin{center}
\includegraphics[width=0.4\linewidth]{figs/berger_limit.pdf}
\caption{\protect From Reference~\cite{fcnczprime}; the shaded area covers the parameter
space consistent with the $A_{FB}$ and $\sigma(t\bar{t})$ from the Tevatron;
The line indicated by the arrow shows the Tevatron limit inferred by the authors
from same sign top searches at the Tevatron; the remaining lines represent the
expectations of Reference~\cite{fcnczprime}.
for LHC searches in 1 fb$^{-1}$. \label{fig:berger_limit}}
\end{center}
\end{figure}

Monte Carlo events for this model were generated using Madgraph in the same way as 
for the 2010 analysis (see Reference~\cite{ttAN}).



\subsubsection{Signal region definition for same sign top from $Z'$}
\label{sec:sstopsigdefinition}
In this study we search for same-sign dileptons originating from $tt$ 
or $ttj$ pair production as described above.  At the LHC $uu \to tt$ 
dominates over $\bar{u}\bar{u} \to \bar{t}\bar{t}$, thus we concentrate
on same-sign positive leptons.  The \met~~and $H_T$ cuts are typical 
of a dilepton top analysis: two or more jets of $P_T>40$ GeV, 
$\met > 30$ GeV and $H_T > 80$ GeV.  This corresponds to Table~\ref{tab:yieldBase_pp}:
5 events observed and 4.42 $\pm$ 0.80 $\pm$ 1.39 expected from background.

\subsubsection{Limits on the $Z'$ model}
\label{sec:sstopslimits}
Using the results from Section~\ref{sec:sstopsigdefinition}, we set 
a limit at 95\% CL of 7.2 events using the CL$_{\rm S}$ method.
The expected limit is 6.4 events.
% $7.8^{+3.6}_{-3.1}$ events.
In the MC we find $Acc \times Eff \times BR = 0.00233$, independent of $Z'$ mass. 
This results in an upper limit on the cross-section of 0.67 pb.
The limit includes uncertainty
on JES (12\%), btagging (10\%), lepton efficiencies (11\%), luminosity (4.5\%),
and PDF (3\%).

The cross-section limit is turned into an exclusion limit in the $m(Z')$ vs $f_R$
plane using the LO calculation of the $pp \to tt$ cross-section in this model.
This is shown in Figure~\ref{fig:sstopexclusion}, together with the corresponding
plot from the 2010 analysis.


For $M_{Z'} >> M_{\rm top}$ the Lagrangian of equation 1 is 
equivalent to 
$\mathcal{L} = -\frac{1}{2}\frac{C_{RR}}{\Lambda^2}
 [\bar{u} \gamma^\mu t][\bar{u} \gamma_{\mu} t] + h.c.$~\cite{cdfth2},
with $\frac{C_{RR}}{\Lambda^2} = \frac{2 g_W^2 f_R^2}{M_{Z'}^2}$.
 Our limit on $f_R$, calculated for $M_{Z'}=2$ TeV, 
would then correspond to $\frac{C_{RR}}{\Lambda^2} < 0.6$ TeV$^{-2}$ at 
95\% confidence.  This is more stringent than the limit recently reported
by CDF: $\frac{C_{RR}}{\Lambda^2} < 3.7$ TeV$^{-2}$~\cite{cdflimit}.


\begin{figure}[htb]
\begin{center}
\includegraphics[width=0.45\linewidth]{figs/zprimecombined.pdf}
\includegraphics[width=0.45\linewidth]{figs/sscomb.pdf}
\caption{Exclusion regions from the 2011 analysis (left) and the 2010 analysis (right).
The exclusions are obtained using the LO cross-section for $tt$ production.  
Note that the cross-section is proportional to $f_R^4$.
\label{fig:sstopexclusion}}
\end{center}
\end{figure}

%\subsubsection{What is still missing for the $Z'$ model}
%\begin{itemize}
%\item Double check calculation of $\frac{C_{RR}}{\Lambda^2}$
%\item Double check acceptance numbers
%\end{itemize}


%\clearpage


\subsection{Maximally Flavor Violation Model (MXFV)}
\label{sec:mxfv}

\subsubsection{Theoretical discussion of MXFV}
\label{sec:mxfvtheory}

This is a model~\cite{mxflv1,mxflv2,mxflv3} with a new scalar 
SU(2) doublet field $\Phi_{FV} = (\eta^0,\eta^+)$ that couples the first and third 
generation quarks ($q_1,q_3$) via a Lagrangian term 
$\mathcal{L}_{FV} = \xi_{13} \Phi_{FV} q_1 q_3$.  Remarkably, it appears that this
model is largely consistent with constraints from flavor physics.

The model results in same sign top pairs in the final state as foolows
\begin{itemize}

\item Single $\eta^0$ production: $ug \to t\eta^0 \to tt\bar{u}, t\bar{t}u$

\item $\eta^0$ pair production: $u \bar{u} \to \eta^0 \eta^0 \to tt\bar{u}\bar{u},
uu\bar{t}\bar{t}, t\bar{t}u\bar{u}$

\item $\eta^0$ $t$-channel exchange: $uu \to tt$, $\bar{u}\bar{u} \to \bar{t}\bar{t}$
\end{itemize}

Monte Carlo events were generated using LHE files\cite{simplifiedModel} interfaced 
with Madgraph.  Madgraph was used to decay the top quarks in order to preserve 
spin-correlations.  The cross-sections at LO for same sign $tt$ pairs for the three
processes in the MXFV model is shown in Figure~\ref{fig:mxvxsec}.  The $t$-channel
process is the most important.

\begin{figure}[htb]
\begin{center}
\includegraphics[width=0.55\linewidth]{figs/mxvxsec.pdf}
\caption{Cross section at LO for the $tt$ final state in the three MXFV modes
as a function of $\eta^0$ mass for $\xi = 1$.
\label{fig:mxvxsec}}
\end{center}
\end{figure}

\subsubsection{Signal region definition for the MXFV model}
\label{sec:mxfvdefinition}
The properties of the final state in this model are basically the same as in the $Z'$ model.
Thus, we use the same signal region definition (see Section \ref{sec:sstopsigdefinition}).


\subsubsection{Limits for the MXFV model}
\label{sec:mxfvlimits}
Our limits in the $\xi$-M($\eta^0$) plane are shown in 
Figure~\ref{fig:MxVExcl}.
They are calculated using the LO cross-section for this model.

\begin{figure}[htb]
\begin{center}
\includegraphics[width=0.45\linewidth]{figs/MxVExclCombined.pdf}
\includegraphics[width=0.45\linewidth]{figs/CDFlimit.png}
\caption{Limits in the $\xi$-Mass($\chi^0$) plane.  Left: CMS.  Right: CDF
\label{fig:MxVExcl}}
\end{center}
\end{figure}

%\subsubsection{What is still missing for the MXFV model}
%\begin{itemize}
%\item Need to include the $tt\bar{u}$ final state
%\item Rescale everything to take into account that the leptonic BR in MG is not quite right
%\item Signal Contamination effects (should be tiny)
%\item Prettify the exclusion plot, put CDF on same plot perhaps
%\end{itemize}



\subsection{$\widetilde{g} \to t\widetilde{t}$ Model}
\label{sec:firststopmodel}

\subsubsection{Theoretical discussion of the $\widetilde{g} \to t\widetilde{t}$ Model}
\label{sec:firststopmodeltheory}

This is an interesting model for stop pair production through gluino 
decays\cite{susyssbtags}\cite{susyssbtags2}\cite{wacker}\cite{naturalness4}.
It is a ``realistic'' and well-motivated 
model in the sense that it applies to the situation 
where all the squarks except the stop are very heavy.  A ``light'' stop is of course
generally favored in SUSY, and LHC results are pointing to ``heavy'' superpartners.
Then if the stop
is light enough the gluino would decay with 100\% BR as $\widetilde{g} \to t\widetilde{t}$
and then the stop would decay as $\widetilde{t} \to t \chi_1^0$, if kinematically 
accessible.
The parameters of the model are $M(\widetilde{g})$, $M(\widetilde{t})$, $M(\chi_1^0)$.

The final state after gluino pair production is then $tt\bar{t}\bar{t}\chi_1^0\chi_1^0$.
It is the same final state as the {\tt T1tttt} 
simplified model\cite{T1tttt}, except that 
it proceeds through an intermediate stop.  This final state is rich in leptons,
and has four b-quarks.  The same sign dilepton $+$ btags $+$ 
\met~~signature is a 
particularly good way to go after it.


\subsubsection{Signal region definition for the $\widetilde{g} \to t\widetilde{t}$ Model}
\label{sec:firststopdefinition}

\begin{figure}[htb]
\begin{center}
\includegraphics[width=0.65\linewidth]{figs/gluinostopsigreg.pdf}
\caption{The signal region with the best expected limit as a function of 
$m(\widetilde{g}$ vs. $m(\widetilde{t})$ plane for $m(\chi^0_1)$=50 GeV.
The coding is: 1=(200-50), 2=(200-150), 3=(320-50), and 4=(320-120), where
the first (second) number is the $H_T$ (\met) threshold in GeV. The number
of requested btags is 2 or more.
\label{fig:gluinostoptimize}}
\end{center}
\end{figure}


For each point in parameter space we use the signal region that gives
the best expected limit.  
{\bf (Note: so far the region with 3 btags has not been used).}
Limits are calculated using all experimental
uncertainties; the JES and btag uncertainties are calculated point-by-point.
An example of this optimization is shown in Figure~\ref{fig:gluinostoptimize},
where we show the choice of signal region that gives the best expected limit
in the $m(\widetilde{g})$ vs. $m(\widetilde{t})$ plane for the choice
$m(\chi^0_1)$=50 GeV.



%Using $\met > 50$ GeV and $H_T > 200$ GeV, which corresponds 
%to Table~\ref{tab:yield_ht200met50}: 5 events observed and 3.723 $\pm$ 0.787 $\pm$ 1.23 expected from backgrounds,
%we set a limit at 95\% CL of 7.65 events with the  CL$_{\rm S}$ method. The expected limit is 6.13. 

\subsubsection{Limits for the $\widetilde{g} \to t\widetilde{t}$ Model}
\label{sec:firststoplimits}



The limits on the production cross-section in this model in the 
gluino mass vs. stop mass plane for two choices of the 
LSP mass are shown in Figure~\ref{fig:mglinoStop}
Using the 
NLO$+$NLL cross-section for gluino pair production, we also place a limit
on the mass parameters of this model.


\begin{figure}[htb]
\begin{center}
\includegraphics[width=0.47\linewidth]{figs/gluinostop50.pdf}
\includegraphics[width=0.47\linewidth]{figs/gluinostop150.pdf}
\caption{Cross section limits in the $m(\widetilde{g})$ vs. $m(\widetilde{t})$ plane
for $m(\chi_1^0)$ = 50 GeV (left) and 150 GeV (right).
\label{fig:mglinoStop}}
\end{center}
\end{figure}


%\subsubsection{What is missing for the $\widetilde{g} \to t\widetilde{t}$ Model}
%\begin{itemize}
%\item Perhaps more details on the MC signal generation???
%\item Need a reference for the {\tt T1tttt} model
%\item Maybe also a plot for LSP mass = 100 GeV? 
%\end{itemize}

\subsection{{\tt T1tttt} Model}
\label{t1ttmodel}

\subsubsection{Theoretical discussion of the {\tt T1tttt} Model}
\label{sec:t1tttheory}
The {\tt T1tttt} simplified model\cite{T1tttt} is very similar to the model of 
Section~\ref{sec:firststopmodel}.  In this model it is assumed that all squarks 
are very heavy, but the stop is somewhat lighter than the other 
quarks\cite{stopVirtual}\cite{stopVirtualPRD}.
Then the gluino would decay as $\widetilde{g} \to t\bar{t}\chi_1^0$ through virtual stops.
Other gluino decay modes would be suppressed because the stop is the lightest squark.
The final state after gluino pair production is $tt\bar{t}\bar{t}\chi_1^0\chi_1^0$,
just as in Section~\ref{sec:firststopmodel}.
The model parameters are $M(\widetilde{g})$ and $M(\chi_1^0)$.



\subsubsection{Signal region definition for the {\tt T1tttt} Model}
\label{sec:t1ttttdefinition}
For each point in parameter space we use the signal region that gives
the best expected limit, see Figure~\ref{fig:t1tttoptimize}.
{\bf (Note: so far the region with 3 btags has not been used).}
The limits include all experimental 
uncertainties.   The JES and btag uncertainties are calculated point-by-point.


\begin{figure}[htb]
\begin{center}
\includegraphics[width=0.65\linewidth]{figs/t1ttttsigreg.pdf}
\caption{The signal region with the best expected limit as a function of 
$m(\widetilde{g}$ vs. $m(\widetilde{t})$ plane for $m(\chi^0_1)$=50 GeV.
The coding is: 1=(200-50), 2=(200-150), 3=(320-50), and 4=(320-120), where
the first (second) number is the $H_T$ (\met) threshold in GeV. The number
of requested btags is 2 or more.
\label{fig:t1tttoptimize}}
\end{center}
\end{figure}


\subsubsection{Limits for the {\tt T1tttt} Model}
\label{sec:t1ttttlimits}
The limit on the production cross-section in this model in the 
gluino mass vs. LSP mass plane shown in Figure~\ref{fig:T1ttttLimit}.  
Using the 
NLO$+$NLL cross-section for gluino pair production, we place a limit
on the mass parameters as shown in Figure~\ref{fig:T1ttttLimit}.

\begin{figure}[htb]
\begin{center}
\includegraphics[width=0.65\linewidth]{figs/T1tttt.pdf}
\caption{Cross section limits in the $m(\widetilde{g})$ vs. $m(\chi_1^0)$ plane for the
{\tt T1tttt} model.  
\label{fig:T1ttttLimit}}
\end{center}
\end{figure}

%\subsubsection{What is missing for the {\tt T1tttt} Model}
%\begin{itemize}
%\item Need at least a sentence to say something which signal region contributes.  Or a plot.
%\item Need a reference for the {\tt T1tttt} model
%\end{itemize}


\subsection{Sbottom pair production model}
\label{sec:sbottompair}
In this model we have $pp \to \tilde{b}\tilde{b}$.  The sbottom decays 
as $\tilde{b} \to t\chi^{-}$ followed by $\chi^{-} \to W^- \chi_1^0$. 
The final state is $t\bar{t}W^+W^- \chi_1^0 \chi_1^0$. 
The model parameters are $M(\widetilde{b})$, $M(\chi_1^0)$, and $M(\chi^{\pm})$.
For simplicity we only consider mass parameters such that the $\chi^{-}$ is on shell.

\subsubsection{Signal region definition for the sbottom pair production model}
\label{sec:sbottompairdefinition}
For each point in parameter space we use the signal region that gives
the best expected limit, see Figure~\ref{fig:sbottomoptimize}.
{\bf (Note: so far the region with 3 btags has not been used).}
The limits include all experimental 
uncertainties.   The JES and btag uncertainties are calculated point-by-point.


\begin{figure}[htb]
\begin{center}
\includegraphics[width=0.65\linewidth]{figs/sbottom_regions.pdf}
\caption{The signal region with the best expected limit as a function of 
$m(\chi^{\pm}))$ vs. $m(\widetilde{b})$ plane for $m(\chi^0_1)$=XX GeV.
{\bf What is the LSP mass???}
The coding is: 1=(80,30),
2=(200-50), 3=(200-150), 4=(320-50), and 5=(320-120), where
the first (second) number is the $H_T$ (\met) threshold in GeV. The number
of requested btags is 2 or more.
\label{fig:sbottomoptimize}}
\end{center}
\end{figure}


\subsubsection{Limits for the sbottom pair production model}
\label{sec:sbottompairlimits}
The limit on the production cross-section in this model in the 
sbottom mass vs. $\chi^{\pm}$ mass plane shown in 
Figure~\ref{fig:sbottomLimit}.  {\bf For what alue of LSP mass?}
Using the 
NLO$+$NLL cross-section for gluino pair production, we place a limit
on the mass parameters as shown in Figure~\ref{fig:sbottomLimit}.

\begin{figure}[htb]
\begin{center}
\includegraphics[width=0.65\linewidth]{figs/sbottom_limit.pdf}
\caption{Cross section limits in the $m(\widetilde{b})$ vs. $m(\chi^{\pm})$ 
plane for the sbottom pair production model. {\bf What is the LSP mass?}
\label{fig:sbottomLimit}}
\end{center}
\end{figure}





%\subsubsection{What is missing for the sbottom pair production model}
%\begin{itemize}
%\item Everything in Sections~\ref{sec:sbottompairdefinition} and \ref{sec:sbottompairlimits}
%\item It would be nice to have a reference.  I am not sure that the references that
%we have on our twiki are appropriate. 
%\item Perhaps more details on the MC signal generation
%\end{itemize}


\subsection{$\widetilde{g} \to \widetilde{b}\bar{b}$ Model}
\label{sec:gbb}
This model is mostly gluino pair production followed by 
$\widetilde{g} \to \widetilde{b}\bar{b}$, $\widetilde{b} \to t \chi^{-}$ and
$\chi^{-} \to W^- \chi_1^0$. 
The final state is $t\bar{t}b\bar{b}W^+W^- \chi_1^0 \chi_1^0$
or $ttb\bar{b}W^+W^- \chi_1^0 \chi_1^0$ $(+ c.c.)$.
The model also includes the $b g \to \widetilde{b} \widetilde{g}$ process,
in which case the final state is
$tb\bar{b}W^+W^- \chi_1^0 \chi_1^0$ $(+ c.c.)$. 
The model parameters are $M(\widetilde{g})$
$M(\widetilde{b})$, $M(\chi_1^0)$, and $M(\chi^{\pm})$.
For simplicity we only consider mass parameters such that the $\chi^{-}$ is on shell.

\subsubsection{Signal region definition for the $\widetilde{g} \to \widetilde{b}\bar{b}$ Model}
\label{sec:gbbdefinition}

\begin{figure}[htb]
\begin{center}
\includegraphics[width=0.65\linewidth]{figs/gl_sb_300_50_regions.pdf}
\caption{The signal region with the best expected limit as a function of 
$m(\widetilde{g}$ vs. $m(\widetilde{b})$ for $m(\chi^0_1)$=50 GeV
and $m(\chi^{\pm})$=200 GeV. {\bf Are these parameters right?}
The coding is: 1=(200-50), 2=(200-150), 3=(320-50), and 4=(320-120), where
the first (second) number is the $H_T$ (\met) threshold in GeV. The number
of requested btags is 2 or more.
\label{fig:gluinosboptimize}}
\end{center}
\end{figure}


For each point in parameter space we use the signal region that gives
the best expected limit.  
{\bf (Note: so far the region with 3 btags has not been used).}
Limits are calculated using all experimental
uncertainties; the JES and btag uncertainties are calculated point-by-point.
An example of this optimization is shown in Figure~\ref{fig:gluinosboptimize},
where we show the choice of signal region that gives the best expected limit
in the $m(\widetilde{g})$ vs. $m(\widetilde{b})$ plane for the choice
$m(\chi^0_1)$=50 GeV and $m(\chi^{\pm})$=200 GeV. 
{\bf Are these parameters right?}



\subsubsection{Limits for the $\widetilde{g} \to \widetilde{b}\bar{b}$ Model}
\label{sec:gbblimits}

The limits on the production cross-section in this model in the 
gluino mass vs. sbottom mass plane for two choices of the 
chargino mass and an LSP mass of 50 GeV 
are shown in Figure~\ref{fig:mglinoSbottom}
Using the 
NLO$+$NLL cross-section for gluino pair production, we also place a limit
on the mass parameters of this model.


\begin{figure}[htb]
\begin{center}
\includegraphics[width=0.47\linewidth]{figs/gl_sb_300_50.pdf}
\includegraphics[width=0.47\linewidth]{figs/gl_sb_200_50.pdf}
\caption{Cross section limits in the $m(\widetilde{g})$ vs. 
$m(\widetilde{b})$ plane
for $m(\chi_1^0)$ = 50 GeV and 
$m(\chi^{\pm})$ = 300 GeV (left) and 200 GeV (right). 
\label{fig:mglinoSbottom}}
\end{center}
\end{figure}

%\subsubsection{What is missing for the $\widetilde{g} \to \widetilde{b}\bar{b}$ Model}
%\begin{itemize}
%\item Everything in Sections~\ref{sec:gbbdefinition} and \ref{sec:gbblimits}
%\item It would be nice to have a reference.  I am not sure that the 
%references that
%we have on our twiki are appropriate. 
%\item Perhaps more details on the MC signal generation
%\end{itemize}

