\section{Introduction}
\label{sec:intro}

In this note, we present the estimates of CMS sensitivity to SUSY for 95\% C.L. limits as well as $5\sigma$ discover reach
in the mSUGRA framework with R-parity conservation at a proton proton center of mass of 7TeV. This model is characterized by five
free parameters described as follows:

\begin{itemize}
\item $m_{0}$: the common scalar mass at the GUT scale;
\item $m_{1/2}$: the common gaugino mass at the GUT scale;
\item $A_{0}$: the common soft trilinear SUSY breaking parameter at the GUT scale;
\item tan$\beta$: the ratio of the Higgs vacuum expectation values at the electroweak scale;
\item sign $\mu$: the sign of the Higgsino mass term.
\end{itemize}

 We set $A_{0} = 0$, sign$\mu > 0$ and tan$\beta = 3$ in order to be able to directly
compare with the recent Tevatron results\cite{cdf}\cite{dzero}. The gluino-squark mass plane is then scanned 
via variations of $m_{0}$ and $m_{1/2}$ parameters. In this framework, all supersymmetric particles
except the neutralino are unstable and thus will decay into their SM counterparts right
after being produced. This cascade decay will result in dilepton final states associated 
with several jets, plus missing transverse energy (\met~) from the LSP.

The note is organized as follows. In Section~\ref{sec:datasamples} we list the Monte Carlo data samples, as well as the 
software tags used in this analysis. In Section~\ref{sec:eventselection} we describe the same (SS) and opposite (OS) sign 
dilepton event selection used in this study. The statistical procedure used for exclusion as well as 
the discovery potential is summarized in Section~\ref{sec:mSUGRA}. In Section~\ref{sec:samesign} we discuss the exclusion limits and
mass reach using SS dileptons followed by similar study involving OS dileptons in Section~\ref{sec:osstudies}. Finally, in 
Section~\ref{sec:conclusion} we summarize the results.

The work presented here updates work previously documented in\cite{osnote} and \cite{ssnote} from 10TeV center of mass to 7TeV,
and from 2 series to 3 series MC.
 
 



