\section{Same Sign Dileptons}
\label{sec:samesign}

This section summarizes the results of the SUSY parameter space scan
in the same sign dilepton channel. The measurement technique is
described in detail in a CMS note~\cite{ssnote}. The technique
utilizes a data-driven method to estimate background characterized
by the presence of two high $P_T$, isolated, same sign leptons,
$\met$, and significant jet activity. This generic signature is
sensitive to many new physics scenarios such as SUSY.  For the purposes
of this note we restrict ourselves to the $ee$, $e\mu$, and $\mu\mu$
final states, {\em i.e.}, we do not consider $\tau$'s, except in the
case that the $\tau$ decays leptonically. 

As we will show in Section~\ref{sec:ssyields}, for a reasonable event
selection the main background is \ttbar decays. The data-driven
background prediction is based on a 
combination of estimating ``fake leptons''\cite{fakenote} and 
electrons reconstructed with the wrong sign\cite{ssnote}. The probability
for muons to be reconstructed with the wrong sign is so small at the relevant momenta
that it is negligible.

\subsection{Event Yields}
\label{sec:ssyields}

The expected event yields in 100~pb$^{-1}$ after applying the event selections
described in Section~\ref{sec:eventselection} to the data sets described in
Section~\ref{sec:datasamples} are detailed below: the SM yields are listed in
Table~\ref{tab:ssyields}, and the mSUGRA scan point yields are illustrated in
Fig.~\ref{fig:hss_100pb}.

\begin{table}[hbt]
\begin{center}
\begin{tabular}{|l|c|c|c|c|}\hline
Sample           & expected bkg yields   \\ \hline
$t\overline{t}$  &   $0.36\pm 0.10 $         \\ 
$WW$             &   0.0          \\ 
$WZ$             &   $0.02\pm 0.01$          \\ 
$ZZ$             &   0.0          \\ 
$W$+jets         &   0.0             \\
$Z$+jets         &   0.0             \\ 
Single top       &   0.0          \\ \hline
\end{tabular}
\caption{Expected SM event yields in 100~pb$^{-1}$. Errors are MC statistics only.\label{tab:ssyields}}
\end{center}
\end{table}

\subsection{Procedure for Determining $5\sigma$ Discovery Reach}
\label{sec:significance}

We  determine the  $5\sigma$  discovery reach  in the  $m_{0}-m_{1/2}$
plane by  performing our analysis at  each of the  mSUGRA scan points.
For each  point, we  determine expected yield based on the number of events passing all cuts
scaled by the LO cross section for that point assuming 100/pb or 1/fb of data respectively.
We then in addition, perform our data driven background estimation procedure at each point
to determine the ``signal contamination'' at that point. We find the latter to be small,
less than 10\% of the expected signal yield. 

We quantify the
significance  of  the  discrepancy  between  the  observed  yield  and
predicted   background  yield   using  two   significance  estimators:
$Z_{Bi}$~\cite{cite:cousins}    and    $Z_N$~\cite{cite:conway}.   The
quantities required to calculate these estimators are:

\begin{itemize}
\item The predicted background yield
\item The relative systematic  uncertainty on the predicted background
yield (set to 50\%)
\item The  statistical uncertainty  on the predicted  background yield
($Z_N$ only, set to 0)
\item The observed yield
\end{itemize}

\subsection{Procedure for Excluding a Region of the mSUGRA Parameter Space}
\label{sec:exclusion}

Next we  determine the region of  the mSUGRA parameter  space which we
expect  to exclude  at 95\%  confidence level  (CL) if  we see
the standard model (SM) expected yields in data. 
We  assume  that  we find  the  same
predicted background yield  and observed yield in data  that we expect
to  find based on  our SM  MC. 
However, as the latter is 0.4,
we work the math for both an observation of 0 or 1 events in 100/pb.
We  use this  information to  exclude a
subset of the mSUGRA points using the following procedure.

The first  step is to  determine the 95\%  CL upper limit (UL)  on the
signal yield using a Bayesian  method from John Conway, implemented in
the program bayes.f. The required  inputs are: the observed yield, the
relative  uncertainty in  the  signal acceptance  (set  to 15\%),  the
predicted  background yield,  and  the total  error  on the  predicted
background yield. We assume $0.4 \pm 0.6$ for background yield and its uncertainty.

These values lead to 
to  95\% CL  ULs  of 3.2 and 4.7  signal  events respectively for an assumed observation of 0 or 1 events.
%It should  be noted that
%bayes.f assumes a Gaussian  error distribution, while for our analysis
%(especially  at 100~pb$^{-1}$ where  the background  yield is  4), our
%errors are Poisson-distributed.

Next, we  wish to exclude mSUGRA  points based on the  signal yield UL
derived above.   The most obvious  way to do  so is to  exclude points
which  lead  to a  difference  between  observed  yield and  predicted
background yield  which exceeds the  UL on the signal  yield. 
As the effect of signal contamination is small, we stick to this simplest 
of possible ways.

