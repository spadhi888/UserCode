\section{Event Selection}
\label{sec:eventselection}

The event selection used is not optimized for any specific SUSY
scenario. It is based on small modifications to the di-lepton event
selections  that we used in the approved $WW$\cite{ww} and
\ttbar\cite{ttbar} cross section analyses.  The OS and SS analyses
summarized here are documented in~\cite{osnote} and~\cite{ssnote}. The
only difference to those analysis notes is that we now use the 3
series MC at 7 TeV instead of the 2 series MC at 10 TeV.  
A quick summary of the event selection is:

\begin{itemize}
\item We require inclusive lepton triggers with no isolation, $i.e.$, the
  logical OR of {\tt HLT\_Ele15\_SW\_L1R} and {\tt HLT\_Mu9}. 
  The combined trigger efficiency is $\sim 99$\% for di-lepton events that pass the event selection.
\item Two isolated, same or opposite sign leptons ($ee$, $e\mu$, and $\mu\mu$). 
\item Leptons must have $P_T > 10$ GeV, $|\eta|< 2.4$ and at least one of them must have $P_T > 20$ GeV.
\item We consider L2L3 corrected caloJets with $P_T > 30$ GeV and
	$|\eta|< 2.4$ for both analyses. 
\item The scalar sum of the $P_T$ of all jets passing the requirements above should be $>$ 200 GeV.
\item For SS analysis:
\begin{itemize}
      \item  we veto the candidate lepton, if an extra lepton in the event pairs with the candidate lepton
             to form a $Z$ within the mass range between $76 < m_{\ell\ell} $ (GeV) $< 106$. This requirement is 
             designed to reject $WZ$ events.
      \item At least three jets.
      \item We require \met~$>$ 80 GeV.
\end{itemize}
\item For the OS analysis: 
\begin{itemize}
      \item We remove $ee$ and $\mu\mu$ pairs consistent with a $Z$ by requiring mass($\ell\ell$) $<$ 76 GeV or mass($\ell\ell$) $>$ 106 GeV.
      \item At least two jets.
      \item We have a general requirement that $\met>$~50 GeV. We additionally define a ``tight'' \met~requirement of \met~$>$ 175 GeV
	that is used for reporting predicted and observed event yields.
	This latter cut is intended to allow just a few SM events to
	pass in 100 pb$^{-1}$.  For \met, we use tcMET \cite{tcmet}
	corrected for $\mu$.
\end{itemize}
\end{itemize}
\noindent More details on the lepton %and trigger 
selections are given below.

\subsection{Electron Selection}
\label{sec:electron}

\begin{itemize}
\item In our corresponding 10 TeV analyses with 2 series MC, we used ``e-gamma category based tight'' for electron ID.
      An exact equivalent does not exist in the 3 series, and we thus tried out a variety of different options, including ``e-gamma category based looseID''.
      We find differences in efficiency and background at the 10\% level, and consider them negligible for the purpose of this study.
\item No muon candidate within $\Delta R < 0.1$.
\item $|d_0| < 200~\mu m$ (corrected for beamspot).
\item Iso $<$ 0.1, where Iso=Sum/Max(20 GeV, $P_T$), and Sum = tkIso + hcalIso +  Max(0 GeV, ecalIso - 2GeV).
All isolation sums are the standard sums used in release 3\_1\_6 from the egamma group (cone of
0.4 for ecal, jurassic, rec-hit based; cone of 0.3 for tracker, and cone of 0.4 for hcal).
\item Conversion rejection~\cite{conversionnote} using tracks within a cone of 0.3 of the candidate electron for SS studies: 
\begin{itemize}
\item $|\Delta \cot\theta| < 0.02$; the difference between cotangent polar angles of tracks parallel to 
each other.
\item $|d_{2d}| < 0.02$ cm; the two dimensional distance between points within nearest tracks.
\end{itemize} 
\item For the SS analysis, the charge of the associated GSF and CTF tracks must be consistent.
If the CTF track is not reconstructed, the electron is kept.
\end{itemize}

\subsection{Muon Selection}
\label{sec:muon}
\begin{itemize}
\item Must be a global muon {\bf and} a tracker muon~\cite{glbtrk}.
\item GlobalMuonPromptTight (global $\chi^2$/ndof$<$10)~\cite{muonid}.
\item At least 11 valid hits for the silicon track~\cite{muonid}.
\item $|d_0| < 200~\mu m$ (from silicon track, corrected for beamspot).
\item Global fits must have hits in the muon chambers.
\item Minimum ionizing: EcalVetoEnergy $<$ 4 GeV and HcalVetoEnergy $<6$ GeV~\cite{vplusj}. 
\item Iso $<$ 0.1, where Iso=Sum/Max(20 GeV, $P_T$), and Sum = tkIso + hcalIso +  ecalIso.
All isolation sums are the standard sums stored in the muon object in release 3\_1\_X, and
are calculated in a cone of 0.3.
\end{itemize}

%\subsection{Trigger Selection}
%\label{sec:trigger}
%We use inclusive lepton triggers with no isolation, $i.e.$, the logical OR of {\tt HLT\_Ele15\_SW\_L1R} and {\tt HLT\_Mu9}.  
%The combined trigger efficiency is $\sim 99$\% for di-lepton events that pass the event selection.
%These triggers are expected to be present in the data taking trigger table.
