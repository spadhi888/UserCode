\documentclass{cmspaper}
\usepackage{graphicx}
\usepackage{rotate}
\usepackage{rotating}
\usepackage{relsize}
\usepackage{lineno}
\usepackage{slashed}
\usepackage{slashbox}
\usepackage{url}
\newcommand{\ptll} {\ensuremath{P_T(\ell\ell)}}
\newcommand{\ptllres} {\ensuremath{P^{\rm res}_T(\ell\ell)}}
\def\ack{\section*{Acknowledgments}}

\linenumbers

\begin{document}

%==============================================================================
% title page for many authors
%
\begin{titlepage}
\title{Higher Order SUSY Cross Sections and associated uncertainties}

  \begin{Authlist}
  Anna Kulesza
 \Instfoot{rwth}{ Institut für Theoretische Teilchenphysik und Kosmologie, RWTH Aachen,   Germany}
 Michael Kraemer
\Instfoot{rwth}{ Institut für Theoretische Teilchenphysik und Kosmologie, RWTH Aachen, Germany}
Sanjay Padhi
 \Instfoot{ucsd}{University of California, San Diego, USA}
Tilman Plehn
 \Instfoot{heidelburg}{Institut für Theoretische Physik, Heidelberg, Germany}
Xavier Portell 
 \Instfoot{cern}{European Organization for Nuclear Research, CERN, Switzerland}

 \end{Authlist}

\begin{abstract}
We summarize the state-of-art higher order cross section for several SUSY processes in the framework of CMSSM as well as parton parton corrections for 
various SUSY sub-processes. The NLO+NLL cross sections are provided for the production of squarks, gluinos, stops and sbottoms with respective theoretical 
errors from scale and PDF choices. NLO corrections are used for electroweak processes involving neutralinos, charginos, and sleptons the theory errors. 

\end{abstract}
\end{titlepage}

\clearpage
\section{Introduction}

\section{Higher Order calculations - NLO+NLL}

\section{Cross sections for SUSY colored particles}

\section{Theory uncertanities}



\appendix

\end{document}
