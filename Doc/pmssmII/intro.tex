\section{Introduction}
\label{sec:intro}

After the successful operation of the Large Hadron Collider (LHC) and the CMS detector
in 2010 and 2011, and with good prospects for the future, the LHC is now ready to shed light on a number 
of open questions in Particle Physics 
such as the mechanism of electroweak (EW) symmetry breaking, or the 
new physics, Beyond the Standard Model (BSM), that stabilizes the EW scale. 

A wealth of theories that extend the Standard Model have been put forth during the past decades. Supersymmetry (SUSY) is
arguably the best motivated BSM theory --- and certainly the most 
thoroughly studied. 
Indeed, searches for SUSY are among the primary objectives of the 
CMS experiment. SUSY is exceedingly popular not 
only for its theoretical beauty but also because SUSY phenomenology 
is extremely rich, 
%in fact is can mimic almost any other new physics scenario. 
leading to a large variety of possible new signals at the LHC. 
In spite of this, the majority of SUSY studies focus on a very special 
setup: the so-called Constrained Minimal Supersymmetric Standard Model (CMSSM). 
This was justified in the preparation for discoveries as the CMSSM, 
having just a handful of new parameters, is very predicive. However, 
the simplifying assumption of universality at the GUT scale lacks a sound 
theoretical motivation. Consequently, the CMSSM should be regarded as a showcase 
model. When it comes to interpreting experimental results, it is reasonable and interesting to do this within the CMSSM because it 
provides (to some degree) an easy way to show performances, 
compare limits or reaches, etc. However, the interpretation of experimental results in the 
$(m_0,m_{1/2})$ plane risks imposing unwarranted constraints on SUSY, as many 
mass patterns and signatures that are possible a priori are not covered in the CMSSM. 
The same problem arises in any analysis that assumes a particular 
SUSY breaking scheme. 

In this document, we therefore introduce a different approach, which uses only 
minimal assumptions on the underlying SUSY parameters. In particular, given the absence of experimental guidance, we choose
not to rely on a particular SUSY breaking scheme.
Instead, we use a 19-dimensional 
parametrization of the MSSM, called the \emph{phenomenological MSSM} (pMSSM),
with parameters defined not at the GUT scale but instead at the SUSY scale 
(by convention the geometric mean of the two stop masses).
We demonstrate the feasibility of our approach by applying it to 
the 2011 CMS data-set corresponding to 1~fb$^{-1}$ of integrated luminosity.  
Using profile likelihoods, we combine 
the dijet $\alpha_T$ analysis, the opposite-sign dilepton 
analysis and the same-sign dilepton analysis and derive constraints 
on the SUSY particles with as few simplifying assumptions as possible.
Results from other SUSY analyses in CMS will be added as soon as they become available.

We first give the motivation to go beyond the CMSSM and work in 
a generic MSSM setup. After this, the pMSSM and its parametrization is defined. 
We then outline our analysis, giving details on the pMSSM points we have used, 
the detector simulation and the CMS analyses, and describe the statistical method based on 
profile likelihoods used for coping with the 19-dimensional model. Finally, we discuss our results and summarize our conclusions.

