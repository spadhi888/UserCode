\section{Results}
\label{sec:results}

We present the results obtained by following the analysis path in Section~\ref{sec:analysis}, in terms of  plots of likelihood ratios $L_p / L_{max}$ as a function of each parameter of interest.  Further details on this ratio as a  measure can be found in the Appendix.  

Plots of the likelihood ratio $L_p/L_{max}$ are shown for the 19 input pMSSM parameters in Figures~\ref{fig:LRwcms_msq} to~\ref{fig:LRwcms_tbmu}.  Similarly Figures~\ref{fig:LRwcms_sq} to~\ref{fig:LRwcms_Higgs} show the likelihood ratio for the physical sparticle masses.  The relationships between the scalar SUSY breaking mass parameters and the physical scalar masses as well as those between the gaugino mass parameters and the physical gaugino masses should be noted.  The colored and shaded histograms in each plot depict the likelihood ratio before and after the inclusion of CMS results, respectively.  We observe that the CMS results indeed introduce a variation in the likelihood ratios, where the variation is more enhanced for some parameters/masses and milder for others.  

We see that all squark/slepton mass parameters and all squark/slepton masses shift upwards systematically after the addition of CMS results.  The gaugino mass parameter $M_3$ and the gluino mass also move up simultaneously due to the constraints coming from the di-jet $\alpha_T$ analysis.  An important aspect to note is that no significant change is observed in the distribution for the mass of $\tilde{\chi}^0_1$, which is the lightest supersymmetric particle (LSP), since neither a stringent $E^T_{miss}$ cut nor any other technique dedicated to constraining the $\tilde{\chi}^0_1$ mass were imposed in the analyses considered.  This is an interesting example of
the freedom permitted by the pMSSM parameterization, in which neutralino/chargino masses can vary independently of the gluino mass.  Had the interpretation been done using 
the CMSSM, the behavior of the mass parameters would be constrained by the strict gaugino mass relationship described in Section~\ref{sec:motivation}, which forces the $\tilde{\chi}^0_1$ mass to move upwards along with the gluino mass.

Figure~\ref{fig:LRwcms_omg} shows $L_p/L_{max}$ for the dark matter relic density calculated using {\tt micrOMEGAs 2.4}~\cite{Belanger:2006is} assuming $\tilde{\chi}^0_1$ is the LSP and the dark matter candidate.  It must be noted that Berger {\it et al.} imposed the WMAP upper limit $\Omega_{\tilde{\chi}^0_1}h^2 \le 0.1210$ as a constraint on the points.  Information from CMS does modify the distribution, however the effect is not sufficient to impose a concrete constraint on $\Omega_{\tilde{\chi}^0_1}h^2$.  This, of course, is related to the result on the $\chi^0_1$ mass.

Finally, Figures~\ref{fig:LRwcms_EWobs_s1} and~\ref{fig:LRwcms_EWobs_s2} show distributions for low energy observables as predicted by the pMSSM, calculated using {\tt micrOMEGAs 2.4} and {\tt Superiso 2.7}~\cite{Mahmoudi:2008tp}.  We note, again, that the selected pMSSM points satisfy the constraints, based on experimental measurements for a subset of these observables, that were imposed by Berger {\it et al.} Similarly, as we found for the relic density, the 2010 CMS measurements do not yet allow us to constrain these observables further.  However, the prospect of making quantitative statements about the predictions for low energy observables, given the anticipated data set and diverse analysis channels, is a strong motivation for interpreting the data within the framework of supersymmetry.

The results we have presented in this Note, with only 35 pb$^{-1}$ of CMS data, show that even with a modest amount of data, we are able to start making inferences of a general nature
about supersymmetry. 
%on a sufficiently generic and well-motivated construction of supersymmetry.  
Therefore, we expect with (at least) an order of magnitude more data, the approach we have developed will allow us to make definitive statements about a broad class of supersymmetric models.  But, this is only the starting point in a vast landscape of investigation. In the near term, 
we will develop this approach further by considering the following:

\begin{itemize}
\item An improved treatment of the constraints from the EW observables.  We shall replace the box-like likelihood of Berger {\it et al.} by a more appropriate likelihood function.
\item A significant increase in the sample of pMSSM points. 
\item Inclusion of a wider range of final states.
\item The development of a more quantitative treatment of the bounds on the pMSSM parameter space.
\item The inclusion of experimental uncertainties and, if feasible, theoretical uncertainties.
\end{itemize}




