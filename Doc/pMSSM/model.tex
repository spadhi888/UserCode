\section{Phenomenological MSSM (pMSSM)}
\label{sec:model}

As mentioned, the pMSSM is a 19-dimensional realization of the MSSM 
with parameters defined at the so-called SUSY scale\footnote{Sometimes 
also referred to as the scale of electroweak symmetry breaking, $M_{\rm EWSB}$.} 
$M_{\rm SUSY}=\sqrt{m_{\tilde t_1}m_{\tilde t_2}}$. How do we come to 
this number of parameters and what are its implications?

In its most generic form, just assuming R-parity conservation, 
the MSSM has over 120 free parameters: 
SUSY-breaking mass terms and trilinear couplings, Yukawa matrices, 
CP phases, two VEVs, and a superpotential $\mu$ term. 
Model builders construct theoretically-motivated economic models 
that provide relations between these parameters at some high scale, 
{\it e.g.}, the GUT scale. 
These then evolved to $M_{\rm SUSY}$ by renormalization group equations 
to derive model-specific testable predictions. On the other 
hand, it is expected that once SUSY particles are discovered, measurements 
of their masses and interactions will allow to reconstruct (at least 
part of) the Lagrange parameters and thus infer the underlying SUSY 
breaking mechanism.
 
Luckily not all of the $\sim$120 MSSM parameters are of equal relevance to this end.
In fact, a couple of reasonable assumptions motivated by experiment serve to 
simplify the problem a lot. In particular, it is reasonable to restrict ourselves 
to the CP-conserving MSSM ({\it i.e.}\ no new CP phases) with minimal flavor violation (MFV).  
Constraints from the flavor sector moreover suggest that the first two generations 
of sfermions be taken to be degenerate. Regarding Yukawa and trilinear couplings, 
only those of the third generation matter at leading order. 
Finally, we assume that the gravitino is heavy\footnote{To be completely honest, 
this implies after all some assumption on the SUSY breaking mechanism; for instance it 
excludes low-scale gauge mediation.} and the lightest supersymmetric 
particle (LSP) is the lightest neutralino, $\tilde\chi^0_1$.
 
This leaves us with the 19 real, weak-scale SUSY Lagrange parameters 
that define the pMSSM:
\begin{itemize}
   \item 3 gaugino masses $M_1$, $M_2$, and $M_3$ 
         (pertaining to U(1), SU(2), and SU(3) gauginos, respectively);         
   \item the higgsino mass parameter $\mu$;
   \item the ratio of the Higgs VEVs $\tan\beta=v_2/v_1$;
   \item the pseudo-scalar Higgs mass $m_A$;\footnote{The parameters $\mu$ and $m_A$  
         can be swapped for the Higgs mass parameters $m_{H_1}^2$ and $m_{H_2}^2$.}
   \item 10 sfermion mass parameters $m_{\tilde{F}}$, where 
         $\tilde{F} = \tilde{Q}_1, \tilde{U}_1, \tilde{D}_1, 
                      \tilde{L}_1, \tilde{E}_1, 
                      \tilde{Q}_3, \tilde{U}_3, \tilde{D}_3, 
                      \tilde{L}_3$, and $\tilde{E}_3$ \\ 
(recall that $m_{\tilde{Q}_1}\equiv m_{\tilde{Q}_2}$, 
           $m_{\tilde{L}_1}\equiv m_{\tilde{L}_2}$, etc.); and          
   \item 3 trilinear couplings $A_t$, $A_b$ and $A_\tau$                  
\end{itemize}
 
In \cite{Berger:2008cq}, Berger {\it et al.}\ performed a scan of the pMSSM parameters  
to find out what regions of parameter space are consistent with theoretical and experimental constrains. In particular they did a uniform random sampling of 
points from within the following ranges:\footnote{Here note that $M_1$, $M_2$, 
$\mu$, $A_t$, $A_b$ and $A_\tau$ can have 
arbitrary signs; $M_3$ was chosen to be positive because only six 
of all possible sign combinations of $M_i$, $A_i$, $\mu$ are physical.} 
\begin{eqnarray}
  \mbox{100 GeV} \leq 	& m_{\tilde{F}}  & \leq \mbox{1000 GeV},   \nonumber \\
  \mbox{50 GeV}  \leq	& |M_{1,2},\mu|  & \leq \mbox{1000 GeV}, 	 \nonumber \\
  \mbox{100 GeV} \leq	& M_3 	     & \leq \mbox{1000 GeV},   \\
	                  & |A_{t,b,\tau}| & \leq \mbox{1000 GeV}, 	 \nonumber \\
    1  \leq			& \tan\beta		& \leq 50,	  \nonumber \\
    \mbox{43.5 GeV}	\leq	& m_A			& \leq \mbox{1000 GeV},  \nonumber 
\end{eqnarray}

Imposing SUSY and Higgs mass limits and requiring consistency with low-energy 
constraints and with the dark matter relic density, 
they found \cite{Berger:2008cq} that 
{\it ``the pMSSM leads to a much broader set of predictions for the properties 
of the SUSY partners as well as for a number of experimental observables than 
those found in any of the conventional SUSY breaking scenarios such as mSUGRA [CMSSM]. This set of models can easily lead to atypical expectations for SUSY signals 
at the LHC.''}

As a first step towards a full pMSSM analysis, we use a 6K subset of the 
pMSSM points generated by Berger {\it et al.}\ and investigate to which 
extent CMS SUSY analyses with 35~pb$^{-1}$ of data already constrain 
this sample. 
