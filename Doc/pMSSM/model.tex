\section{Phenomenological MSSM (pMSSM)}

Phenomenological MSSM is a model that makes no assumptions on the SUSY breaking mechanism.  It is parametrized at the so-called "SUSY scale (i.e. the geometric mean of the two stop masses).  At this low scale, SUSY is defined by over 120 parameters at its most generic form.  However simplifying assumptions can be made without loosing much from the generality of the scenario.  pMSSM construction stays within the CP-conserving MSSM (i.e. no new phases) with minimal flavor violation.  Moreover, to help soften the impact of experimental constraints arising from the flavor sector, the first two generations of sfermions are taken to be degenerate.  This results in the following 19-dimensional parameterization:

\begin{enumerate}
         \item 10 scalar masses: $m_{\tilde{f}}$ (where 
$\tilde{f} = \tilde{Q}_L, \tilde{Q}_3, \tilde{L}_1, \tilde{L}_3, 
\tilde{u}_1, \tilde{d}_1, 
\tilde{u}_3, \tilde{d}_3, \tilde{e}_1$, and $\tilde{e}_3$),

         \item 3 gaugino masses: $M_{1,2,3}$ (pertaining to U(1), SU(2), 
and SU(3), respectively),
         
         \item 1 $\tan\beta$ parameter,
         
         \item 3 trilinear couplings: $A_{b, t, \tau}$,
         
         \item 1 pseudo-scalar Higgs mass: $m_A$.
         
         \item 1 $\mu$ parameter. 
         
\end{enumerate}
The parameters $m_A$ and $\mu$ can be swapped for the Higgs mass parameters 
$m_{H_u}$ and $m_{H_d}$.

Berger et. al. performed a multi-dimensional scan to find out what regions in the pMSSM parameter space are consistent with theoretical and experimental constrains.  They did a uniform random sampling of points from within the pMSSM subspace defined by the parameter ranges below
\begin{eqnarray}
  \mbox{100 GeV} \leq 	& m_{\tilde{f}} 	& \leq \mbox{1000 GeV}, 	
  \\ \nonumber
  \mbox{50 GeV}  \leq	& |M_{1,2},\mu| & \leq \mbox{1000 GeV}, 	
    \\ \nonumber
    \mbox{100 GeV} \leq	& M_3 		& \leq  \mbox{1000 GeV},
    \\ \nonumber
	& |A_{b,t,\tau}| & \leq \mbox{1000 GeV}, 		
    \\ \nonumber
    1  \leq				& \tan\beta		& \leq 50,				
    \\ \nonumber
    \mbox{43.5 GeV}	\leq	& m_A			& \leq \mbox{1000 GeV},
\end{eqnarray}


