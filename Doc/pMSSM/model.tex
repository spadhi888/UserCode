\section{Phenomenological MSSM (pMSSM)}

In the parameterization  of Berger \emph{et al.}~\cite{pMSSM1}, 
the  19-parameter 
phenomenological MSSM (pMSSM) is defined at the SUSY scale by

\begin{enumerate}
         \item 10 scalar masses: $m_{\tilde{f}}$ (where 
$\tilde{f} = \tilde{Q}_L, \tilde{Q}_3, \tilde{L}_1, \tilde{L}_3, 
\tilde{u}_1, \tilde{d}_1, 
\tilde{u}_3, \tilde{d}_3, \tilde{e}_1$, and $\tilde{e}_3$),

         \item 3 gaugino masses: $M_{1,2,3}$ (pertaining to U(1), SU(2), 
and SU(3), respectively),
         
         \item 1 $\tan\beta$ parameter,
         
         \item 3 trilinear couplings: $A_{b, t, \tau}$,
         
         \item 1 pseudo-scalar Higgs mass: $m_A$.
         
         \item 1 $\mu$ parameter. 
         
\end{enumerate}
The parameters $m_A$ and $\mu$ can be swapped for the Higgs mass parameters 
$m_{H_u}$ and $m_{H_d}$.
The constrained MSSM (cMSSM), perhaps
the most widely studied sub-model of the MSSM,
is defined by the 5 parameters,

\begin{enumerate}
         \item 1 universal scalar mass: $m_0 = m_{\tilde{f}}, 
m_{H_u}, m_{H_d}$,
    
         \item 1 universal gaugino mass: $m_{1/2} = M_{1,2,3}$,
         
         \item 1 $\tan\beta$ parameter,
         
         \item 1 universal trilinear coupling: $A_0 = A_{b, t, \tau}$,
         
         \item sign of the $\mu$ parameter: sgn($\mu$).
\end{enumerate}
Geometrically, the cMSSM is confined to 
the cartesian product of the
diagonal lines in the 4 pMSSM sub-spaces
\begin{eqnarray} 
(m_{\tilde{Q}_L}, m_{\tilde{Q}_3}, m_{\tilde{L}_1}, 
m_{\tilde{L}_3}, m_{\tilde{u}_1}, 
m_{\tilde{d}_1}, 
m_{\tilde{u}_3}, 
m_{\tilde{d}_3}, 
m_{\tilde{e}_1}, 
m_{\tilde{e}_3},  
m_{H_u}, 
m_{H_d}), \\ \nonumber
(M_1, 
M_2, 
M_3), \\ \nonumber
(A_b, A_t, A_{\tau}), \\ \nonumber 
(\tan\beta),
\end{eqnarray}
with one such space for each sign of $\mu$. 


The pMSSM sub-space
\begin{eqnarray}
  \mbox{100 GeV} \leq 	& m_{\tilde{f}} 	& \leq \mbox{1000 GeV}, 	
  \\ \nonumber
  \mbox{50 GeV}  \leq	& |M_{1,2},\mu| & \leq \mbox{1000 GeV}, 	
    \\ \nonumber
    \mbox{100 GeV} \leq	& M_3 		& \leq  \mbox{1000 GeV},
    \\ \nonumber
	& |A_{b,t,\tau}| & \leq \mbox{1000 GeV}, 		
    \\ \nonumber
    1					& \tan\beta		& \leq 50,				
    \\ \nonumber
    \mbox{43.5 GeV}	\leq	& m_A			& \leq \mbox{1000 GeV},
\end{eqnarray}
was recently sampled \emph{uniformly} by Berger \emph{et al.}~\cite{pMSSM1} 
keeping only those
points that are broadly consistent with current data. Such
scans can of course be done much more efficiently using MCMC.
Berger \emph{et al.}'s scan is equivalent to sampling from the
likelihood function of the data they used. 
(In principle, their uniform sampling  makes it straightforward
to study the effect of non-flat
priors simply by 
weighting each point by the desired prior at that point.) 

\newpage
In addition to the 19 parameters of the pMSSM, the model includes an
additional $\sim 19$ parameters that define the standard
model, as shown in the figure below.
%\begin{figure}[htbp]
%\begin{center}
%\includegraphics[width=0.7\textwidth,clip=]{figs/SMTable}
%\caption{
%Reproduction of Table 1 from Ref.~\cite{pMSSM1} showing the standard model
%parameter values used in the study of Berger \emph{et al.}~\cite{pMSSM1}.}
%\end{center}
%\end{figure}

\bigskip
\bigskip
