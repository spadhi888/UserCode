\documentclass{cmspaper}
\usepackage{graphicx}
\usepackage{rotate}
\usepackage{relsize}
\usepackage{amsmath}
\usepackage{amssymb}
\usepackage{cite}

\newcommand{\met} {\ensuremath{E\!\!\!\!/_T}}
\newcommand{\ttbar} {\ensuremath{t\bar{t}~}}
\newcommand{\ptll} {\ensuremath{P_T(\ell\ell)}}
\newcommand{\ptllres} {\ensuremath{P^{\rm res}_T(\ell\ell)}}

\begin{document}

%==============================================================================
% title page for many authors
%
\begin{titlepage}
\title{Implication of CMS results on phenomenological MSSM}

  \begin{Authlist}
    Maurizio Pierini
    \Instfoot{cern}{CERN}    
    Maria Spiropulu
    \Instfoot{caltech}{CERN \& California Institute Of Technology, Pasadena}    
    Filip Moortgat, Luc Pape
    \Instfoot{eth}{ETH, Zurich, Switzerland}
    Joseph Lykken
    \Instfoot{fnal}{Fermi National Accelerator Laboratory, Batavia, Illinois}
    Harrison Prosper, Sezen Sekmen
    \Instfoot{fsu}{Florida State University, Tallahassee}
    Sabine Kraml
    \Instfoot{lpsc}{LPSC, Grenoble}
    Sanjay Padhi
    \Instfoot{ucsd}{University of California, San Diego}
    
  \end{Authlist}

\begin{abstract}
We discuss the implication of CMS results on phenomenological MSSM (pMSSM). The pMSSM
consists of a huge number of free parameters, we address the parameters sensitive 
to the recent CMS studies in a statistical consistent manner. The sensitivity of one as a function 
of others are studied here. The results provide new constraints on parameters, sensitive to hadronic as well as 
multilepton final states using 35 pb$^{-1}$ of integrated luminosity.
\end{abstract}
\end{titlepage}

\section{Introduction}
\label{sec:intro}
The LHC has recently started delivering proton-proton collisions at a centre-of-mass energy of 7 TeV. 
Physics analyses at the LHC frequently depend on various inputs from theory that are only known with
limited accuracy. An important parameter to compare simulations and data is the predicted cross section for the former. The cross section calculations depend on various orders of perturbation theory
as well as the determination of PDFs. 

Most often there is no unique choice of which prescription should be used in 
a given analysis when comparing the simulations to the data. This study aims at establishing a convention as well 
as a certain set of choices based on inputs from the Monte Carlo simulations that are currently being used in the 
CMS collaboration. The higher order cross sections are then computed using these choices, and  
uncertanities due to the underlining assumptions are also computed.

In Section~\ref{sec:normalization}, guidelines for the calculation of K-factors based on higher order cross sections
along with given scales and PDF uncertainties are provided. In Section~\ref{sec:assumptions}
the assumptions made for these calculations are given, followed by the results in Section~\ref{sec:results}
and finally, in Section~\ref{sec:conclusion} we summarize the results.  

\section{Details of the statistical procedure}
\label{sec:stat}
We describe the (frequentist) TK
statistical procedures we have used in this study.  Suppose 
that our goal is to make a statement about the parameter $\theta_1$,
say the gluino mass
parameter $M_3$, independently of the remaining 18 pMSSM parameters. Since the 
expected signal $s$ is a function of  $d = 19$ parameters (see Sect.~(\ref{sec:model}),
which we denote by $\theta = \theta_1,\cdots,\theta_{19}$,  we need to eliminate $d - 1$ of them from the likelihood function so that the latter becomes a function of $\theta_1$ only. In general, it is extremely difficult to do this in 
a frequentist calculation in a way that preserves \emph{exact} coverage over the entire parameter
space. However, let $L_p(\theta_1) \equiv L(\theta_1, \hat{\theta}_2(\theta_1), \cdots)$ be the 
1-dimensional \emph{profile likelihood}, that is, the function obtained by maximizing the likelihood function $L(\theta_1, \theta_2, \cdots, \theta_{19})$ with respect to $\theta_2,\cdots,\theta_{19}$ for \emph{fixed} $\theta_1$ and replacing the exact, but unknown, values of $\theta_2,\cdots,\theta_{19}$
by their maximum likelihood estimates (MLE),  $\hat{\theta}_2(\theta_1),\cdots,\hat{\theta}_{19}(\theta_1)$. 

Replacing exact values by estimates is clearly an approximation. We should therefore not
expect any procedure that uses this approximation to yield confidence limits and intervals
with exact coverage. However, in practice, 1-dimensional profile likelihoods created from
multi-parameter likelihood functions often perform
surprisingly well~\cite{James}.
Let $L_{max}$ be the maximum of the likelihood function $L(\theta)$ and let 
$\Lambda = L_p /  L_{max}$ be the likelihood ratio. 
If the partial derivatives with respect to $\theta_i$ of the likelihood function, $L(\theta)$, exist up to second order and they form a $d \times d$ non-singular matrix (the Hessian), 
the following result holds,
\begin{equation}
    W = -2\log\Lambda \rightarrow \chi^2,
\end{equation}
as the amount of data grows without limit. This is  Wilks theorem~\cite{Wilks, James}. For an 
approximate 95\% C.L. lower limit on $\theta_1$, we set $W = 1.64$, that is, $\Lambda = 0.44$,
and solve for the lower limit.

\subsection{Non-parametric profiling algorithm}
The problem we need to solve is the following: for a fixed value of a parameter, say $Q$, we want to find 
the maximum value of the likelihood function when the latter is available only as a weighted swarm of points. The quantity $Q$ could be a pMSSM parameter, a predicted observable of a 
sparticle mass. Here, written as pseudo-code, is our algorithm for finding the profile likelihood:
\begin{verbatim}
1	pMSSMPOINTS, QBIN, Q = inputs()
2	NBOOTSTRAP = 100
3	profile = 0

4	repeat NBOOTSTRAP times:
	      
5	      POINTS = generateBootstrapSample(pMSSMPOINTS)
6	      histogram = histogramPoints(POINTS)
	
7	      DMAX = -1
8	      for point in POINTS:
	      
9	           if Q not in QBIN: continue
	      
10	           d = histogram.density(point)	      
11	           if d > DMAX: DMAX = d
	 
12	      profile = profile + DMAX
	        
13	 profile = profile / NBOOTSTRAP	 
14	 return profile	      
\end{verbatim}
\begin{itemize}
	\item[1] Get the pMSSM points, the bin $QBIN$ for which the profile likelihood
	is to be computed, and the value of $Q$.
	
	\item[6] Generate a $d$-dimensional histogram from current bootstrap sample.
	
	\item[9] Make sure $Q$ lies in desired bin $QBIN$.
	
	\item[10,11] Find largest density $DMAX$ so far.
	
	\item[12--14] Return average of estimates of profile likelihood.
\end{itemize}
The above algorithm is implemented in a class we developed called {\tt KDTProfileLikelihood}, which
makes use of the multi-dimensional histogrammer {\tt TKDTreeBinning} in Root.
The $d$-dimensional histogram is created through recursive binary partitioning of the parameter
space in such a way that bins have equal counts. The underlying data structure is 
a kd-tree~\cite{TKDTreeBinning}.


%\begin{thebibliography}
%	\bibitem{Wilks}
%	S.S~Wilks, ``The large-sample distribution of the likelihood ratio for testing composite hypotheses,'' Ann. Math. Statist. {\bf 9}, 60-62 (1938).
%	
%	\bibitem{James}
%	F.~James, ``Statistical Methods in Experimental Physics,''  2nd Edition, (World Scientific, Singapore, 2008).
%	
%\end{thebibliography}

\section{Current experimental and theoretical constraints}
\label{sec:limits}

The most relevant constraints today are the [SUSY and Higgs] mass 
limits from LEP2 and the Tevatron, electroweak precision observables, 
the branching ratios of the decays 
$B\to X_s\gamma$ and $B_s\to \mu^+\mu^-$, the anomalous magnetic moment of the 
muon $(g-2)_\mu$, and the relic density of dark matter $\Omega h^2$.  
They are compiled in Table~\ref{tab:observables} (to be extended).

Other important constraints, which we are not yet taking into account, 
come from $m_W$, $A^{\rm FB}_b$ and $\Delta M_{b}$.


\begin{table}[h]\begin{center}
\begin{tabular}{| l | c | c || c | l |}
  \hline
  Observable & exp.\ constraint & ref. & add. theory error & ref. \\ 
  \hline
  $M_Z$ [GeV] & $91.1875\pm 0.0021$ 
              & \cite{Nakamura:2010zzi} & -- & \\
  $M_t$ [GeV] & $173.3\pm 1.1$ 
              & \cite{top:1900yx} & -- & \\
  $\mbmb$ [GeV] &  $4.19^{+0.18}_{-0.06}$ 
                & \cite{Nakamura:2010zzi} & -- & \\
  $\asmz$ & $0.1184\pm 0.0007$ & \cite{Nakamura:2010zzi} & -- & \\
          & or $0.1176\pm 0.002$ & \cite{Nakamura:2010zzi} & -- & \\
  \hline
  $m_h$ [GeV] & $\ge 114.4$ & \cite{Schael:2006cr} 
              & $\pm 1.5$ & \cite{Degrassi:2002fi} \\
  $\Delta a_\mu \times 10^{-10}$ 
       & $e^+e^-:$ $29.6\pm 8.1$ 
       & \cite{Davier:2010nc} & $\pm 2$ & \\
       & $\tau's:$ $15.7\pm 8.2$ 
       & \cite{xxx} &  & \\
  ${\rm BR}(b\to s\gamma)\times 10^{-4}$  
       & $3.55 \pm 0.24_{\rm stat}\pm 0.09_{\rm sys}$  
       & \cite{HFAG:2010qj}  & ? &  \\
  ${\rm BR}(B_s\to \mu^+\mu^-)$  
       & $\le 3.6 \times 10^{-8}$ & \cite{HFAG:2010qj} & ? &  \\
  $\Omega_{\rm DM} h^2$ 
       & $0.1123\pm 0.0035$ 
       & \cite{Jarosik:2010iu} 
       & ? &   \\
  SUSY masses & LEP2 limits 
              & \cite{lepsusy} 
	      & ? & \\
  \hline
\end{tabular}
\end{center}
\caption{\label{tab:observables} Important observables, to be used in the likelihood calculation. }
\end{table}


\section{Results}
\label{sec:results}

In this section we present the results obtained by following the analysis path in Section~\ref{sec:analysis}, in terms of  distributions of profile likelihood to maximum likelihood ratios $L_p / L_{max}$ for each parameter of interest.  Further description on this ratio a as measure can be found in the Appendix.  

Distributions of the ratio $L_p/L_{max}$ are shown for the 19 input pMSSM parameters in Figures~\ref{fig:LRwcms_msq} to~\ref{fig:LRwcms_tbmu}.  Similarly Figures~\ref{fig:LRwcms_sq} to~\ref{fig:LRwcms_Higgs} show the ratio for the physical sparticle masses.  The relations between the scalar SUSY breaking mass parameters with the physical scalar masses as well as the gaugino mass parameters with the physical gaugino masses should be noted.  The colored and shaded histograms in each plot depict the ratio before and after the inclusion of CMS results respectively.  We observe that the CMS results indeed introduce a variation in the likelihood ratio distributions, where the variation is more enhanced for some parameters/masses and milder for others.  

We see that all squark/slepton mass parameters and all squark/slepton masses shift upwards systematically after adding the CMS results.  Gaugino mass parameter $M_3$ and the gluino mass also simultaneously move up due to the constrains dominantly from the di-jet $\alpha_T$ analysis.  An important aspect to note is that no significant change is observed in the distribution for the mass of $\tilde{\chi}^0_1$, which is the lightest supersymmetric particle, since neither a stringent $E^T_{miss}$ cut nor any other technique dedicated to constraining the $\tilde{\chi}^0_1$ mass were imposed in the analyses considered.  We owe our ability to observe this effect to the freedom offered by the pMSSM parameterization, in which neutralino/chargino masses are allowed to vary independently from the gluino mass.  Had the interpretation been done using CMSSM, we would be influenced by the strict gaugino mass relation described in Section~\ref{sec:motivation}, and $\tilde{\chi}^0_1$ mass would be forced to move upwards in correlation with the gluino mass.

Figure~\ref{fig:LRwcms_omg} shows $L_p/L_{max}$ for the dark matter relic density calculated using {\tt micrOMEGAs 2.4}~\cite{Belanger:2006is} assuming $\tilde{\chi}^0_1$ is the lightest supersymmetric particle (LSP) and the dark matter candidate.  It must be noted that Berger {\it et.al.} scans imposed the WMAP upper limit $\Omega_{\tilde{\chi}^0_1} \le 0.1210$ as a constraint on the points.  Information from CMS does modify the distribution, however the effect is not sufficient to impose a concrete constraint on $\Omega_{\tilde{\chi}^0_1}$.

Finally Figures~\ref{fig:LRwcms_EWobs_s1} and~\ref{fig:LRwcms_EWobs_s2} show distributions for low energy observables as predicted by pMSSM.  We again remind that constraints based on experimental measurements for a subset of these observables were taken into account while selecting the pMSSM points by Barger {\it et. al.}.  Similar to the case for relic density, 2010 CMS measurements do not yet allow us to constrain these observables further. 

The results we have presented here which were obtained using only 35 pb$^{-1}$ of CMS data show that even with this modest amount of data, we are able to start making inference on a sufficiently generic and well-motivated construction of supersymmetry.  Upcoming analyses done with at least an order of magnitude 
This is only the starting point that has opened up a vast amount of investigation.  We will develop this study further by considering the following 

\begin{itemize}
\item The sampling by Berger {\it et. al.} assumes a box-like likelihood with fixed boundaries on observables.  A formally defined likelihood can be used for 
\item The number of pMSSM points used in the analysis can be increased from 6K to a much larger number.  
\item A more thorough scan can be done.
\item Statistical method can be improved
\end{itemize}


\begin{figure}[htbp]
\begin{center}
\includegraphics[height=5.5cm]{figs/fig_m_Q_L.pdf} 
\includegraphics[height=5.5cm]{figs/fig_m_Q_3.pdf} \\
\includegraphics[height=5.5cm]{figs/fig_m_u_1.pdf}
\includegraphics[height=5.5cm]{figs/fig_m_u_3.pdf} \\
\includegraphics[height=5.5cm]{figs/fig_m_d_1.pdf}
\includegraphics[height=5.5cm]{figs/fig_m_d_3.pdf}
\caption{Ratios of profile likelihood $L_p$ to maximum likelihood $L_{max}$ shown for the squark mass parameters at SUSY scale.  The colored and shaded histograms show the distributions before and after the inclusion of the CMS results.}
\label{fig:LRwcms_msq}
\end{center}
\end{figure}


\begin{figure}[htbp]
\begin{center}
\includegraphics[height=5.5cm]{figs/fig_m_L_L.pdf} 
\includegraphics[height=5.5cm]{figs/fig_m_L_3.pdf} \\
\includegraphics[height=5.5cm]{figs/fig_m_e_1.pdf}
\includegraphics[height=5.5cm]{figs/fig_m_e_3.pdf}
\caption{Ratios of profile likelihood $L_p$ to maximum likelihood $L_{max}$ shown for the slepton mass parameters at SUSY scale.  The colored and shaded histograms show the distributions before and after the inclusion of the CMS results.}
\label{fig:LRwcms_msl}
\end{center}
\end{figure}


\begin{figure}[htbp]
\begin{center}
\includegraphics[height=5.5cm]{figs/fig_M_1.pdf} 
\includegraphics[height=5.5cm]{figs/fig_M_2.pdf} \\
\includegraphics[height=5.5cm]{figs/fig_M_3.pdf}
\caption{Ratios of profile likelihood $L_p$ to maximum likelihood $L_{max}$ shown for gaugino mass parameters at  SUSY scale.  The colored and shaded histograms show the distributions before and after the inclusion of the CMS results.}
\label{fig:LRwcms_M}
\end{center}
\end{figure}


\begin{figure}[htbp]
\begin{center}
\includegraphics[height=5.5cm]{figs/fig_A_t.pdf} 
\includegraphics[height=5.5cm]{figs/fig_A_b.pdf} \\
\includegraphics[height=5.5cm]{figs/fig_A_tau.pdf}
\caption{Ratios of profile likelihood $L_p$ to maximum likelihood $L_{max}$ shown for trilinear couplings at SUSY scale.  The colored and shaded histograms show the distributions before and after the inclusion of the CMS results.}
\label{fig:LRwcms_A}
\end{center}
\end{figure}

\begin{figure}[htbp]
\begin{center}
\includegraphics[height=5.5cm]{figs/fig_tanbeta.pdf} 
\includegraphics[height=5.5cm]{figs/fig_mu.pdf} 
\caption{Ratios of profile likelihood $L_p$ to maximum likelihood $L_{max}$ shown for $\tan\beta$ and $\mu$ parameter at SUSY scale.  The colored and shaded histograms show the distributions before and after the inclusion of the CMS results.}
\label{fig:LRwcms_tbmu}
\end{center}
\end{figure}



\begin{figure}[htbp]
\begin{center}
\includegraphics[height=5.5cm]{figs/fig_u_L.pdf} 
\includegraphics[height=5.5cm]{figs/fig_u_R.pdf} \\
\includegraphics[height=5.5cm]{figs/fig_d_L.pdf} 
\includegraphics[height=5.5cm]{figs/fig_d_R.pdf} \\
\includegraphics[height=5.5cm]{figs/fig_b_1.pdf} 
\includegraphics[height=5.5cm]{figs/fig_b_2.pdf} \\
\includegraphics[height=5.5cm]{figs/fig_t_1.pdf} 
\includegraphics[height=5.5cm]{figs/fig_t_2.pdf} \\
\caption{Ratios of profile likelihood $L_p$ to maximum likelihood $L_{max}$ shown for squark masses.  The colored and shaded histograms show the distributions before and after the inclusion of the CMS results.}
\label{fig:LRwcms_sq}
\end{center}
\end{figure}


\begin{figure}[htbp]
\begin{center}
\includegraphics[height=5.5cm]{figs/fig_e_L.pdf} 
\includegraphics[height=5.5cm]{figs/fig_e_R.pdf} \\
\includegraphics[height=5.5cm]{figs/fig_nu_e_L.pdf} 
\includegraphics[height=5.5cm]{figs/fig_nu_tau_L.pdf} \\
\includegraphics[height=5.5cm]{figs/fig_tau_1.pdf} 
\includegraphics[height=5.5cm]{figs/fig_tau_2.pdf}
\caption{Ratios of profile likelihood $L_p$ to maximum likelihood $L_{max}$ shown for predictions for slepton masses.  The colored and shaded histograms show the distributions before and after the inclusion of the CMS results.}
\label{fig:LRwcms:sl}
\end{center}
\end{figure}

\begin{figure}[htbp]
\begin{center}
\includegraphics[height=5.5cm]{figs/fig_g.pdf} 
\caption{Ratios of profile likelihood $L_p$ to maximum likelihood $L_{max}$ shown for the gluino mass.  The colored and shaded histograms show the distributions before and after the inclusion of the CMS results.}
\label{fig:LRwcms:g}
\end{center}
\end{figure}


\begin{figure}[htbp]
\begin{center}
\includegraphics[height=5.5cm]{figs/fig_chi_1_0.pdf} 
\includegraphics[height=5.5cm]{figs/fig_chi_2_0.pdf} \\
\includegraphics[height=5.5cm]{figs/fig_chi_3_0.pdf} 
\includegraphics[height=5.5cm]{figs/fig_chi_4_0.pdf} 
\caption{Ratios of profile likelihood $L_p$ to maximum likelihood $L_{max}$ shown for the neutralino masses.  The colored and shaded histograms show the distributions before and after the inclusion of the CMS results.}
\label{fig:LRwcms_chi0}
\end{center}
\end{figure}

\begin{figure}[htbp]
\begin{center}
\includegraphics[height=5.5cm]{figs/fig_chi_1_pm.pdf} 
\includegraphics[height=5.5cm]{figs/fig_chi_2_pm.pdf}
\caption{Ratios of profile likelihood $L_p$ to maximum likelihood $L_{max}$ shown for chargino masses.  The colored and shaded histograms show the distributions before and after the inclusion of the CMS results.}
\label{fig:LRwcms_chipm}
\end{center}
\end{figure}


\begin{figure}[htbp]
\begin{center}
\includegraphics[height=5.5cm]{figs/fig_h.pdf} 
\includegraphics[height=5.5cm]{figs/fig_H0.pdf} \\
\includegraphics[height=5.5cm]{figs/fig_A.pdf} 
\includegraphics[height=5.5cm]{figs/fig_H_pm.pdf} 
\caption{Ratios of profile likelihood $L_p$ to maximum likelihood $L_{max}$ shown for the Higgs masses.  The colored and shaded histograms show the distributions before and after the inclusion of the CMS results.}
\label{fig:LRwcms_Higgs}
\end{center}
\end{figure}


\begin{figure}[htbp]
\begin{center}
\includegraphics[height=5.5cm]{figs/fig_omega_m.pdf} 
\caption{Ratio of profile likelihood $L_p$ to maximum likelihood $L_{max}$ shown for lightest neutralino dark matter relic density.  The colored and shaded histograms show the distributions before and after the inclusion of the CMS results.}
\label{fig:LRwcms_omg}
\end{center}
\end{figure}


%\begin{figure}[htbp]
%\begin{center}
%\includegraphics[height=5.5cm]{figs/fig_drho_m.pdf} 
%\includegraphics[height=5.5cm]{figs/fig_gmu_m.pdf} \\
%\includegraphics[height=5.5cm]{figs/fig_bsgamma_m.pdf} 
%\includegraphics[height=5.5cm]{figs/fig_bsmumu_m.pdf} \\
%\includegraphics[height=5.5cm]{figs/fig_rbtaunu_m.pdf} 
%\caption{Ratios of profile likelihood $L_p$ to maximum likelihood $L_{max}$ shown for predictions for weak scale observables as calculated by micromegas.  The colored and shaded histograms show the distributions before and after the inclusion of the CMS results.}
%\label{fig:LRwcms_EWobs_m}
%\end{center}
%\end{figure}


\begin{figure}[htbp]
\begin{center}
%\includegraphics[height=5.5cm]{figs/fig_delta0_s.pdf} 
\includegraphics[height=5.5cm]{figs/fig_drho_m.pdf} 
\includegraphics[height=5.5cm]{figs/fig_muon_gm2_s.pdf} \\
\includegraphics[height=5.5cm]{figs/fig_bsgamma_s.pdf} 
\includegraphics[height=5.5cm]{figs/fig_Bsmumu_s.pdf} \\
\includegraphics[height=5.5cm]{figs/fig_Btaunu_s.pdf} 
\includegraphics[height=5.5cm]{figs/fig_RBtaunu_s.pdf} 
\caption{Ratios of profile likelihood $L_p$ to maximum likelihood $L_{max}$ shown for predictions for weak scale observables as calculated by superiso - I.  The colored and shaded histograms show the distributions before and after the inclusion of the CMS results.}
\label{fig:LRwcms_EWobs_s1}
\end{center}
\end{figure}


\begin{figure}[htbp]
\begin{center}
\includegraphics[height=5.5cm]{figs/fig_BDtaunu_s.pdf} 
\includegraphics[height=5.5cm]{figs/fig_BDtaunu_BDenu_s.pdf} \\
\includegraphics[height=5.5cm]{figs/fig_Dmunu_s.pdf} 
\includegraphics[height=5.5cm]{figs/fig_Dsmunu_s.pdf} \\
\includegraphics[height=5.5cm]{figs/fig_Dstaunu_s.pdf} 
\includegraphics[height=5.5cm]{figs/fig_Kmunu_pimunu_s.pdf} \\
\includegraphics[height=5.5cm]{figs/fig_Rl23_s.pdf} 
\caption{Ratios of profile likelihood $L_p$ to maximum likelihood $L_{max}$ shown for predictions for weak scale observables as calculated by superiso - II.  The colored and shaded histograms show the distributions before and after the inclusion of the CMS results.}
\label{fig:LRwcms_EWobs_s2}
\end{center}
\end{figure}






\section{Conclusion}
\label{sec:conclusion}
In conclusion, the first search using same-sign dileptons with $b$-jets and \met~~has 
been presented. In the
proton-proton collision data sample corresponding to an integrated luminosity of 
 \intLumi~at $\sqrt{s}$ = 7 TeV,
no significant deviations from the Standard Model expectations are observed. 
We use this data to set 95\% CL. on the
number of observed events for a number of plausible signal regions
defined in terms of requirements in \met and $H_T$, the number of
$b$-tagged jets (2 or 3), and also the sign of the leptons (only positive dileptons
or both positive and negative dileptons).
We also provide enough information so that interested phenomenologists
could interpret our limits in their favorite new physics models.

In addition, we set limits on the parameter space of six new physics models:
\begin{enumerate}
\item A model with a $Z'$ vector boson with flavor violating couplings to $u-$ and $t$-quarks.

\item A model with a neutral scalar with flavor violating couplings to $u-$ and $t$-quarks.

\item A SUSY model of stop production from two body gluino decays: 
$pp \to \widetilde{g} \widetilde{g}$ followed by
$\widetilde{g} \to t\widetilde{t}$ and $\widetilde{t} \to t \chi_1^0$.

\item A SUSY model of stop pair production from three body gluino decays:
$pp \to \widetilde{g} \widetilde{g}$ followed by
$\widetilde{g} \to t\widetilde{t}\chi_1^0$.

\item A SUSY model of sbottom pair production: $pp \to \tilde{b}\tilde{b}$ followed
by $\tilde{b} \to t\chi^{-}$ and $\chi^{-} \to W^- \chi_1^0$.

\item A SUSY model of sbottom production from gluino decays:
$pp \to \widetilde{g} \widetilde{g}$ followed by
$\widetilde{g} \to \widetilde{b}b$,
$\widetilde{b} \to t\chi^-$, and $\chi^{-} \to W^- \chi_1^0$.
\end{enumerate}

And that's all for now.


\clearpage
\begin{thebibliography}{10}

\bibitem{Choi:2007ka}
K.~Choi and H.~P. Nilles,
\newblock JHEP {\bf 04}, 006 (2007), hep-ph/0702146.

\bibitem{Martin:2009ad}
S.~P. Martin,
\newblock Phys. Rev. {\bf D79}, 095019 (2009), 0903.3568.

\bibitem{Horton:2009ed}
D.~Horton and G.~G. Ross,
\newblock Nucl. Phys. {\bf B830}, 221 (2010), 0908.0857.

\bibitem{Allanach:2006fy}
B.~C. Allanach {\em et~al.},
\newblock (2006), hep-ph/0602198.

\bibitem{Kolda:1995iw}
C.~F. Kolda and S.~P. Martin,
\newblock Phys. Rev. {\bf D53}, 3871 (1996), hep-ph/9503445.

\bibitem{Polonsky:1994sr}
N.~Polonsky and A.~Pomarol,
\newblock Phys. Rev. Lett. {\bf 73}, 2292 (1994), hep-ph/9406224.

\bibitem{Polonsky:1994rz}
N.~Polonsky and A.~Pomarol,
\newblock Phys. Rev. {\bf D51}, 6532 (1995), hep-ph/9410231.

\bibitem{Ellis:2002wv}
J.~R. Ellis, K.~A. Olive, and Y.~Santoso,
\newblock Phys. Lett. {\bf B539}, 107 (2002), hep-ph/0204192.

\bibitem{Berger:2008cq}
C.~F. Berger, J.~S. Gainer, J.~L. Hewett, and T.~G. Rizzo,
\newblock JHEP {\bf 02}, 023 (2009), 0812.0980.

\bibitem{Conley:2010du}
J.~A. Conley, J.~S. Gainer, J.~L. Hewett, M.~P. Le, and T.~G. Rizzo,
\newblock (2010), 1009.2539.

\bibitem{Lyons:2003bw}
L.~Lyons, (ed.~), R.~P. Mount, (ed.~), and R.~Reitmeyer, (ed.~),
\newblock Prepared for PHYSTAT2003: Statistical Problems in Particle Physics,
  Astrophysics, and Cosmology, Menlo Park, California, 8-11 Sep 2003.

\bibitem{Nakamura:2010zzi}
Particle Data Group, K.~Nakamura,
\newblock J. Phys. {\bf G37}, 075021 (2010).

\bibitem{top:1900yx}
CDF and D0, and others,
\newblock (2010), 1007.3178.

\bibitem{Schael:2006cr}
{ALEPH, DELPHI, L3 and OPAL collaborations and the LEP Working Group for Higgs
  Boson Searches}, S.~Schael {\em et~al.},
\newblock Eur. Phys. J. {\bf C47}, 547 (2006), hep-ex/0602042.

\bibitem{Degrassi:2002fi}
G.~Degrassi, S.~Heinemeyer, W.~Hollik, P.~Slavich, and G.~Weiglein,
\newblock Eur. Phys. J. {\bf C28}, 133 (2003), hep-ph/0212020.

\bibitem{Davier:2010nc}
M.~Davier, A.~Hoecker, B.~Malaescu, and Z.~Zhang,
\newblock (2010), 1010.4180.

\bibitem{HFAG:2010qj}
The Heavy Flavor Averaging Group, D.~Asner {\em et~al.},
\newblock (2010), 1010.1589.

\bibitem{Jarosik:2010iu}
N.~Jarosik {\em et~al.},
\newblock (2010), 1001.4744.

\bibitem{lepsusy}
ALEPH, DELPHI, L3 and OPAL, {LEP2 SUSY Working Group,},
\newblock {\tt http://lepsusy.web.cern.ch/lepsusy/}.

\bibitem{Skands:2003cj}
P.~Z. Skands {\em et~al.},
\newblock JHEP {\bf 07}, 036 (2004), hep-ph/0311123.

\bibitem{Sjostrand:2006za}
T.~Sjostrand, S.~Mrenna, and P.~Z. Skands,
\newblock JHEP {\bf 05}, 026 (2006), hep-ph/0603175.

\bibitem{Ovyn:2009tx}
S.~Ovyn, X.~Rouby, and V.~Lemaitre,
\newblock (2009), 0903.2225.

%\cite{Skands:2003cj}
\bibitem{Skands:2003cj}
  P.~Z.~Skands {\it et al.},
  %``SUSY Les Houches Accord: Interfacing SUSY Spectrum Calculators, Decay
  %Packages, and Event Generators,''
  JHEP {\bf 0407} (2004) 036
  [arXiv:hep-ph/0311123].
  %%CITATION = JHEPA,0407,036;%%

%\cite{Sjostrand:2006za}
\bibitem{Sjostrand:2006za}
  T.~Sjostrand, S.~Mrenna and P.~Z.~Skands,
  %``PYTHIA 6.4 Physics and Manual,''
  JHEP {\bf 0605} (2006) 026
  [arXiv:hep-ph/0603175].
  %%CITATION = JHEPA,0605,026;%%

%\cite{Ovyn:2009tx}
\bibitem{Ovyn:2009tx}
  S.~Ovyn, X.~Rouby and V.~Lemaitre,
  %``Delphes, a framework for fast simulation of a generic collider
  %experiment,''
  arXiv:0903.2225 [hep-ph].
  %%CITATION = ARXIV:0903.2225;%%
%
\bibitem{Belanger:2006is}
  G.~Belanger, F.~Boudjema, A.~Pukhov and A.~Semenov,
  %``micrOMEGAs2.0: A program to calculate the relic density of dark matter  in
  %a generic model,''
  Comput.\ Phys.\ Commun.\  {\bf 176}, 367 (2007)
  [arXiv:hep-ph/0607059].
  %%CITATION = CPHCB,176,367;%%
%
\bibitem{Mahmoudi:2008tp}
  F.~Mahmoudi,
  %``SuperIso v2.3: A Program for calculating flavor physics observables in
  %Supersymmetry,''
  Comput.\ Phys.\ Commun.\  {\bf 180}, 1579 (2009)
  [arXiv:0808.3144 [hep-ph]].
  %%CITATION = CPHCB,180,1579;%%
%

\bibitem{Wilks}
S.S~Wilks, ``The large-sample distribution of the likelihood ratio for testing composite hypotheses,'' Ann. Math. Statist. {\bf 9}, 60-62 (1938).
	
\bibitem{James}
F.~James, ``Statistical Methods in Experimental Physics,''  2nd Edition, (World Scientific, Singapore, 2008).

\bibitem{sspaper}
CMS Collaboration, ``Search for new physics with same-sign isolated dilepton events with jets and missing transverse energy at the LHC", e-Print: arXiv:1104.3168 [hep-ex] (2011).

\bibitem{alphatpaper}
CMS Collaboration, ``Search for Supersymmetry in pp Collisions at 7 TeV in Events with Jets and Missing Transverse Energy", Phys.Lett. B698 196-218 (2011).

\bibitem{ospaper}
CMS Collaboration, ``Search for Physics Beyond the Standard Model in Opposite-Sign Dilepton Events at  $\sqrt{s} = 7$ TeV", e-Print: arXiv:1103.1348 [hep-ex] (2011).


\bibitem{TKDTreeBinning}
{\tt Root 5.28}.

\end{thebibliography}

\end{document}
