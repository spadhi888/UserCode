\section{Motivation for a generic MSSM setup}
\label{sec:motivation}

Perhaps the least understood aspect of SUSY is its breaking, 
which in turn determines the boundary conditions for the Lagrange 
parameters at some high scale. Theorists have come up with a 
long list of possible candidate scenarios, including 
supergravity (SUGRA), 
gauge mediation (GMSB), 
anomaly mediation (AMSB), 
gaugino mediation, radion mediation, etc.. 
They come in minimal (mSUGRA, mGMSB, ....) and  
less minimal ({\it e.g.}, non-universal Higgs masses, NUHM,  
or non-universal gaugino masses) variants, as well as in general setups 
(general gauge mediation, ...). 
Moreover, there are models of compressed, effective or split SUSY, 
GUT-inspired models, string-inspired models, and so on. 
Each of these possibilities features characteristic relations between 
fundamental parameters and hence characteristic mass spectra, decay 
patterns and properties of the dark matter candidate. 

The CMSSM covers just a subset of this spectrum. To give some 
examples:

\begin{itemize}

\item The CMSSM assumes universal gaugino masses $M_1=M_2=M_3\equiv m_{1/2}$ 
at the GUT scale, leading to 
\begin{equation}
  M_1 : M_2 : M_3 \approx 1 : 2 : 7 ~{\rm with}~ M_1 \approx 0.4\,m_{1/2}
  \label{eq:gauginos}
\end{equation}
at the EW scale, which is equivalent to 
$m_{\tilde\chi^0_2}\approx 2m_{\tilde\chi^0_1}\approx 0.8\,m_{1/2}$ 
and $m_{\tilde g}\approx 7m_{\tilde\chi^0_1}\approx 2.8\,m_{1/2}$. 
Other models can have very different relations between 
$M_1$, $M_2$, $M_3$, giving rise to the so-called 
``gaugino code''~\cite{Choi:2007ka}, 
which can be very useful for model discrimination. Besides, models with 
non-universal gaugino masses are quite natural~\cite{Martin:2009ad} even 
within the SUGRA context, and they can have very low finetuning~\cite{Horton:2009ed}.

\item Over most of the CMSSM parameter space $|\mu|^2\gtrsim m_{1/2}^2$. 
The lightest neutralino is then mostly bino, the second-lightest mostly wino, 
and the heavier ones mostly higgsinos. Light higgsinos and large gaugino--higgsino 
mixing (mixed bino--higgsino dark matter) occur only in the focus point region, 
{\it i.e.}\ when squarks and sleptons are very heavy. This has a strong impact on 
squark and gluino cascade decays, as well as on the part of parameter 
space that is compatible with dark matter constraints. 

\item Turning to the sfermion sector, the slepton-mass parameters are to good approximation
\begin{equation}
  m_{\tilde E}^2 \approx m_0^2 + 0.15\,m_{1/2}^2\,, \qquad 
  m_{\tilde L}^2 \approx m_0^2 + 0.5\,m_{1/2}^2\,. 
  \label{eq:sleptons}
\end{equation}
Note that this implies that right-chiral states are alway lighter 
than the left-chiral ones. Combining Eqs.~(\ref{eq:gauginos}) and 
(\ref{eq:sleptons}) we see that for small $m_0$ (but large enough 
to have a neutralino LSP) this leads to the typical mass pattern 
$m_{\tilde\chi^0_1}<m_{\tilde e_R}<m_{\tilde\chi^0_2}<m_{\tilde e_L}$.  
For the first two generations of squarks we have 
\begin{equation}
  m_{\tilde U, \tilde D}^2 \approx m_0^2 + K m_{1/2}^2\,, \qquad 
  m_{\tilde Q}^2 \approx m_0^2 + (K+0.5)m_{1/2}^2\,, 
\end{equation}
with $K\sim4.5$ to $6.5$, and the dependence on $m_{1/2}$ dominated by the 
gluino contribution, {\it i.e.}\ by $M_3$. It is clear that any limit on 
or determination of $m_0$ is completely dominated by the slepton 
sector~\cite{Allanach:2006fy}. 
Non-universal scalar masses are heavily constrained by flavour-changing 
neutral currents (FCNC), at least for the first and second generations. 
For the third generation, the FCNC constraints are much less severe. 
One possibility to motivate universal mass parameters for sfermions is 
to embed them in a higher gauge group, like SO(10). But even then, 
non-universalities can occur through D-term contributions \cite{Kolda:1995iw} 
and/or GUT-scale threshold corrections \cite{Polonsky:1994sr,Polonsky:1994rz}. 
Besides, there is no sound theoretical motivation for unifying the 
mass-squared terms of the Higgs fields, $m_{H_1}^2$ and $m_{H_2}^2$,
with those of the other scalars. 
If this is given up $m_{H_{1,2}}^2$, or equivalently $\mu$ and $m_A$,
become free parameters of the model~\cite{Ellis:2002wv} 
(cf.\ the discussion of the value of $\mu$ above).

\item The assumption of scalar mass universality has another important 
implication, namely that the renormalization-group invariant quantity  
\begin{equation}
  S = \left(m_{H_2}^2-m_{H_1}^2\right) + {\rm Tr}\left(
      m_{\tilde Q}^2-2m_{\tilde U}^2+m_{\tilde D}^2
      +m_{\tilde E}^2-m_{\tilde L}^2\right)
\end{equation}
vanishes. This so-called $S$-parameter, if non-zero, influences 
the running of the scalar mass parametres $m_\phi^2$ proportional to 
their hypercharge $Y_\phi$
\begin{equation}
   16\pi^2\frac{d}{dt}m_\phi^2= .... + \frac{6}{5}Y_\phi g_1^2S\,.
\end{equation}
This can change the mass ordering of left- and right-chiral states 
or have an important influence on the Higgs sector. 
In the CMSSM however $S\equiv 0$.

\end{itemize}

From these considerations, which are just exemplary and by no means complete, 
it is clear that it is interesting and necessary to go beyond the na\"ive 
CMSSM. We need to search for SUSY without prejudice \cite{Berger:2008cq,Conley:2010du},  
even more so as we have all the necessary knowledge and machinery at 
our disposal. In this context note that major efforts have recently been 
devoted to developing precise statistical tools for analyzing new physics 
at the LHC \cite{Lyons:2003bw}. 
This includes sophisticated methods and tools for the investigation of 
multi-dimensional parameter spaces, as typical for SUSY models. 
Below we therefore lay out a program for the investigation of the 
phenomenological MSSM, which gets by with only a minimal set of assumptions. 
Most importantly it allows us to cover the full range of possible mass patterns 
and of neutralino dark matter properties.
