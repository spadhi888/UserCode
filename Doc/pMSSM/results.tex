\section{Results}
\label{sec:results}

In this section we present the results obtained by following the analysis path in Section~\ref{sec:analysis}, in terms of  distributions of profile likelihood to maximum likelihood ratios $L_p / L_{max}$ for each parameter of interest.  Further description on this ratio a as measure can be found in the Appendix.  

Distributions of the ratio $L_p/L_{max}$ are shown for the 19 input pMSSM parameters in Figures~\ref{fig:LRwcms_msq} to~\ref{fig:LRwcms_tbmu}.  Similarly Figures~\ref{fig:LRwcms_sq} to~\ref{fig:LRwcms_Higgs} show the ratio for the physical sparticle masses.  The relations between the scalar SUSY breaking mass parameters with the physical scalar masses as well as the gaugino mass parameters with the physical gaugino masses should be noted.  The colored and shaded histograms in each plot depict the ratio before and after the inclusion of CMS results respectively.  We observe that the CMS results indeed introduce a variation in the likelihood ratio distributions, where the variation is more enhanced for some parameters/masses and milder for others.  

We see that all squark/slepton mass parameters and all squark/slepton masses shift upwards systematically after adding the CMS results.  Gaugino mass parameter $M_3$ and the gluino mass also simultaneously move up due to the constrains dominantly from the di-jet $\alpha_T$ analysis.  An important aspect to note is that no significant change is observed in the distribution for the mass of $\tilde{\chi}^0_1$, which is the lightest supersymmetric particle, since neither a stringent $E^T_{miss}$ cut nor any other technique dedicated to constraining the $\tilde{\chi}^0_1$ mass were imposed in the analyses considered.  We owe our ability to observe this effect to the freedom offered by the pMSSM parameterization, in which neutralino/chargino masses are allowed to vary independently from the gluino mass.  Had the interpretation been done using CMSSM, we would be influenced by the strict gaugino mass relation described in Section~\ref{sec:motivation}, and $\tilde{\chi}^0_1$ mass would be forced to move upwards in correlation with the gluino mass.

Figure~\ref{fig:LRwcms_omg} shows $L_p/L_{max}$ for the dark matter relic density calculated using {\tt micrOMEGAs 2.4}~\cite{Belanger:2006is} assuming $\tilde{\chi}^0_1$ is the lightest supersymmetric particle (LSP) and the dark matter candidate.  It must be noted that Berger {\it et.al.} scans imposed the WMAP upper limit $\Omega_{\tilde{\chi}^0_1} \le 0.1210$ as a constraint on the points.  Information from CMS does modify the distribution, however the effect is not sufficient to impose a concrete constraint on $\Omega_{\tilde{\chi}^0_1}$.

Finally Figures~\ref{fig:LRwcms_EWobs_s1} and~\ref{fig:LRwcms_EWobs_s2} show distributions for low energy observables as predicted by pMSSM.  We again remind that constraints based on experimental measurements for a subset of these observables were taken into account while selecting the pMSSM points by Barger {\it et. al.}.  Similar to the case for relic density, 2010 CMS measurements do not yet allow us to constrain these observables further. 

The results we have presented here which were obtained using only 35 pb$^{-1}$ of CMS data show that even with this modest amount of data, we are able to start making inference on a sufficiently generic and well-motivated construction of supersymmetry.  Upcoming analyses done with at least an order of magnitude 
This is only the starting point that has opened up a vast amount of investigation.  We will develop this study further by considering the following 

\begin{itemize}
\item The sampling by Berger {\it et. al.} assumes a box-like likelihood with fixed boundaries on observables.  A formally defined likelihood can be used for 
\item The number of pMSSM points used in the analysis can be increased from 6K to a much larger number.  
\item A more thorough scan can be done.
\item Statistical method can be improved
\end{itemize}


\begin{figure}[htbp]
\begin{center}
\includegraphics[height=5.5cm]{figs/fig_m_Q_L.pdf} 
\includegraphics[height=5.5cm]{figs/fig_m_Q_3.pdf} \\
\includegraphics[height=5.5cm]{figs/fig_m_u_1.pdf}
\includegraphics[height=5.5cm]{figs/fig_m_u_3.pdf} \\
\includegraphics[height=5.5cm]{figs/fig_m_d_1.pdf}
\includegraphics[height=5.5cm]{figs/fig_m_d_3.pdf}
\caption{Ratios of profile likelihood $L_p$ to maximum likelihood $L_{max}$ shown for the squark mass parameters at SUSY scale.  The colored and shaded histograms show the distributions before and after the inclusion of the CMS results.}
\label{fig:LRwcms_msq}
\end{center}
\end{figure}


\begin{figure}[htbp]
\begin{center}
\includegraphics[height=5.5cm]{figs/fig_m_L_L.pdf} 
\includegraphics[height=5.5cm]{figs/fig_m_L_3.pdf} \\
\includegraphics[height=5.5cm]{figs/fig_m_e_1.pdf}
\includegraphics[height=5.5cm]{figs/fig_m_e_3.pdf}
\caption{Ratios of profile likelihood $L_p$ to maximum likelihood $L_{max}$ shown for the slepton mass parameters at SUSY scale.  The colored and shaded histograms show the distributions before and after the inclusion of the CMS results.}
\label{fig:LRwcms_msl}
\end{center}
\end{figure}


\begin{figure}[htbp]
\begin{center}
\includegraphics[height=5.5cm]{figs/fig_M_1.pdf} 
\includegraphics[height=5.5cm]{figs/fig_M_2.pdf} \\
\includegraphics[height=5.5cm]{figs/fig_M_3.pdf}
\caption{Ratios of profile likelihood $L_p$ to maximum likelihood $L_{max}$ shown for gaugino mass parameters at  SUSY scale.  The colored and shaded histograms show the distributions before and after the inclusion of the CMS results.}
\label{fig:LRwcms_M}
\end{center}
\end{figure}


\begin{figure}[htbp]
\begin{center}
\includegraphics[height=5.5cm]{figs/fig_A_t.pdf} 
\includegraphics[height=5.5cm]{figs/fig_A_b.pdf} \\
\includegraphics[height=5.5cm]{figs/fig_A_tau.pdf}
\caption{Ratios of profile likelihood $L_p$ to maximum likelihood $L_{max}$ shown for trilinear couplings at SUSY scale.  The colored and shaded histograms show the distributions before and after the inclusion of the CMS results.}
\label{fig:LRwcms_A}
\end{center}
\end{figure}

\begin{figure}[htbp]
\begin{center}
\includegraphics[height=5.5cm]{figs/fig_tanbeta.pdf} 
\includegraphics[height=5.5cm]{figs/fig_mu.pdf} 
\caption{Ratios of profile likelihood $L_p$ to maximum likelihood $L_{max}$ shown for $\tan\beta$ and $\mu$ parameter at SUSY scale.  The colored and shaded histograms show the distributions before and after the inclusion of the CMS results.}
\label{fig:LRwcms_tbmu}
\end{center}
\end{figure}



\begin{figure}[htbp]
\begin{center}
\includegraphics[height=5.5cm]{figs/fig_u_L.pdf} 
\includegraphics[height=5.5cm]{figs/fig_u_R.pdf} \\
\includegraphics[height=5.5cm]{figs/fig_d_L.pdf} 
\includegraphics[height=5.5cm]{figs/fig_d_R.pdf} \\
\includegraphics[height=5.5cm]{figs/fig_b_1.pdf} 
\includegraphics[height=5.5cm]{figs/fig_b_2.pdf} \\
\includegraphics[height=5.5cm]{figs/fig_t_1.pdf} 
\includegraphics[height=5.5cm]{figs/fig_t_2.pdf} \\
\caption{Ratios of profile likelihood $L_p$ to maximum likelihood $L_{max}$ shown for squark masses.  The colored and shaded histograms show the distributions before and after the inclusion of the CMS results.}
\label{fig:LRwcms_sq}
\end{center}
\end{figure}


\begin{figure}[htbp]
\begin{center}
\includegraphics[height=5.5cm]{figs/fig_e_L.pdf} 
\includegraphics[height=5.5cm]{figs/fig_e_R.pdf} \\
\includegraphics[height=5.5cm]{figs/fig_nu_e_L.pdf} 
\includegraphics[height=5.5cm]{figs/fig_nu_tau_L.pdf} \\
\includegraphics[height=5.5cm]{figs/fig_tau_1.pdf} 
\includegraphics[height=5.5cm]{figs/fig_tau_2.pdf}
\caption{Ratios of profile likelihood $L_p$ to maximum likelihood $L_{max}$ shown for predictions for slepton masses.  The colored and shaded histograms show the distributions before and after the inclusion of the CMS results.}
\label{fig:LRwcms:sl}
\end{center}
\end{figure}

\begin{figure}[htbp]
\begin{center}
\includegraphics[height=5.5cm]{figs/fig_g.pdf} 
\caption{Ratios of profile likelihood $L_p$ to maximum likelihood $L_{max}$ shown for the gluino mass.  The colored and shaded histograms show the distributions before and after the inclusion of the CMS results.}
\label{fig:LRwcms:g}
\end{center}
\end{figure}


\begin{figure}[htbp]
\begin{center}
\includegraphics[height=5.5cm]{figs/fig_chi_1_0.pdf} 
\includegraphics[height=5.5cm]{figs/fig_chi_2_0.pdf} \\
\includegraphics[height=5.5cm]{figs/fig_chi_3_0.pdf} 
\includegraphics[height=5.5cm]{figs/fig_chi_4_0.pdf} 
\caption{Ratios of profile likelihood $L_p$ to maximum likelihood $L_{max}$ shown for the neutralino masses.  The colored and shaded histograms show the distributions before and after the inclusion of the CMS results.}
\label{fig:LRwcms_chi0}
\end{center}
\end{figure}

\begin{figure}[htbp]
\begin{center}
\includegraphics[height=5.5cm]{figs/fig_chi_1_pm.pdf} 
\includegraphics[height=5.5cm]{figs/fig_chi_2_pm.pdf}
\caption{Ratios of profile likelihood $L_p$ to maximum likelihood $L_{max}$ shown for chargino masses.  The colored and shaded histograms show the distributions before and after the inclusion of the CMS results.}
\label{fig:LRwcms_chipm}
\end{center}
\end{figure}


\begin{figure}[htbp]
\begin{center}
\includegraphics[height=5.5cm]{figs/fig_h.pdf} 
\includegraphics[height=5.5cm]{figs/fig_H0.pdf} \\
\includegraphics[height=5.5cm]{figs/fig_A.pdf} 
\includegraphics[height=5.5cm]{figs/fig_H_pm.pdf} 
\caption{Ratios of profile likelihood $L_p$ to maximum likelihood $L_{max}$ shown for the Higgs masses.  The colored and shaded histograms show the distributions before and after the inclusion of the CMS results.}
\label{fig:LRwcms_Higgs}
\end{center}
\end{figure}


\begin{figure}[htbp]
\begin{center}
\includegraphics[height=5.5cm]{figs/fig_omega_m.pdf} 
\caption{Ratio of profile likelihood $L_p$ to maximum likelihood $L_{max}$ shown for lightest neutralino dark matter relic density.  The colored and shaded histograms show the distributions before and after the inclusion of the CMS results.}
\label{fig:LRwcms_omg}
\end{center}
\end{figure}


%\begin{figure}[htbp]
%\begin{center}
%\includegraphics[height=5.5cm]{figs/fig_drho_m.pdf} 
%\includegraphics[height=5.5cm]{figs/fig_gmu_m.pdf} \\
%\includegraphics[height=5.5cm]{figs/fig_bsgamma_m.pdf} 
%\includegraphics[height=5.5cm]{figs/fig_bsmumu_m.pdf} \\
%\includegraphics[height=5.5cm]{figs/fig_rbtaunu_m.pdf} 
%\caption{Ratios of profile likelihood $L_p$ to maximum likelihood $L_{max}$ shown for predictions for weak scale observables as calculated by micromegas.  The colored and shaded histograms show the distributions before and after the inclusion of the CMS results.}
%\label{fig:LRwcms_EWobs_m}
%\end{center}
%\end{figure}


\begin{figure}[htbp]
\begin{center}
%\includegraphics[height=5.5cm]{figs/fig_delta0_s.pdf} 
\includegraphics[height=5.5cm]{figs/fig_drho_m.pdf} 
\includegraphics[height=5.5cm]{figs/fig_muon_gm2_s.pdf} \\
\includegraphics[height=5.5cm]{figs/fig_bsgamma_s.pdf} 
\includegraphics[height=5.5cm]{figs/fig_Bsmumu_s.pdf} \\
\includegraphics[height=5.5cm]{figs/fig_Btaunu_s.pdf} 
\includegraphics[height=5.5cm]{figs/fig_RBtaunu_s.pdf} 
\caption{Ratios of profile likelihood $L_p$ to maximum likelihood $L_{max}$ shown for predictions for weak scale observables as calculated by superiso - I.  The colored and shaded histograms show the distributions before and after the inclusion of the CMS results.}
\label{fig:LRwcms_EWobs_s1}
\end{center}
\end{figure}


\begin{figure}[htbp]
\begin{center}
\includegraphics[height=5.5cm]{figs/fig_BDtaunu_s.pdf} 
\includegraphics[height=5.5cm]{figs/fig_BDtaunu_BDenu_s.pdf} \\
\includegraphics[height=5.5cm]{figs/fig_Dmunu_s.pdf} 
\includegraphics[height=5.5cm]{figs/fig_Dsmunu_s.pdf} \\
\includegraphics[height=5.5cm]{figs/fig_Dstaunu_s.pdf} 
\includegraphics[height=5.5cm]{figs/fig_Kmunu_pimunu_s.pdf} \\
\includegraphics[height=5.5cm]{figs/fig_Rl23_s.pdf} 
\caption{Ratios of profile likelihood $L_p$ to maximum likelihood $L_{max}$ shown for predictions for weak scale observables as calculated by superiso - II.  The colored and shaded histograms show the distributions before and after the inclusion of the CMS results.}
\label{fig:LRwcms_EWobs_s2}
\end{center}
\end{figure}





