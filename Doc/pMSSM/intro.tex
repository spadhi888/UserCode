\section{Introduction}
\label{sec:intro}

After the successful operation of the Large Hadron Collider (LHC) and the CMS detector
in 2010, and with good prospects for the future, the LHC is now ready to shed light on a number 
of open questions in Particle Physics, 
such as the mechanism of electroweak (EW) symmetry breaking, or the nature 
of new physics Beyond the Standard Model (BSM) needed to stabilize the EW scale. 

A wealth of theories that extend the Standard Model has been put forth in the past decades. 
The arguably best motivated such BSM theory --- and certainly the most 
thouroughly studied one --- is supersymmetry (SUSY). 
Indeed, searches for SUSY are among the primary objectives of the 
CMS experiment. Note that SUSY is exceedingly popular not 
only for its theoretical beauty but also because SUSY phenomenology 
is extremely rich, 
%in fact is can mimic almost any other new physics scenario. 
leading to a large variety of possible new signals at the LHC. 
In spite of this, the majority of SUSY studies focusses on a very special 
setup: the so-called Constrained Minimal Supersymmetric Standard Model (CMSSM). 
This was justified in the preparation for discoveries as the CMSSM, 
having just a handful of new parameters, is very predicive. However, 
the simple assumption of universality at the GUT scale lacks a sound 
theoretical motivation; the CMSSM should hence be regarded as a showcase 
model. When it comes to interpreting experimental results, on the 
one hand it is reasonable and interesting to do this within the CMSSM, 
as it provides (to a certain extent) an easy way to show performances, 
compare limits or reaches, etc. On the other hand, the interpretation in the 
$(m_0,m_{1/2})$ plane risks to heavyly overconstrain SUSY, as many 
possible mass patterns and signatures are not covered in the CMSSM. 
The same problem actually arises in any analysis that assumes a particular 
SUSY breaking scheme. 

In this Note we therefore introduce a different approach, which uses only 
minimal assumptions on the underlying SUSY parameters. In particular we do 
not rely on a particular SUSY breaking scheme. Instead, we use a 19-dimensional 
parametrization of the general MSSM, with parameters defined at the SUSY scale 
(by convention the geometric mean of the two stop masses), 
the so-called \emph{phenomenological MSSM} (pMSSM). 
We here demonstrate the feasibility of our approach by applying it to 
the 2010 CMS data-set corresponding to 35~pb$^{-1}$ of integrated luminosity.  
Using profile likelihoods, we combine 
the dijet $\alpha_T$ analysis, the opposite-sign dilepton 
analysis and the same-sign dilepton analysis and derive constraints 
on the SUSY particles that imply as few simplifying assumptions as possible.
Results from other SUSY analyses in CMS will be added as soon as they become available.

In this Note, we first give the motivation to go beyond CMSSM and work in 
a generic MSSM setup. After this, the pMSSM and its parametrization is defined. 
We then outline our analysis, giving details on the pMSSM points we have used, 
the detector simulation and the CMS analyses, and describe the statistical method based on 
profile likelihoods used for coping with the 19-dimensional setup 
of our model. Finally, we discuss our results and summarize our conclusions.

