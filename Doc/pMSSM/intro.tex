\section{Introduction}
\label{sec:intro}

With the extremely successful performence of both machine and detectors
at $\sqrt{s}=7$~TeV and good prospects to go to higher energies soon, 
the LHC is finally opening the window to the Terascale. 
Revolutionary new insights are expected from the LHC data, most importantly 
on the mechanism of electroweak (EW) symmetry breaking and on the nature 
of new physics Beyond the Standard Model (BSM) stabilizing the EW scale. 

A wealth of BSM theories has been put forth by the theoretical community 
and it is now up to experiments to test which, if any, of these theories 
are correct. 
The arguably best motivated such BSM theory---but certainly the most 
thouroughly studied one---is supersymmetry, or SUSY for short. 
Indeed, searches for SUSY are among the primary objectives of the 
CMS collaboration. Here note that SUSY is exceedingly popular not 
only for its theoretical beauties but also because SUSY phenomenology 
is extremely rich, 
%in fact is can mimic almost any other new physics scenario. 
leading to a large variety of possible new signals at the LHC. 

Nevertheless, the majority of SUSY studies focusses on a very special 
setup: the so-called Constrained Minimal Supersymmetric Standard Model (CMSSM). 
This was justified in the preparation for discoveries as the CMSSM, 
having just a handful of new parameters, is very predicive. However, 
the simple assumption of universality at the GUT scale lacks a sound 
theoretical motivation; the CMSSM should hence be regarded as a showcase 
model. When it comes to interpreting experimental results, on the 
one hand it is reasonable and interesting to do this within the CMSSM, 
as it provides (to a certain extent) an easy way to show performances, 
compare limits or reaches, etc. On the other hand, the interpretation in the 
$(m_0,m_{1/2})$ plane risks to heavyly overconstrain SUSY, as many 
possible mass patterns and signatures are not covered in the CMSSM. 
The same problem actually arises in any analysis that assumes a particular 
SUSY breaking scheme. 

In this Note we therefore introduce a different approach, which uses only 
minimal assumptions on the underlying SUSY parameters. In particular we do 
not rely on a particular SUSY breaking scheme. Instead, we use a 19-dimensional 
parametrization of the general MSSM, with parameters defined at the SUSY scale 
(by convention the geometric mean of the two stop masses), 
the so-called {\emph phenomenological MSSM} (pMSSM). 
We here demonstrate the feasibility of our approach by applying it to 
the 2010 CMS data-set corresponding to 35~pb$^{-1}$ of integrated luminosity.  
Using profile likelihoods, we combine 
the dijet $\alpha_T$ analysis, the opposite-sign dilepton 
analysis and the same-sign dilepton analysis and derive constraints 
on the SUSY particles that imply as few simplifying assumptions as possible.

We start this Note with giving the motivation to go beyond CMSSM and work with 
a generic MSSM setup, which is followed by the definition and parameterization 
of the pMSSM. We then outline our analysis, giving details on the pMSSM points we have used, the detector simulation and the CMS analyses, and later describe the statistical method based on profile likelihoods used for coping with the 19-dimensional setup 
of our model. This is followed by the results and conclusions.

