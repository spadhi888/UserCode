\section{Introduction}
\label{sec:intro}

With the extremely successful performence of both machine and detectors
at $\sqrt{s}=7$~TeV and good prospects to go soon into higher energies, 
the LHC is finally opening the window to the Terascale. 
Importantly, new insights are expected from the 
LHC data, most of all in the mechanism of electroweak (EW) symmetry 
breaking and, related to this, in the nature of new physics Beyond 
the Standard Model (BSM) stabilizing the EW scale. 

A wealth of BSM theories has been put forth by the theoretical community 
 and it is now up to experiments to test which, if any, of these theories are correct. 
The arguably best motivated, but certainly the best studied, such BSM 
theory is supersymmetry, SUSY for short. Indeed, searches for SUSY are 
among the primary objectives of the CMS collaboration. Here note that 
SUSY is exceedingly popular not only for its theoretical beauties but 
also because SUSY phenomenology is extremely rich, 
%in fact is can mimic almost any other new physics scenario. 
leading to a large variety of possible new signals at the LHC. 

Over the years, it has become common practice to interpret collider results in terms of a strictly constrained parameterization of supersymmetry, namely the constrained minimal supersymmetric model (CMSSM).  Though this setup has its practical implications, it lacks a sound theoretical motivation.  In this work we would like to initiate an approach that uses a more generic, and theoretically more feasible expression of supersymmetry.  We introduce here  the phenomenological MSSM (pMSSM), studied recently by Berger et. al., which is a 19 parameter realization of SUSY defined at the SUSY scale $\sqrt{m_{\tilde{t}_1}m_{\tilde{t}_2}}$ as an acceptably generic scenario for interpreting the early LHC results of 35pb$^{-1}$.  We use results from the dijet $\alpha_T$ analysis, the opposite-sign dilepton analysis and the same-sign dilepton analysis to see the effects of these early observations on our knowledge on pMSSM.

We start the note with giving the motivation to go beyond CMSSM and to work with pMSSM, which is followed by the definition and parameterization of pMSSM.  We then outline our analysis, giving details on the pMSSM points we have used, the detector simulation and the CMS analyses, and later describe the statistical method based on profile likelihoods used for coping with the 19 dimensional setup of pMSSM.  This is followed by the results and conclusion.

