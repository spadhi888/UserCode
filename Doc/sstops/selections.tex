\section{Event Selection}
\label{sec:eventselection}

The event selection used is not optimized for any specific non-standard model 
scenario. It is based on small modifications to the baseline 
di-lepton event selection that we used in our same-sign published study~\cite{ssnote1, sspaper}. The 
only difference is that we now require both leptons to have the transverse 
momentum ($p_t$) $> 20$ GeV. A quick summary of the event selection is:

\begin{itemize}
\item We require a mixture of unprescaled single and double lepton triggers as mentioned in~\cite{ssnote1}.
 The combined trigger efficiency is $\sim 99.9 \pm 0.1$\% for di-lepton events that pass the event selection.
\item At least two isolated same sign leptons ($ee$, $e\mu$, and $\mu\mu$). 
\item Leptons must have $P_T > 20$ GeV, $|\eta|< 2.4$.
\item We consider L2L3 corrected particle flow Jets with $P_T > 30$ GeV and
	$|\eta|< 2.4$.
\item The scalar sum of the $P_T$ of all jets passing the requirements above should be $>$ 60 GeV.
\item At least two jets.
\item We remove di-lepton events with invariant mass $ < 5$ GeV.

\item Additional Z-Veto:
\begin{itemize}
      \item  we veto the candidate lepton, if an extra lepton in the event pairs with the candidate lepton
             to form a $Z$ within the mass range between $76 < m_{\ell\ell} $ (GeV) $< 106$. This requirement is 
             designed to reject $WZ$ events.
\end{itemize}
\item We require particle flow \met~$>$ 30(20) GeV for $ee,\mu\mu (e\mu)$ events. 
We find that both tcMET and pfMET lead to similar results.
\item We require that all three charge measurements from GSF, CTF and Supercluster Charge algorithms agree. The
Supercluster Charge is determined from the relative position of the supercluster with respect to the projected
track from the pixel seed.
\end{itemize}
\noindent More details on the selections can be found elsewhere~\cite{ssnote1, sspaper}.
