\section{Higher order cross sections}
\label{sec:results}
The NNLO cross sections computed with FEWZ for $W$ and $Z$ boson production 
are summarized in Table~\ref{tab:nnlo}-~\ref{tab:nnlo66}. The total PDF uncertainties are based on 
the respective general-purpose PDF sets from CTEQ, MSTW08 and NNPDF2.0. 
The combined uncertainties are then computed using:

\begin{equation}
   \Delta X_{total} =  \frac{1}{2} [Max(X_1 + \Delta X_1, X_2 + \Delta X_2, X_3 + \Delta X_3 ) - Min(X_1 - \Delta X_1, X_2 - \Delta X_2, X_3 - \Delta X_3) ]
\end{equation}

where $X_1, X_2, X_3$ are central values from  CTEQ6.6, MSTW08 and NNPDF2.0. The $\Delta X_i$'s are the
associated uncertainties with variation in their eigenvectors along $\pm$ directions.

\vspace{3mm}
\begin{table}[hbt]
\begin{center}
\renewcommand{\arraystretch}{1.2}
\begin{tabular}{|l|c|c|c|c|c|c|}\hline
Processes & Generator & Phase Space& Order & Final state & Cross-section (pb)& Error (pb) \\ 
 &  &  cuts & & & PDF = CTEQ6M & Scale, PDF \\ \hline
$W^+$ & FEWZ & - & NNLO & $W \rightarrow l \nu_l, l=e,\mu,\tau$ & 16670 & $\pm 114$, $\pm$ 843 \\ \hline
$W^-$ & FEWZ & - & NNLO & $W \rightarrow l \nu_l, l=e,\mu,\tau$ & 11379 & $\pm 146$, $\pm$ 759 \\ \hline
Total $W$ & FEWZ & - & NNLO & $W \rightarrow l \nu_l, l=e,\mu,\tau$ & 28049 & $\pm 186$, $\pm$ 1134 \\ \hline
$Z/\gamma^*$ & FEWZ & $m_{ll} > 20$ GeV & NNLO & $Z \rightarrow l^+l^-, l=e,\mu,\tau$ & 4486 & $\pm 111$, $\pm 220$ \\ \hline
$Z/\gamma^*$ & FEWZ & $m_{ll} > 50$ GeV & NNLO & $Z \rightarrow l^+l^-, l=e,\mu,\tau$ & 2906 & $\pm 55$, $\pm 111$ \\ \hline
\end{tabular} 
\caption{NNLO cross sections for $W$ and $Z$ bosons using CTEQ6M. The cross sections are computed for
exclusive decays to leptons. The final inclusive values for $W$ are obtained using appropriate 
branching fractions from the PDG~\cite{pdg}. \label{tab:nnlo}}
\end{center}
\end{table}


\vspace{3mm}
\begin{table}[hbt]
\begin{center}
\renewcommand{\arraystretch}{1.2}
\begin{tabular}{|l|c|c|c|c|c|c|}\hline
Processes & Generator & Phase Space& Order & Final state & Cross-section (pb)& Error (pb) \\ 
 &  &  cuts & & & PDF = CTEQ6.6 & Scale, PDF \\ \hline
$W^+$ & FEWZ & - & NNLO & $W \rightarrow l \nu_l, l=e,\mu,\tau$ & 17137 & $\pm 115$, $\pm$ 942 \\ \hline
$W^-$ & FEWZ & - & NNLO & $W \rightarrow l \nu_l, l=e,\mu,\tau$ & 11534 & $\pm 149$, $\pm$ 754 \\ \hline
Total $W$ & FEWZ & - & NNLO & $W \rightarrow l \nu_l, l=e,\mu,\tau$ & 28671 & $\pm 188$, $\pm$ 1307 \\ \hline
$Z/\gamma^*$ & FEWZ & $m_{ll} > 20$ GeV & NNLO & $Z \rightarrow l^+l^-, l=e,\mu,\tau$ & 4644 & $\pm 164$, $\pm 352$ \\ \hline
$Z/\gamma^*$ & FEWZ & $m_{ll} > 50$ GeV & NNLO & $Z \rightarrow l^+l^-, l=e,\mu,\tau$ & 3087 & $\pm 99$, $\pm 230$ \\ \hline
\end{tabular} 
\caption{NNLO cross sections for $W$ and $Z$ bosons using CTEQ6.6. The cross sections are computed for
exclusive decays to leptons. The final inclusive values for $W$ are obtained using appropriate 
branching fractions from the PDG~\cite{pdg}. \label{tab:nnlo66}}
\end{center}
\end{table}


The uncertainties due to scale variations contribute to $\approx 1, 2$\% for 
$W$ and $Z/\gamma^*$ decays. The PDF variations are $\approx X_1, X_2$\%, 
respectively. Table~\ref{tab:nlo} shows the LO and NLO cross sections for 
various SM processes. Processes with $c$ and $b$ quarks are treated using the 
massive-quark scheme.

\vspace{3mm}
\begin{table}[hbt]
\begin{center}
\renewcommand{\arraystretch}{1.2}
\begin{tabular}{|l|c|c|c|c|c|c|}\hline
Processes & Generator & Phase Space& Order & Final state & Cross-section (pb)& Error (pb) \\ 
 &  &  cuts & & & PDF = CTEQ6M & Scale, PDF \\ \hline
$t\bar{t}$ & MCFM & - & NLO & Inclusive & 154.5 & $\pm 20.1$, $\pm$ xx \\ \hline
$t^+$ (s-channel) & MCFM & - & NLO & Inclusive & 2.6 & $\pm 0.1$, $\pm$ xx \\ \hline
$t^-$ (s-channel) & MCFM & - & NLO & Inclusive & 1.4 & $\pm 0.1$, $\pm$ xx \\ \hline
Total $t$ (s-channel) & MCFM & - & NLO & Inclusive & 4.0 & $\pm 0.1$, $\pm$ xx \\ \hline
$t^+$ (t-channel) & MCFM & - & NLO & Inclusive & 41.7 & $\pm 1.2$, $\pm$ xx \\ \hline
$t^-$ (t-channel) & MCFM & - & NLO & Inclusive & 21.5 & $\pm 0.6$, $\pm$ xx \\ \hline
Total $t$ (t-channel) & MCFM & - & NLO & Inclusive & 63.2 & $\pm 1.3$, $\pm$ xx \\ \hline
$W^+ \bar{t}$ & MCFM & - & NLO & $\bar{t}$ as inclusive, $W\rightarrow l \nu_l$ & 5.3 & $\pm 0.6$, $\pm$ xx \\ \hline
$W^- t$ & MCFM & - & NLO & $t$ as inclusive, $W\rightarrow l \nu_l$ & 5.3 & $\pm 0.6$, $\pm$ xx \\ \hline
Total $tW$ & MCFM & - & NLO & $t$ as inclusive, $W\rightarrow l \nu_l$ & 10.6 & $\pm 0.8$, $\pm$ xx \\ \hline
$W^+ \bar{c}$ & MCFM & - & NLO & Inclusive & 1718 & $\pm 157$, $\pm$ xx \\ \hline
$W^- c$ & MCFM & - & NLO & Inclusive & 1910 & $\pm 164$, $\pm$ xx \\ \hline
Total $Wc$ & MCFM & - & NLO & Inclusive & 3628 & $\pm 227$, $\pm$ xx \\ \hline
$W^+ b\bar{b}$ & MCFM & - & LO & Inclusive & 22.1 & $\pm 4,4$, $\pm$ xx \\ \hline
$W^- b\bar{b}$ & MCFM & - & LO & Inclusive & 13.2 & $\pm 2.5$, $\pm$ xx \\ \hline
Total $Wb\bar{b}$ & MCFM & - & LO & Inclusive & 35.3 & $\pm 5.1$, $\pm$ xx \\ \hline
$Z/\gamma^* b\bar{b}$ & MCFM & $m_{ll} > 20$ GeV & LO & Inclusive & 67.3 & $\pm 18.8$, $\pm$ xx \\ \hline
$W^+W^-$ & MCFM & - & NLO & Inclusive & 43 & $\pm 1.5$, $\pm$ xx \\ \hline
$W^+Z/\gamma^*$ & MCFM & $m_{ll} > 40$ GeV & NLO & Inclusive & 11.8 & $\pm 0.6$, $\pm$ xx \\ \hline
$W^-Z/\gamma^*$ & MCFM & $m_{ll} > 40$ GeV & NLO & Inclusive & 6.4 & $\pm 0.4$, $\pm$ xx \\ \hline
Total $WZ/\gamma^*$ & MCFM & $m_{ll} > 40$ GeV & NLO & Inclusive & 18.2 & $\pm 0.7$, $\pm$ xx \\ \hline
$Z/\gamma^*Z/\gamma^*$ & MCFM & $m_{ll} > 40$ GeV & NLO & Inclusive & 5.9 & $\pm 0.15$, $\pm$ xx \\ \hline
\end{tabular} 
\caption{LO and NLO cross sections for various SM processes. The cross sections are generally
computed for inclusive decays. For the single top $Wt$ contribution, the subtraction scheme
used in~\cite{Wtsubscheme} is used. \label{tab:nlo}}
\end{center}
\end{table}

The cross sections for $t\bar{t}$ are in very good agreement with 
$\sigma^{NNLL}_{t\bar{t}} = 165 \pm 10$~pb using NNLL 
resummations~\cite{nnllttbar}, with a top mass of $173$~GeV. Similarly, the 
single top production in s-channel agrees well with the NNLL approximations 
studies~\cite{nnllschannel} within uncertainties 
$\sigma^{NNLL}_{Single top} = 4.6 \pm 0.06 \pm 0.13$~pb. The results presented 
here can serve to compute K-factors, which can be defined as the ratio 
N$^k$LO/LO for the analysis, keeping into accounts the comments in section~\ref{kf}.


\vspace{3mm}
\begin{table}[hbt]
\begin{center}
\renewcommand{\arraystretch}{1.2}
 {\footnotesize
\begin{tabular}{|l|c|c|c|c|c|c|}\hline
Processes & Phase Space & Final state & CTEQ6M  & CTEQ6.6 & MSTW08NLO 68\% C.L & NNPDF2.0  \\ 
 & cuts & & $\sigma \pm$ PDF $\pm$ Scale & $\sigma \pm$ PDF $\pm$ Scale & $ \sigma \pm$ PDF $\pm$ Scale &  $\sigma \pm$ PDF $\pm$ Scale \\ \hline
$t\bar{t}$ & - & Inclusive & $154.5 \pm xx ^{+18.5}_{-20.1}$ & $150.5 \pm xx ^{+17.9}_{-19.5}$ & $162 \pm xx ^{+19.9}_{-21.7}$ & $163 \pm xx ^{+20.2}_{-20.7}$  \\ \hline
$\bar{t}$ (s-channel) & - & Inclusive & $1.4 \pm xx ^{+0.03}_{-0.03}$ & $1.4 \pm xx ^{+0.03}_{-0.03}$ & $1.5 \pm xx ^{+0.04}_{-0.03}$ & $1.4 \pm xx ^{+0.03}_{-0.02}$ \\ \hline
$t$ (s-channel) & - & Inclusive & $2.6 \pm xx ^{+0.06}_{-0.05}$ & $2.6 \pm xx ^{+0.08}_{-0.05}$ & $2.7 \pm xx ^{+0.07}_{-0.05}$ & $2.7 \pm xx ^{+0.07}_{-0.05}$ \\ \hline
Total $t$ (s-channel) & - & Inclusive & & & & \\ \hline
$\bar{t}$ (t-channel) & - & Inclusive & $21.6 \pm xx ^{+0.51}_{-0.45}$ & $21.3 \pm xx ^{+0.58}_{-0.42}$ & $22.1 \pm xx ^{+0.67}_{-0.41}$ & $23.0 \pm xx ^{+0.79}_{-0.30}$  \\ \hline
$t$ (t-channel) & - & Inclusive & $41.5 \pm xx ^{+1.23}_{-0.70}$ & $40.8 \pm xx ^{+1.29}_{-0.52}$ & $41.4 \pm xx ^{+1.30}_{-0.51}$ & $43.4 \pm xx ^{+1.56}_{-0.50}$  \\ \hline
Total $t$ (t-channel) & - & Inclusive & & & & \\ \hline
$W^+ \bar{t}$ & - & $\bar{t}$ as Incl., $W\rightarrow l \nu_l$ & $5.3 \pm xx ^{+0.34}_{-0.56}$ & $5.1 \pm xx ^{+0.33}_{-0.54}$ & $5.5 \pm xx ^{+0.35}_{-0.59}$ & $5.9 \pm xx ^{+0.37}_{-0.60}$ \\ \hline
$W^- t$ & - & $t$ as Incl., $W\rightarrow l \nu_l$ & $5.3 \pm xx ^{+0.34}_{-0.56}$ & $5.1 \pm xx ^{+0.33}_{-0.55}$ & $5.5 \pm xx ^{+0.36}_{-0.59}$ & $5.9 \pm xx ^{+0.38}_{-0.58}$ \\ \hline
Total $tW$ & - & $t$ as Incl., $W\rightarrow l \nu_l$ & & & & \\ \hline
$W^+ \bar{c}$ & - & Inclusive & $1736 \pm xx ^{+62.0}_{-65.3}$ & $1969 \pm xx ^{+79.1}_{-86.1}$ & $1823 \pm xx ^{+87.8}_{-77.8}$ & $1661 \pm xx ^{+59.8}_{-44.1}$ \\ \hline
$W^- c$ & - & Inclusive & $1889 \pm xx ^{+108.6}_{-58.9}$ & $2132 \pm xx ^{+126.1}_{-103.9}$ & $2066 \pm xx ^{+113.3}_{-96.2}$ & $1914 \pm xx ^{+88.1}_{-73.4}$ \\ \hline
Total $Wc$ & - & Inclusive & & & &  \\ \hline
$W^+ b\bar{b}$ & - & Inclusive, LO & $21.8 \pm xx ^{+4.2}_{-3.6}$ & $22.2 \pm xx ^{+4.6}_{-3.4}$ & $23.2 \pm xx ^{+4.9}_{-3.8}$ & $22.4 \pm xx ^{+4.5}_{-3.3}$ \\ \hline
$W^- b\bar{b}$ & - & Inclusive, LO & $13.0 \pm xx ^{+2.6}_{-1.9}$ & $13.3 \pm xx ^{+2.6}_{-2.1}$ & $14.4 \pm xx ^{+2.9}_{-2.3}$ & $13.5 \pm xx ^{+2.6}_{-2.0}$ \\ \hline
Total $Wb\bar{b}$ & - & Inclusive, LO & & & &  \\ \hline
$Z/\gamma^* b\bar{b}$ & $m_{ll} > 20$ GeV & Inclusive, LO & $68.0 \pm xx ^{+18.2}_{-13.9}$ & $66.7 \pm xx ^{+18.4}_{-13.7}$ & $71.6 \pm xx ^{+20.0}_{-14.9}$ & $71.1 \pm xx ^{+20.3}_{-14.9}$ \\ \hline
$W^+W^-$ & - & Inclusive & $43.0 \pm xx ^{+1.5}_{-1.2}$ & $43.5 \pm xx ^{+1.6}_{-1.0}$ & $45.1 \pm xx ^{+1.5}_{-1.2}$ & $43.8 \pm xx ^{+1.5}_{-1.1}$ \\ \hline
$W^+Z/\gamma^*$ & $m_{ll} > 40$ GeV & Inclusive & $11.8 \pm xx ^{+0.6}_{-0.5}$ & $11.9 \pm xx ^{+0.6}_{-0.5}$ & $12.1 \pm xx ^{+0.6}_{-0.5}$ & $12.0 \pm xx ^{+0.7}_{-0.4}$ \\ \hline
$W^-Z/\gamma^*$ & $m_{ll} > 40$ GeV & Inclusive & $6.4 \pm xx ^{+0.3}_{-0.3}$ & $6.4 \pm xx ^{+0.3}_{-0.3}$ & $6.9 \pm xx ^{+0.3}_{-0.3}$ & $6.6 \pm xx ^{+0.4}_{-0.3}$ \\ \hline
Total $WZ/\gamma^*$ & $m_{ll} > 40$ GeV & Inclusive & & & &  \\ \hline
$Z/\gamma^*Z/\gamma^*$ & $m_{ll} > 40$ GeV & Inclusive & $5.9 \pm xx ^{+0.2}_{-0.1}$ & $6.0 \pm xx ^{+0.2}_{-0.1}$ & $6.2 \pm xx ^{+0.2}_{-0.1}$ & $6.0 \pm xx ^{+0.1}_{-0.1}$ \\ \hline
\end{tabular} }
\caption{LO and NLO cross sections for various SM processes. The cross sections are generally
computed for inclusive decays. For the single top $Wt$ contribution, the subtraction scheme
used in~\cite{Wtsubscheme} is used. \label{tab:nlo1}}
\end{center}
\end{table}
