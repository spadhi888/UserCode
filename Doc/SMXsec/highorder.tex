\section{Higher order cross sections}
\label{sec:results}
The NNLO cross section computed for $W$ and $Z$ bosons are summerized in 
Table~\ref{tab:nnlo}. The PDF uncertainties are based on general-purpose 
CTEQ6M along with 40 eigenvector sets~\cite{cteq6m}.

\vspace{3mm}
\begin{table}[hbt]
\begin{center}
\renewcommand{\arraystretch}{1.2}
\begin{tabular}{|l|c|c|c|c|c|c|}\hline
Processes & Generator & Phase Space& Order & Final state & Cross-section (pb)& Error (pb) \\ 
 &  &  cuts & & & PDF = CTEQ6M & Scale, PDF \\ \hline
$W^+$ & FEWZ & - & NNLO & $W \rightarrow l \nu_l, l=e,\mu,\tau$ & 16670 & $\pm 114$, $\pm$ 843 \\ \hline
$W^-$ & FEWZ & - & NNLO & $W \rightarrow l \nu_l, l=e,\mu,\tau$ & 11379 & $\pm 146$, $\pm$ 759 \\ \hline
Total $W$ & FEWZ & - & NNLO & $W \rightarrow l \nu_l, l=e,\mu,\tau$ & 28049 & $\pm 186$, $\pm$ 1134 \\ \hline
$Z/\gamma^*$ & FEWZ & $m_{ll} > 20$ GeV & NNLO & Exclusive $Z \rightarrow e^+e^-$ & 1495 & $\pm 37$, $\pm 74$ \\ \hline
$Z/\gamma^*$ & FEWZ & $m_{ll} > 50$ GeV & NNLO & Exclusive $Z \rightarrow e^+e^-$ & 969 & $\pm 19$, $\pm 37$ \\ \hline
\end{tabular} 
\caption{NNLO cross sections for $W$ and $Z$ bosons. The cross sections are computed for
exclusive decays to leptons. The final inclusive values for $W$ are obtained using appropriate 
branching fractions from the PDG~\cite{pdg}. \label{tab:nnlo}}
\end{center}
\end{table}

The uncertainties due to scale variations contribute to $\approx 1, 2$\% for 
$W$ and $Z\gamma^*$ decays. The PDF variations are $\approx 4, 5$\%, 
respectively. Table~\ref{tab:nlo} shows the LO and NLO cross sections for 
various SM processes. Processes with $c$ and $b$ quarks are treated using the 
massive-quark scheme.

\vspace{3mm}
\begin{table}[hbt]
\begin{center}
\renewcommand{\arraystretch}{1.2}
\begin{tabular}{|l|c|c|c|c|c|c|}\hline
Processes & Generator & Phase Space& Order & Final state & Cross-section (pb)& Error (pb) \\ 
 &  &  cuts & & & PDF = CTEQ6M & Scale, PDF \\ \hline
$t\bar{t}$ & MCFM & - & NLO & Inclusive & 154.5 & $\pm 20.1$, $\pm$ xx \\ \hline
$t^+$ s-channel & MCFM & - & NLO & Inclusive & 2.6 & $\pm 0.1$, $\pm$ xx \\ \hline
$t^-$ s-channel & MCFM & - & NLO & Inclusive & 1.4 & $\pm 0.1$, $\pm$ xx \\ \hline
Total $t$ s-channel & MCFM & - & NLO & Inclusive & 4.0 & $\pm 0.1$, $\pm$ xx \\ \hline
$t^+$ t-channel & MCFM & - & NLO & Inclusive & 41.7 & $\pm 1.2$, $\pm$ xx \\ \hline
$t^-$ t-channel & MCFM & - & NLO & Inclusive & 21.5 & $\pm 0.6$, $\pm$ xx \\ \hline
Total $t$ t-channel & MCFM & - & NLO & Inclusive & 63.2 & $\pm 1.3$, $\pm$ xx \\ \hline
$W^+ \bar{t}$ & MCFM & - & NLO & Inclusive & 5.3 & $\pm 0.6$, $\pm$ xx \\ \hline
$W^- t$ & MCFM & - & NLO & Inclusive & 5.3 & $\pm 0.6$, $\pm$ xx \\ \hline
Total $tW$ & MCFM & - & NLO & Inclusive & 10.6 & $\pm 0.8$, $\pm$ xx \\ \hline
$W^+ \bar{c}$ & MCFM & - & NLO & Inclusive & 1718 & $\pm 157$, $\pm$ xx \\ \hline
$W^- c$ & MCFM & - & NLO & Inclusive & 1910 & $\pm 164$, $\pm$ xx \\ \hline
Total $Wc$ & MCFM & - & NLO & Inclusive & 3628 & $\pm 227$, $\pm$ xx \\ \hline
$W^+ b\bar{b}$ & MCFM & - & LO & Inclusive & 22.1 & $\pm 4,4$, $\pm$ xx \\ \hline
$W^- b\bar{b}$ & MCFM & - & LO & Inclusive & 13.2 & $\pm 2.5$, $\pm$ xx \\ \hline
Total $Wb\bar{b}$ & MCFM & - & LO & Inclusive & 35.3 & $\pm 5.1$, $\pm$ xx \\ \hline
$Z/\gamma^* b\bar{b}$ & MCFM & $m_{ll} > 20$ GeV & LO & Inclusive & 67.3 & $\pm 18.8$, $\pm$ xx \\ \hline
$W^+W^-$ & MCFM & - & NLO & Inclusive & 43 & $\pm 1.5$, $\pm$ xx \\ \hline
$W^+Z\gamma^*$ & MCFM & $m_{ll} > 40$ GeV & NLO & Inclusive & 11.8 & $\pm 0.6$, $\pm$ xx \\ \hline
$W^-Z\gamma^*$ & MCFM & $m_{ll} > 40$ GeV & NLO & Inclusive & 6.4 & $\pm 0.4$, $\pm$ xx \\ \hline
Total $WZ\gamma^*$ & MCFM & $m_{ll} > 40$ GeV & NLO & Inclusive & 18.2 & $\pm 0.7$, $\pm$ xx \\ \hline
$Z/\gamma^*Z/\gamma^*$ & MCFM & $m_{ll} > 40$ GeV & NLO & Inclusive & 5.9 & $\pm 0.15$, $\pm$ xx \\ \hline
\end{tabular} 
\caption{LO and NLO cross sections for various SM processes. The cross sections are generally
computed for inclusive decays. \label{tab:nlo}}
\end{center}
\end{table}

The cross sections for $t\bar{t}$ are in very good agreement with 
$\sigma^{NNLL}_{t\bar{t}} = 165 \pm 10$~pb using NNLL 
resummations~\cite{nnllttbar}, with a top mass of $173$~GeV. Similarly, the 
single top production in s-channel agrees well with the NNLL approximations 
studies~\cite{nnllschannel} within uncertainties 
$\sigma^{NNLL}_{Single top} = 4.6 \pm 0.06 \pm 0.13$~pb. The results presented 
here can serve to compute K-factors, which can be defined as the ratio 
N$^k$LO/LO for the analysis.
