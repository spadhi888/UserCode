\documentclass{cmspaper_pdf}
\usepackage{graphicx}

\begin{document}

%==============================================================================
% title page for few authors

\begin{titlepage}

  % select one of the following and type in the proper number:
  \cmsnote{2010/XXX}
  % \analysisnote{2005/000}
  % \internalnote{2007/007}
  % \conferencereport{2005/000}
  \date{21 June 2010}

  \title{Higher Order Standard Model \\
         cross-sections at 7 TeV}

  \begin{Authlist}
%    The CMS Collaboration
     Juan Alcaraz
    \Instfoot{cnrs}{CIEMAT, Madrid}
     Roberto Chierici
    \Instfoot{cnrs}{Institut de Physique Nucl\'eaire de Lyon}
    Fabio Cossutti 
    \Instfoot{infn}{Sezione di Trieste, INFN}
    Guillelmo G\'omez-Ceballos 
    \Instfoot{mit}{Massachusetts Institute of Technology}
    Sanjay Padhi
    \Instfoot{ucsd}{University of California, San Diego}
    Fabian Stoeckli
    \Instfoot{cern}{CERN}
    Silvano Tosi
    \Instfoot{ipn}{Institut de Physique Nucl\'eaire de Lyon}

  \end{Authlist}

  \author{\bf(On behalf of the CMS Collaboration)}

  \begin{abstract}
This study summarizes some of the higher order Standard Model (SM) cross sections using the latest 
available calculations for proton-proton collisions with 7 TeV centre-of-mass 
energy. The cross-section calculations are made choosing scales and 
parton distribution functions (PDFs) which are widely used in the CMS Collaboration for Monte 
Carlo simulations. The scale and PDF uncertainties are provided 
for these choices. Cross-sections using other higher order PDFs are also 
outlined along with their uncertainties.
  \end{abstract}

\end{titlepage}

\setcounter{page}{2}%

%==============================================================================
%
\section{Introduction}
\label{sec:intro}

After the successful operation of the Large Hadron Collider (LHC) and the CMS detector
in 2010 and 2011, and with good prospects for the future, the LHC is now ready to shed light on a number 
of open questions in Particle Physics 
such as the mechanism of electroweak (EW) symmetry breaking, or the 
new physics, Beyond the Standard Model (BSM), that stabilizes the EW scale. 

A wealth of theories that extend the Standard Model have been put forth during the past decades. Supersymmetry (SUSY) is
arguably the best motivated BSM theory --- and certainly the most 
thoroughly studied. 
Indeed, searches for SUSY are among the primary objectives of the 
CMS experiment. SUSY is exceedingly popular not 
only for its theoretical beauty but also because SUSY phenomenology 
is extremely rich, 
%in fact is can mimic almost any other new physics scenario. 
leading to a large variety of possible new signals at the LHC. 
In spite of this, the majority of SUSY studies focus on a very special 
setup: the so-called Constrained Minimal Supersymmetric Standard Model (CMSSM). 
This was justified in the preparation for discoveries as the CMSSM, 
having just a handful of new parameters, is very predicive. However, 
the simplifying assumption of universality at the GUT scale lacks a sound 
theoretical motivation. Consequently, the CMSSM should be regarded as a showcase 
model. When it comes to interpreting experimental results, it is reasonable and interesting to do this within the CMSSM because it 
provides (to some degree) an easy way to show performances, 
compare limits or reaches, etc. However, the interpretation of experimental results in the 
$(m_0,m_{1/2})$ plane risks imposing unwarranted constraints on SUSY, as many 
mass patterns and signatures that are possible a priori are not covered in the CMSSM. 
The same problem arises in any analysis that assumes a particular 
SUSY breaking scheme. 

In this document, we therefore introduce a different approach, which uses only 
minimal assumptions on the underlying SUSY parameters. In particular, given the absence of experimental guidance, we choose
not to rely on a particular SUSY breaking scheme.
Instead, we use a 19-dimensional 
parametrization of the MSSM, called the \emph{phenomenological MSSM} (pMSSM),
with parameters defined not at the GUT scale but instead at the SUSY scale 
(by convention the geometric mean of the two stop masses).
We demonstrate the feasibility of our approach by applying it to 
the 2011 CMS data-set corresponding to 1~fb$^{-1}$ of integrated luminosity.  
Using profile likelihoods, we combine 
the dijet $\alpha_T$ analysis, the opposite-sign dilepton 
analysis and the same-sign dilepton analysis and derive constraints 
on the SUSY particles with as few simplifying assumptions as possible.
Results from other SUSY analyses in CMS will be added as soon as they become available.

We first give the motivation to go beyond the CMSSM and work in 
a generic MSSM setup. After this, the pMSSM and its parametrization is defined. 
We then outline our analysis, giving details on the pMSSM points we have used, 
the detector simulation and the CMS analyses, and describe the statistical method based on 
profile likelihoods used for coping with the 19-dimensional model. Finally, we discuss our results and summarize our conclusions.


\section{Normalization factors, scale and PDFs}
\label{sec:normalization}
As a general rule, the highest-order available calculation should be used when 
calculating cross-sections along with dependencies on kinematics. These predictions
may be used to normalize or reweight the Monte Carlo distributions in the analyses.
In doing so, generator level cuts which may have been used for a given Monte Carlo 
production should be taken into account also in the calculation of the K-factors. 
This requires that, for instance, the 
reference leading order (LO) parton shower based MC and the next-to-leading order 
(NLO) calculation should use as much as possible the same cuts, and similar PDFs. 
Applying an inclusive K-factor to a (LO) Monte Carlo generation implicitly assumes
that the change in acceptance introduced by the analysis is not very sensitive 
to higher order QCD effects. This is an aspect that should be checked carefully
by those analyses where such a sensitivity may be present. Alternatives to a
constant K-factor are an event reweighting, function of some kinematic variables, 
or a determination of an {\it a posteriori} K-factor, if the same acceptance cuts 
of the analysis can be reproduced at parton level for the higher order calculation.

Other inputs that should be made uniform between leading and higher order 
calculations are the normalization $\mu_R$ and factorization $\mu_F$ scales,
as well as the strong coupling constant and the PDFs. 
Additionally, the order of the PDFs used should match with the order of the 
matrix-element calculations in the ratio for the K-factors, with the exception 
for next-to-next-leading order (NNLO) calculations, for which one can only 
consider NLO PDFs. 

\subsection{Scale uncertainties}
\label{kf}

The calculation of cross-sections in a given order in perturbation theory 
implies a dependence on both renormalization ($\mu_R$) and factorization 
($\mu_F$) scales. These are typically considered to be the same as the central 
value ($\mu_0$) of the scale.  For estimating the scale uncertainty, the scale 
choices are varied in the units of $\mu_0$. Although $\mu_R$ and $\mu_F$ can 
be varied independently, in this study we vary by the same units at the same 
time. The uncertainty on the cross section given by the scale choices is
then conventionally determined by a variation in the range
$1/2 \mu_0 < \mu_R, \mu_F < 2\mu_0$. 

\subsection{PDFs}
In general the most recent PDF sets should be used for cross section and 
acceptance calculations. If an analysis acceptance is studied using 
PYTHIA~\cite{Pythia} or HERWIG~\cite{Herwig}, the LO PDF (CTEQ6M~\cite{cteq6m} 
used in CMS simulations) should be used as a central value. However, the 
uncertainties on cross sections, and hence the uncertainties on acceptance, are 
computed with respect to the nominal choice at higher orders. %ROB confused: is that what you meant ???
We compute the PDF uncertainties using the prescription provided by the CTEQ 
Collaboration~\cite{cteq6m}. 

For a given central choice of scale and PDF, we estimate the uncertainties 
based on $N$ PDF sets of eigenvectors for an observable $X$. We use $2$ PDF 
sets for each of the $N$ eigenvectors, along the $\pm$ directions respectively. 
The uncertainty due to the PDFs is then defined as:

\begin{equation}
	\Delta X^{\pm} = \sqrt{\Sigma_i [max(X_i^\pm-X_0,X_i^\mp-X_0,0)]^2 }
\end{equation}

where $X_i^+$ and $X_i^-$ are the values of $X$ computed from the two PDF sets 
along $\pm$ direction of the $i-$th eigenvector, and $X_0$ the central value. 
The additional statistical 
uncertainties due to limited MC statistics can be evaluated by reweighting the 
MC events as a function of the parton flavours $q_1$ and $q_2$, parton momenta 
$x_1, x_2$ as well as $\mu_F$. Finally, it should be noted that if the MC
simulations are produced using a given LO PDF with a pre-determined choice of 
%ROB what do the following 3 lines mean ? 
$\alpha_s$, and it is difficult to factorize the dependence; thus the residual 
dependence on $\alpha_s$ can be estimated by reweighting it.

\section{Central values and choices for generator parameters}
\label{sec:assumptions}

We study the cross sections using the most appropriate and latest available calculations. FEWZ~\cite{fewz} is used 
for the computation of NNLO cross sections involving $W$ and $Z$ bosons, while we use MCFM 5.8~\cite{mcfm} for the rest 
of the SM processes at LO and NLO in perturbation theory. The study is performed for proton-proton collisions at a 
centre-of-mass energy of 7 TeV. The input parameters are the same used
for the nominal MC simulation in CMS.

Some of the parameter settings in accordence with the PDG~\cite{pdg} recommendation for performing the calculations 
are given in Table~\ref{tab:input_params}.

\vspace{0.9mm}
\begin{table}[hbt]
\begin{center}
\renewcommand{\arraystretch}{1.2}
\begin{tabular}{|l|c|}\hline
Input parameters & Values for Central Choice \\ \hline
PDF Set & CTEQ6M \\
$W$ boson mass & 80.398 GeV \\ \hline	
$W$ boson Width & 2.141 GeV \\ \hline	
$Z$ boson mass & 91.1876 GeV \\ \hline	
$Z$ boson Width & 2.4952 GeV \\ \hline	
$t$ quark mass & 172.5 GeV \\ \hline	
$b$ quark mass & 4.8 GeV \\ \hline	
$c$ quark mass & 1.27 GeV \\ \hline	
fine-structure constant &  0.007297352 \\ \hline	
\end{tabular} 
\caption{Input parameters used for obtaining the central value for various SM processes.\label{tab:input_params}}
\end{center}
\end{table}

--- Need to check if any other jet choices for MCFM are needed ------------

\section{Higher order cross sections}
\label{sec:results}
The NNLO cross sections computed with FEWZ for $W$ and $Z$ boson production 
are summarized in 
Table~\ref{tab:nnlo}. The PDF uncertainties are based on general-purpose 
CTEQ6M along with 40 eigenvector sets~\cite{cteq6m}.

\vspace{3mm}
\begin{table}[hbt]
\begin{center}
\renewcommand{\arraystretch}{1.2}
\begin{tabular}{|l|c|c|c|c|c|c|}\hline
Processes & Generator & Phase Space& Order & Final state & Cross-section (pb)& Error (pb) \\ 
 &  &  cuts & & & PDF = CTEQ6M & Scale, PDF \\ \hline
$W^+$ & FEWZ & - & NNLO & $W \rightarrow l \nu_l, l=e,\mu,\tau$ & 16670 & $\pm 114$, $\pm$ 843 \\ \hline
$W^-$ & FEWZ & - & NNLO & $W \rightarrow l \nu_l, l=e,\mu,\tau$ & 11379 & $\pm 146$, $\pm$ 759 \\ \hline
Total $W$ & FEWZ & - & NNLO & $W \rightarrow l \nu_l, l=e,\mu,\tau$ & 28049 & $\pm 186$, $\pm$ 1134 \\ \hline
$Z/\gamma^*$ & FEWZ & $m_{ll} > 20$ GeV & NNLO & Exclusive $Z \rightarrow e^+e^-$ & 1495 & $\pm 37$, $\pm 74$ \\ \hline
$Z/\gamma^*$ & FEWZ & $m_{ll} > 50$ GeV & NNLO & Exclusive $Z \rightarrow e^+e^-$ & 969 & $\pm 19$, $\pm 37$ \\ \hline
\end{tabular} 
\caption{NNLO cross sections for $W$ and $Z$ bosons. The cross sections are computed for
exclusive decays to leptons. The final inclusive values for $W$ are obtained using appropriate 
branching fractions from the PDG~\cite{pdg}. \label{tab:nnlo}}
\end{center}
\end{table}

The uncertainties due to scale variations contribute to $\approx 1, 2$\% for 
$W$ and $Z\gamma^*$ decays. The PDF variations are $\approx 4, 5$\%, 
respectively. Table~\ref{tab:nlo} shows the LO and NLO cross sections for 
various SM processes. Processes with $c$ and $b$ quarks are treated using the 
massive-quark scheme.

\vspace{3mm}
\begin{table}[hbt]
\begin{center}
\renewcommand{\arraystretch}{1.2}
\begin{tabular}{|l|c|c|c|c|c|c|}\hline
Processes & Generator & Phase Space& Order & Final state & Cross-section (pb)& Error (pb) \\ 
 &  &  cuts & & & PDF = CTEQ6M & Scale, PDF \\ \hline
$t\bar{t}$ & MCFM & - & NLO & Inclusive & 154.5 & $\pm 20.1$, $\pm$ xx \\ \hline
$t^+$ s-channel & MCFM & - & NLO & Inclusive & 2.6 & $\pm 0.1$, $\pm$ xx \\ \hline
$t^-$ s-channel & MCFM & - & NLO & Inclusive & 1.4 & $\pm 0.1$, $\pm$ xx \\ \hline
Total $t$ s-channel & MCFM & - & NLO & Inclusive & 4.0 & $\pm 0.1$, $\pm$ xx \\ \hline
$t^+$ t-channel & MCFM & - & NLO & Inclusive & 41.7 & $\pm 1.2$, $\pm$ xx \\ \hline
$t^-$ t-channel & MCFM & - & NLO & Inclusive & 21.5 & $\pm 0.6$, $\pm$ xx \\ \hline
Total $t$ t-channel & MCFM & - & NLO & Inclusive & 63.2 & $\pm 1.3$, $\pm$ xx \\ \hline
$W^+ \bar{t}$ & MCFM & - & NLO & Inclusive & 5.3 & $\pm 0.6$, $\pm$ xx \\ \hline
$W^- t$ & MCFM & - & NLO & Inclusive & 5.3 & $\pm 0.6$, $\pm$ xx \\ \hline
Total $tW$ & MCFM & - & NLO & Inclusive & 10.6 & $\pm 0.8$, $\pm$ xx \\ \hline
$W^+ \bar{c}$ & MCFM & - & NLO & Inclusive & 1718 & $\pm 157$, $\pm$ xx \\ \hline
$W^- c$ & MCFM & - & NLO & Inclusive & 1910 & $\pm 164$, $\pm$ xx \\ \hline
Total $Wc$ & MCFM & - & NLO & Inclusive & 3628 & $\pm 227$, $\pm$ xx \\ \hline
$W^+ b\bar{b}$ & MCFM & - & LO & Inclusive & 22.1 & $\pm 4,4$, $\pm$ xx \\ \hline
$W^- b\bar{b}$ & MCFM & - & LO & Inclusive & 13.2 & $\pm 2.5$, $\pm$ xx \\ \hline
Total $Wb\bar{b}$ & MCFM & - & LO & Inclusive & 35.3 & $\pm 5.1$, $\pm$ xx \\ \hline
$Z/\gamma^* b\bar{b}$ & MCFM & $m_{ll} > 20$ GeV & LO & Inclusive & 67.3 & $\pm 18.8$, $\pm$ xx \\ \hline
$W^+W^-$ & MCFM & - & NLO & Inclusive & 43 & $\pm 1.5$, $\pm$ xx \\ \hline
$W^+Z\gamma^*$ & MCFM & $m_{ll} > 40$ GeV & NLO & Inclusive & 11.8 & $\pm 0.6$, $\pm$ xx \\ \hline
$W^-Z\gamma^*$ & MCFM & $m_{ll} > 40$ GeV & NLO & Inclusive & 6.4 & $\pm 0.4$, $\pm$ xx \\ \hline
Total $WZ\gamma^*$ & MCFM & $m_{ll} > 40$ GeV & NLO & Inclusive & 18.2 & $\pm 0.7$, $\pm$ xx \\ \hline
$Z/\gamma^*Z/\gamma^*$ & MCFM & $m_{ll} > 40$ GeV & NLO & Inclusive & 5.9 & $\pm 0.15$, $\pm$ xx \\ \hline
\end{tabular} 
\caption{LO and NLO cross sections for various SM processes. The cross sections are generally
computed for inclusive decays. \label{tab:nlo}}
\end{center}
\end{table}

The cross sections for $t\bar{t}$ are in very good agreement with 
$\sigma^{NNLL}_{t\bar{t}} = 165 \pm 10$~pb using NNLL 
resummations~\cite{nnllttbar}, with a top mass of $173$~GeV. Similarly, the 
single top production in s-channel agrees well with the NNLL approximations 
studies~\cite{nnllschannel} within uncertainties 
$\sigma^{NNLL}_{Single top} = 4.6 \pm 0.06 \pm 0.13$~pb. The results presented 
here can serve to compute K-factors, which can be defined as the ratio 
N$^k$LO/LO for the analysis, keeping into accounts the comments in section~\ref{kf}.

\section{Summary and Conclusions}
\label{sec:conclusion}
Calculations of higher-order cross sections for several SM processes in $pp$ collisions at 7 TeV are provided 
in this note. The cross sections are computed using the FEWZ and MCFM calculators for a given choice of parameters.
The dominant systematic errors on the cross sections are due to the uncertainities in the PDFs, which 
are typically of the order of ($y1-y2$) \% on the total cross sections. The scale uncertanities within 
the variation of $1/2 \mu_0 < \mu_R, \mu_F < 2\mu_0$ are found to be at ($x1-x2$) \% level. 
 


\begin{thebibliography}{9}

\bibitem{Pythia} \textbf{PYTHIA}: T. Sj\"ostrand, S. Ask, R. Corke, S. Mrenna, 
P. Skands, http://home.thep.lu.se/~torbjorn/Pythia.html.
\bibitem{Herwig} \textbf{HERWIG}: G. Corcella, I.G. Knowles, G. Marchesini, 
S. Moretti, K. Odagiri, P. Richardson, M.H. Seymour and B.R. Webber, 
JHEP 0101 (2001) 010.
\bibitem{cteq6m} J. Pumplin, D.R. Stump, J. Huston, H.L. Lai, 
Pavel M. Nadolsky and W.K. Tung, JHEP 0207:012,2002. 
\bibitem{fewz}http://www.hep.wisc.edu/~frankjp/FEWZ.html.
\bibitem{mcfm}http://mcfm.fnal.gov/.
\bibitem{pdg}http://pdg.lbl.gov/.
\bibitem{cteq66} Pavel M. Nadolsky et. al, Phys.Rev. D78:013004, 2008. 

\bibitem{Wtsubscheme} R.~K.~Ellis, D.~A.~Ross and A.~E.~Terrano, Nucl. Phys. {\bf B178} (1981) 421.

\bibitem{nnllttbar} Nikolaos Kidonakis, arXiv:0909.0037.
\bibitem{nnllschannel} Nikolaos Kidonakis, arXiv:1001.5034.

 
\end{thebibliography}

\end{document}
