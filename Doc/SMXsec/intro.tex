\section{Introduction}
\label{sec:intro}
The LHC has recently started delivering proton-proton collisions at a 
centre-of-mass energy $\sqrt{S}=7$ TeV. 
Physics analyses at the LHC often depend 
on various inputs from theory that are only known with limited accuracy. The 
predictions on process cross sections are a well known example; their 
accuracy depends on the order of perturbation theory of the calculation, the
parton distribution functions (PDFs) used, the factorization and regularization scales assumed, and the choice of Standard Model parameters.

Most often there is no unique prescription of what calculation and what input 
settings should be used in a given analysis when comparing to the data. 
This study makes use of certain conventions, suggested by the 
Monte Carlo simulations used in CMS, in performing the computations of
higher order cross sections, along with the determination of the associated
errors, and aims at establishing common reference values for most relevant
SM cross sections which analyses can refer to.
For reference, the term Monte Carlo in this note refers to those tools which generate complete events by adding parton showering, fragmentation, hadronization, {\it etc.} to simple, leading order (LO) partonic processes.

In this note, we provide a tabulation of inclusive cross section estimates for
many critical Standard Model processes and a prescription for how to apply them.
The note is organised as follows:
in Section~\ref{sec:normalization}, guidelines for the calculation of K-factors 
based on higher order cross sections, along with given scales and PDF 
uncertainties, are provided. In Section~\ref{sec:assumptions} the assumptions 
made for these calculations are given, followed by the results in 
Section~\ref{sec:results}. Finally, in Section~\ref{sec:conclusion}, the 
results are summarized.
