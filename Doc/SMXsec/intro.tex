\section{Introduction}
\label{sec:intro}
The LHC has recently started delivering proton-proton collisions at a centre-of-mass energy of 7 TeV. 
Physics analyses at the LHC frequently depend on various inputs from theory that are only known with
limited accuracy. The cross section calculations also depends on various orders of perturbation theory
as well as determination of PDFs. 

Most often there is no unique choice of what calculation along with the prescription should be used in 
a given analysis when comparing to the data. This study aims at establishing a convention as well 
as a certain set of choices based on inputs from Monte Carlo simulations, currently being used in the 
CMS collaboration. The higher order cross sections are then computed using these choices as well as 
uncertanities arising due to the assumptions.

In Section~\ref{sec:normalization}, guideline for the calculation of K-factors based on higher order cross sections
along with given scales and PDF uncertainties are provided. In Section~\ref{sec:assumptions}
the assumptions made for these calculations are given, followed by the results in Section~\ref{sec:results}
and finally, in Section~\ref{sec:conclusion} we summarize the results.  
