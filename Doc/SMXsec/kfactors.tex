\section{Normalization factors, scale and PDFs}
\label{sec:normalization}
As a general rule, the highest-order available calculation should be used when 
calculating cross-sections along with dependencies on kinematics. These predictions
may be used to normalize or reweight the Monte Carlo distributions in the analyses.
In doing so, generator level cuts which may have been used for a given Monte Carlo 
production should be taken into account also in the calculation of the K-factors. 
This requires that, for instance, the 
reference leading order (LO) parton shower based MC and the next-to-leading order 
(NLO) calculation should use as much as possible the same cuts, and similar PDFs. 
Applying an inclusive K-factor to a (LO) Monte Carlo generation implicitly assumes
that the change in acceptance introduced by the analysis is not very sensitive 
to higher order QCD effects. This is an aspect that should be checked carefully
by those analyses where such a sensitivity may be present. Alternatives to a
constant K-factor are an event reweighting, function of some kinematic variables, 
or a determination of an {\it a posteriori} K-factor, if the same acceptance cuts 
of the analysis can be reproduced at parton level for the higher order calculation.

Other inputs that should be made uniform between leading and higher order 
calculations are the normalization $\mu_R$ and factorization $\mu_F$ scales,
as well as the strong coupling constant and the PDFs. 
Additionally, the order of the PDFs used should match with the order of the 
matrix-element calculations in the ratio for the K-factors, with the exception 
for next-to-next-leading order (NNLO) calculations, for which one can only 
consider NLO PDFs. 

\subsection{Scale uncertainties}
\label{kf}

The calculation of cross-sections in a given order in perturbation theory 
implies a dependence on both renormalization ($\mu_R$) and factorization 
($\mu_F$) scales. These are typically considered to be the same as the central 
value ($\mu_0$) of the scale.  For estimating the scale uncertainty, the scale 
choices are varied in the units of $\mu_0$. Although $\mu_R$ and $\mu_F$ can 
be varied independently, in this study we vary by the same units at the same 
time. The uncertainty on the cross section given by the scale choices is
then conventionally determined by a variation in the range
$1/2 \mu_0 < \mu_R, \mu_F < 2\mu_0$. 

\subsection{PDFs}
In general the most recent PDF sets should be used for cross section and 
acceptance calculations. If an analysis acceptance is studied using 
PYTHIA~\cite{Pythia} or HERWIG~\cite{Herwig}, the LO PDF (CTEQ6M~\cite{cteq6m} 
used in CMS simulations) should be used as a central value. However, the 
uncertainties on cross sections, and hence the uncertainties on acceptance, are 
computed with respect to the nominal choice at higher orders. %ROB confused: is that what you meant ???
We compute the PDF uncertainties using the prescription provided by the CTEQ 
Collaboration~\cite{cteq6m}. 

For a given central choice of scale and PDF, we estimate the uncertainties 
based on $N$ PDF sets of eigenvectors for an observable $X$. We use $2$ PDF 
sets for each of the $N$ eigenvectors, along the $\pm$ directions respectively. 
The uncertainty due to the PDFs is then defined as:

\begin{equation}
	\Delta X = \frac{1}{2} \sqrt{\Sigma (X_i^+ - X_i^-)^2 }
\end{equation}

where $X_i^+$ and $X_i^-$ are the values of $X$ computed from the two PDF sets 
along $\pm$ direction of the $i-$th eigenvector. The additional statistical 
uncertainties due to limited MC statistics can be evaluated by reweighting the 
MC events as a function of the parton flavours $q_1$ and $q_2$, parton momenta 
$x_1, x_2$ as well as $\mu_F$. Finally, it should be noted that if the MC
simulations are produced using a given LO PDF with a pre-determined choice of 
%ROB what do the following 3 lines mean ? 
$\alpha_s$, and it is difficult to factorize the dependence; thus the residual 
dependence on $\alpha_s$ can be estimated by reweighting it.
