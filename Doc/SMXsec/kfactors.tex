\section{Normalization factors, Scale and PDFs}
\label{sec:normalization}

Normally as a general rule, the highest-order available calculation should be used when calculating cross-sections 
along with dependencies on kinematics. The generator level cuts for a given production should be taken into
account in the calculation of the K-factors. This requires that the leading order (LO) parton shower based MC and the 
next-to-leading order (NLO) calculation use the same or similar order PDF. If possible, the normalization $\mu_R$ and 
factorization scale $\mu_F$ should be taken into account for the given choice. Other ingradients such as strong 
coupling constants, as well as PDFs $\mu_R$ and $\mu_F$ needs to be similar between the leading and higer order calculations. 
Additionally, the order of the PDFs used should match with the order of the matrix-element calculations in the 
ratio for the K-factors, with the exception for next-to-next-leading order (NNLO), where additionally one has 
to take into account a NLO PDF. 

\subsection{Scale Uncertainties}

The calculation of cross-sections in a given order in perturbation theory implies a dependence on both 
renormalization ($\mu_R$) and factorization ($\mu_F$) scales. These are typically considered to be the same 
as the central value ($\mu_0$) of the scale.  For estimating the scale uncertainty, the scale choices are varied 
in the units of $\mu_0$. Although $\mu_R$ and $\mu_F$ can be varied independently, in this study
we vary by the same units at the same time. The uncertainity in the scale choice is then determined
by varying $1/2 \mu_0 < \mu_R, \mu_F < 2\mu_0$. 

\subsection{PDFs}

Generally, most recent PDF sets should be used for cross section and acceptance calculations. If in an analysis 
the acceptance is studied using PYTHIA~\cite{Pythia} or HERWIG~\cite{Herwig}, the LO PDF (CTEQ6M~\cite{cteq6m} used 
in CMS simulations) should be used as a central value. However, the errors on cross sections and hence the errors on 
acceptance are always computed with respect to the nominal choice. We compute the PDF
uncertanities using the prescription provided by the CTEQ Collaboration~\cite{cteq6m}. 

For a given central choice of scale and PDF, we estimate the errors based on $N$ PDF sets of 
eigenvectors for an observable $X$. We use $2$ PDF sets for each of the $N$ eigenvectors, 
along the $\pm$ directions respectively. The uncertanity due to the PDFs is then defined as:

\begin{equation}
	\Delta X = \frac{1}{2} \sqrt{\Sigma (X_i^+ - X_i^-)^2 }
\end{equation}

where $X_i^+$ and $X_i^-$ are the values of $X$ computed from the two PDF sets along $\pm$ direction of the 
$i-$th eigenvector. The additional statistical errors due to limited MC statistics can be evaluated by reweighting
the MC events as a function of the parton flavours $q_1$ and $q_2$, parton momenta $x_1, x_2$ as well as $\mu_F$.
Finally, it should be noted that if the MC simulations are produced using a given LO PDF with a pre-determined 
choice of $\alpha_s$, it is difficult to factorize the dependence, thus the residual dependence on $\alpha_s$ 
can be estimated by reweighting.
